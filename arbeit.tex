%%
%% Dokumentenart
%%
\NeedsTeXFormat{LaTeX2e}
\documentclass[
    a4paper,
    10pt,
    bibliography=totoc,
    twoside,
    openright,
    numbers=noenddot,
    headings=normal,
    DIV=9,
    BCOR=7mm
    ,parskip
    %,draft
]{scrbook}

%%
%% Unterstützung für deutsche Sprache, Umlaute etc.
%%
\usepackage[utf8]{inputenc}
\usepackage[T1]{fontenc}
\usepackage[english]{babel}
\usepackage[babel]{csquotes}

% BibLaTeX
\usepackage[
style=alphabetic,    % Zitierstil
isbn=true,                % ISBN nicht anzeigen, gleiches geht mit nahezu allen anderen Feldern
pagetracker=true,          % ebd. bei wiederholten Angaben (false=ausgeschaltet, page=Seite, spread=Doppelseite, true=automatisch)
maxbibnames=50,            % maximale Namen, die im Literaturverzeichnis angezeigt werden (ich wollte alle)
maxcitenames=3,            % maximale Namen, die im Text angezeigt werden, ab 4 wird u.a. nach den ersten Autor angezeigt
autocite=inline,           % regelt Aussehen für \autocite (inline=\parancite)
block=space,               % kleiner horizontaler Platz zwischen den Feldern
backref=false,              % Seiten anzeigen, auf denen die Referenz vorkommt
%backrefstyle=three+,       % fasst Seiten zusammen, z.B. S. 2f, 6ff, 7-10
date=short,                % Datumsformat
backend=bibtex,
hyperref=true
]{biblatex}
\setlength{\bibitemsep}{1em}     % Abstand zwischen den Literaturangaben
\setlength{\bibhang}{2em}        % Einzug nach jeweils erster Zeile

\bibliography{bibliography}  % Bibtex-Datei wird schon in der Preambel eingebunden


%%
%% Diverse Pakete
%%
\usepackage{fancyvrb}
\usepackage{ccicons} % Creative Commons Lizenzsymbole
\usepackage{lmodern} % verbessert die Schriftdarstellung in pdfs
\usepackage{enumitem} % ändern von listeneinstellungen mit \setlist
\setlist{nolistsep} % kein abstand zwischen items in listen, Befehl von enumitem
\usepackage{float} % use \begin{figure}{H} to really force a figure being placed here
\usepackage[section]{placeins} % floatbarrier at each new section, so every figure appears in its own section
\usepackage{tikz}
\usetikzlibrary{positioning,shapes,shapes.multipart,shadows,arrows,automata,calc,decorations.markings}
\usepackage{tikz-uml}

\usepackage{scrhack}
\usepackage{graphicx}
\usepackage{verbatim}
\usepackage{tabularx}
\usepackage{subfigure}
\usepackage{url}
\usepackage{xcolor}
\usepackage{amssymb}
\usepackage{amsmath}
\usepackage{amsthm}
\usepackage{setspace}
\usepackage{listings}
\usepackage{colortbl}
%\usepackage{showframe} % Seitenspiegel anzeigen
\usepackage{microtype}
\usepackage{hyperref} % muss letztes Paket in der Liste sein


%%
%% hier Namen etc. einsetzen
%%
\newcommand{\fullname}{Daniel Gall}
\newcommand{\email}{daniel.gall@uni-ulm.de}
\newcommand{\titel}{A Rule-Based Implementation of ACT-R Using Constraint Handling Rules}
\newcommand{\jahr}{2013}
\newcommand{\matnr}{645463}
\newcommand{\gutachterA}{Prof.\ Dr.\ Dr.\ Thom Frühwirth}
\newcommand{\gutachterB}{Prof.\ Dr.\ Slim Abdennadher}
\newcommand{\betreuer}{Prof.\ Dr.\ Dr.\ Thom Frühwirth}
\newcommand{\fakultaet}{Ingenieurwissenschaften\\und Informatik}
%\newcommand{\fakultaet}{Mathematik und\\Wirtschaftswissenschaften}
%\newcommand{\fakultaet}{Naturwissenschaften}
%\newcommand{\fakultaet}{Medizin}
\newcommand{\institut}{Institut für Programmiermethodik und Compilerbau}
\newcommand{\arbeit}{Masterarbeit}
%\newcommand{\arbeit}{Bachelorarbeit}



%% amsthm environment definitions
%%

\newtheorem{definition}{Definition}[chapter]
\newtheorem{example}{Example}[chapter]

%%
%% Setzt Autor und Titel in den Metadaten des erzeugten Dokumentes
%%
\pdfinfo{
    /Author (\fullname)
    /Title (\titel)
    /Producer (pdfeTex 3.14159-1.30.6-2.2)
    /Keywords ()
}
\hypersetup{
    pdftitle=\titel,
    pdfauthor=\fullname,
    pdfsubject={\arbeit},
    pdfproducer={pdfeTex 3.14159-1.30.6-2.2},
    colorlinks=false,
    pdfborder=0 0 0
}


%%
%% Tiefe, bis zu der Überschriften in das Inhaltsverzeichnis kommen
%%
\setcounter{tocdepth}{3}


%%
%% Verhindert überhängende Absatzteile
%%
\clubpenalty10000
\widowpenalty10000
\displaywidowpenalty=10000


%%
%% Eigene Farben
%%
\definecolor{Gray}{rgb}{0.80784, 0.86667, 0.90196} %dunkelblau
\definecolor{Lightgray}{rgb}{0.9176, 0.95, 0.95686} %hellblau
\definecolor{Akzent}{rgb}{0.6627, 0.63529, 0.55294} %akzentfarbe

\definecolor{Uulmin}{RGB}{163,38,56} % Uni Ulm, rot der Fakultät für Ingenieurwissenschaften und Informatik


%%
%% Einstellungen für Codelistings
%%
\lstset{
    language=Prolog,
    showstringspaces=false,
    frame=L,%TB,%single
    %backgroundcolor=\color{Lightgray},
    breaklines=true,
    breakatwhitespace=true,
    xleftmargin=20pt,
    numbers=left,
    aboveskip=\parskip,
    basicstyle=\ttfamily,
    numberstyle=\tiny,
    deletekeywords={ time },
    alsoletter= {<,=,>,:,-,@, |, \\, ? },
    morekeywords={ <=>, ==> , :-, @, |, \\, false, ?- },
    morecomment=[s]{\{}{\}},
    moredelim=[is][\textbf]{<<}{>>},
    moredelim=[is][\textit]{<<<}{>>>}
}


%%
%% Formatierung des Literaturverzeichnisses
%%
%\bibliographystyle{plaindin} % Nummern und alphabetisch sortiert
%\bibliographystyle{alphadin} % Buchstaben und sortiert
%\bibliographystyle{amsalpha}
%\bibliographystyle{abbrvdin} % Nummern und abgekürzte Namen
%\bibliographystyle{unsrtdin} % Nummern und unsortiert


%%
%% Eigene Makros
%%
%\newcommand{\FIXME}[1]{\colorbox{yellow}{\bf FIXME: #1}} % debug mode
\newcommand{\FIXME}[1]{} % deploy mode


%%
%% Liniendicke in Tabellen etc.
%%
\setlength{\arrayrulewidth}{0.1pt}


%%
%% Schriftarten
%%
\renewcommand{\sfdefault}{phv}
\renewcommand{\rmdefault}{phv}
\renewcommand{\ttdefault}{pcr}
\KOMAoptions{DIV=last}


%%
%% Seitenlayout
%%
\pagestyle{headings}


%%
%% Trennungsregeln
%%
\hyphenation{Sil-ben-trenn-ung}


%%
%% Schönere Bullets bei Aufzählungen
%%
\renewcommand{\labelitemi}{$\bullet$}
\renewcommand{\labelitemii}{$\circ$}
\renewcommand{\labelitemiii}{$\cdot$}


%%
%% Beginn des eigentlichen Dokumentes
%%
\begin{document}


%%
%% Vorspann
%%
\frontmatter


%%
%% Titelseite
%%
\thispagestyle{empty}
\begin{addmargin*}[4mm]{-32mm}
    % Logo und Wortmarke
    \includegraphics[height=1.8cm]{images/unilogo_bild}
    \hfill
    \includegraphics[height=1.8cm]{images/unilogo_wort}
    \vspace*{2.1em}

    % Briefkopf
    \footnotesize
    \textbf{Universität Ulm} \textbar ~89069 Ulm \textbar ~Germany
    \hfill
    \parbox[t]{42mm}{\bfseries Fakultät für\\\fakultaet\\\mdseries\institut}
    \vspace*{2cm}

    % Titel
    \parbox{140mm}{\bfseries \raggedright \huge \titel}

    % Untertitel
    {\arbeit{} an der Universität Ulm}
    \vspace*{4em}

    % Prüfer etc.
    \textbf{Vorgelegt von:}\\\fullname\\\email\\[2em]
    \textbf{Gutachter:}\\\gutachterA\\\gutachterB\\[2em]
    \textbf{Betreuer:}\\\betreuer\\[1.5em]
    \jahr
\end{addmargin*}


%%
%% Impressum
%%
\clearpage
\thispagestyle{empty}
{
    % vollständiger Titel
    \small \flushleft \enquote{\titel}\\
    Version of \today
    \vfill

    % Danksagung
    %\textsc{Danksagungen:}
    %\FIXME{Danksagungen}
    %\vspace{1cm}

    % Anerkennung
%     Bei der Erstellung dieser \arbeit{} wurde ausschließlich freie Software eingesetzt: \\
%     \begin{figure}[ht]
%         \centering
%         \subfigure{\includegraphics[height=1cm]{images/sw/ubuntu}}
%         \hspace{0.5cm}
%         \subfigure{\includegraphics[height=1cm]{images/sw/texmaker}}
%         \hspace{0.5cm}
%         \subfigure{\includegraphics[height=1cm]{images/sw/inkscape}}
%         \hspace{0.5cm}
%         \subfigure{\includegraphics[height=1cm]{images/sw/subversion}}
%         \hspace{0.5cm}
%         \subfigure{\includegraphics[height=1cm]{images/sw/jabref}}
%         \hspace{0.5cm}
%         \subfigure{\includegraphics[height=1cm]{images/sw/firefox}}
%         \hspace{0.5cm}
%         \subfigure{\includegraphics[height=1cm]{images/sw/evince}}
%     \end{figure}
%     \vspace{1cm}

    % Urheberrechtshinweis
    \copyright{} \jahr{} \fullname{}\\[0.5em]
    % Falls keine Lizenz gewünscht wird bitte den folgenden Text entfernen.
    % Die Lizenz erlaubt es zu nichtkommerziellen Zwecken die Arbeit zu
    % vervielfältigen und Kopien zu machen. Dabei muss aber immer der Autor
    % angegeben werden. Eine kommerzielle Verwertung ist für den Autor
    % weiter möglich.
    This work is licensed under a Creative Commons Attribution-NonCommercial-ShareAlike 3.0 License: \url{http://creativecommons.org/licenses/by-nc-sa/3.0/}\\
    \ccbyncsa\\
    \vspace{0.5cm}
    Typesetting: PDF-\LaTeXe{}\\
    Graphics created with: Ti\textit{k}Z\\
    Printed by: Kommunikations- und Informationszentrum (kiz), Ulm University
}

\setstretch{1.2} % was: 1.4
\renewcommand{\arraystretch}{1.2} % gleicher Zeilenabstand in Tabellen


\chapter*{Abstract}

\emph{Computational Cognitive Modeling} is a research field at the interface of computer science and the cognitive sciences. It enables researchers to build detailed cognitive models upon a cognitive architecture which provides some general assumptions of human cognition to simulate human behaviour. By conducting the same experiments with humans and an executable computational cognitive model, the plausibility of a model can be verified.

\emph{ACT-R} is a cognitive architecture which is widely used in the field of computational cognitive modeling. It is a production-rule system whose models are expressed by a set of declarative knowledge elements, the data, and a set of rules. The rules match the data and -- if applied -- have effects on the data. \emph{Constraint Handling Rules (CHR)} is a high-level rule-based formalism and programming language which offers a lot of analyzing tools and possibilities for its programs. 

This work first presents some fundamental aspects of ACT-R and then introduces a translation of ACT-R models to CHR programs. The implementation of ACT-R using Constraint Handling Rules shows that cognitive models can be expressed very elegantly in CHR. Since CHR provides a lot of analysis tools and its rules have a direct logical reading, this approach may support the verification of cognitive models. Additionally, the translation of the models is automated by a compiler, so it is not compulsory for modelers to learn a whole new language and legacy models may be translated to CHR.

% fix badboxes in toc
\makeatletter
\renewcommand{\@pnumwidth}{1.8em} 
\renewcommand{\@tocrmarg}{2em}
\makeatother

%%
%% Inhaltsverzeichnis
%%
\microtypesetup{protrusion=false}
\tableofcontents
\microtypesetup{protrusion=true}

%%
%% Hauptteil
%%
\mainmatter
\chapter{Introduction}

\emph{Computational psychology} or \emph{computational cognitive modeling} is a research field in the cognitive sciences. Among the other approaches -- mathematical and verbal-conceptual modeling, ``computational modeling appears to be the most promising approach in many ways and offers the flexibility and the expressive power that no other approaches can match'' \cite[vii]{sun_introduction_2008}. It explores human cognition by implementing detailed computational models that enable computers to execute them and simulate human behaviour \cite[3]{sun_introduction_2008}. By conducting the same experiments with humans and with simulations of the suggested underlying cognitive models, the plausibility of models can be checked and models can be improved gradually. This approach is illustrated in figure~\ref{fig:cognitive_modeling_exp}. Furthermore, an important benefit of computational models is that they are -- as a matter of principle -- very detailed and have a very clear semantics, in order that they can be executed by a computer. 

\begin{figure}[hbt]
\centering
% Define block styles
\tikzstyle{decision} = [diamond, draw, fill=blue!20, 
    text width=4.5em, text badly centered, node distance=3cm, inner sep=0pt]
\tikzstyle{block} = [rectangle, draw, fill=blue!20, 
    text width=5em, text centered, rounded corners, minimum height=4em]
\tikzstyle{line} = [draw, -latex']
\tikzstyle{cloud} = [draw, ellipse,fill=red!20, node distance=3cm,
    minimum height=2em]
    
\begin{tikzpicture}[node distance = 2cm, auto]
    % Place nodes
    \node [block] (experiment) {experiment};
    \node [cloud, below left of=experiment] (humansub) {\parbox{2cm}{\centering human subjects}};
    \node [cloud, below right of=experiment] (cogmod) {\parbox{2cm}{\centering cognitive model}};
    \node [decision, below of=experiment, node distance=4cm] (match) {match?};
    \node [block, below of=match, node distance=3cm] (update) {update model};
    % Draw edges
    \path [line] (experiment) -- (humansub);
    \path [line] (experiment) -- (cogmod);
    \path [line] (humansub) -- (match);
    \path [line] (cogmod) -- (match);
    \path [line] (match) -- node {no}(update);
    \path [line] (update) -| (cogmod);
    \path [line,dashed] (cogmod) |- node[pos=0.4,above right] {predictions}  (experiment);
\end{tikzpicture}
\caption{A typical workflow in computational cognitive modeling. After a model has been created, an experiment is built which is performed by humans and the cognitive model. Afterwards the results are checked and the model may be adapted. \cite{about_actr_homepage}}
\label{fig:cognitive_modeling_exp}
\end{figure}


However, psychology is experiencing a movement towards specialization \cite{anderson_integrated_2004}, i.e. there are a lot of independent, highly specialized fields that lack a more global view, which impedes cognitive modeling where a very detailed and complete view is necessary for execution:

\begin{quote}
``In 1972, [\dots it] seemed to me, also, that the cognitive revolution was already well in hand and established. Yet, I found myself concerned about the theory psychology was developing. [\dots] I tried to make that point by noting that what psychologists mostly did for theory was to go from dichotomy to dichotomy.'' \cite[1\psq]{newell_unified_1990}
\end{quote}

\citeauthor{newell_unified_1990} suggested in his book from \citeyear{newell_unified_1990} \cite{newell_unified_1990}, that for developing consistent models of cognition it is necessary to create a theory that tries to put all those highly specialized components together \cite[1036]{anderson_integrated_2004}. He therefore introduced the term \emph{cognitive architecture} \cite[5]{anderson_how_2007}, which today can be defined as ``a specification of the structure of the brain at a level of abstraction that explains how it achieves the function of the mind'' \cite[7]{anderson_how_2007}. A cognitive architecture provides the ability to create models for specific cognitive tasks \cite[29]{taatgen_modeling_2006} by offering ``representational formats together with reasoning and learning mechanisms to facilitate modeling'' \cite[29]{taatgen_modeling_2006}. Nevertheless, a cognitive architecture should also constrain modeling -- ideally it should only allow ``cognitive models that are cognitively plausible'' \cite[29]{taatgen_modeling_2006}. The relation of cognitive models and architecture is illustrated in figure~\ref{fig:cognitive_models_architecture}.

\begin{figure}[hbt]
\centering
% Define block styles
\tikzstyle{block} = [rectangle, draw, fill=blue!20, 
    text width=5em, text centered, rounded corners, minimum height=4em]
\tikzstyle{line} = [draw, -latex']
\tikzstyle{cloud} = [draw, ellipse,fill=red!20]
    
\begin{tikzpicture}[node distance = 2cm, auto]
 \matrix[row sep=5mm] {
 \node[cloud] (experiment) {\parbox{3cm}{\centering subset of psychology experiments}}; \\
    % Place nodes
    \node[cloud] (genassumpt) {\parbox{3cm}{\centering general assumptions about human cognition}}; &
&
 \node [cloud]  (assumptions) {\parbox{3cm}{\centering assumptions about a particular domain}};\\

 \node [block] (cogarch) {cognitive architecture};\\
 &
  \node [block] at (1,) (cogmod) {cognitive model}; \\
};
    % Draw edges
 \path [line] (experiment) -- (genassumpt);
\path [line] (genassumpt) -- (cogarch);
 \path [line] (assumptions) |- (cogmod);
 \path [line] (cogarch) |- (cogmod);
\end{tikzpicture}
\caption{Cognitive models usually are built upon a cognitive architecture by adding domain specific knowledge to the context of the architecture. The cognitive architecture contains general knowledge derived from psychological experiments \cite{about_actr_homepage}.}
\label{fig:cognitive_models_architecture}
\end{figure}

Adaptive Control of Thought-Rational  (ACT-R) is a cognitive architecture, that ``is capable of interacting with the outside world, has been mapped onto brain structures, and is able to learn to interact with complex dynamic tasks'' \cite[29]{taatgen_modeling_2006}, so its theory is well-investigated. It also is one of the most popular cognitive architectures in the field \cite{rutledge2005can} and provides an implementation that allows modelers to execute their models by a computer and hence offers a platform for computational cognitive modeling as defined above.

\section{Motivation and Goal}

This work presents an implementation of ACT-R using Constraint Handling Rules. Constraint Handling Rules (CHR) is a high-level rule-based programming language with several very well-defined operational and declarative semantics \cite[49\psqq]{fru_chr_book_2009}, so the meaning of a CHR program is logically defined. Additionally, there are a lot of methods to analyze CHR programs \cite[96\psqq]{fru_chr_book_2009} and there are a lot of very nice properties of CHR programs, like the anytime and online property \cite[83\psqq]{fru_chr_book_2009}. The declarativity of CHR programs does not affect the efficiency, so every algorithm which can be implemented efficiently in an imperative language can also be implemented efficiently in CHR and the constant slow-down of CHR compared to a C program is very low using high-optimizing compilers \cite[94]{fru_chr_book_2009}.

Since models are expressed by production rules in ACT-R, the idea to implement such models in CHR seems likely, because, at a first glance, the semantics of a CHR rule does not seem to be very different from an ACT-R production rule, so a lot of work like the \emph{efficient} implementation of the matching process can be saved. Due to the clear semantics and the several analysis methods, there is a hope that through the translation of the ACT-R theory to CHR there will be a benefit to the analysis of models and the inspection of the logical implications of a cognitive model.

The aim of this work is to implement the basic concepts of ACT-R and stick as closely as possible to the theory using the original implementation as a reference. Hence, a goal is to find translation schemes of ACT-R production rules to CHR rules and the implementation of the underlying cognitive architecture in CHR.

\section{Related Work}

There are several implementations of the ACT-R theory in different languages. First of all, there is the official ACT-R implementation in Lisp \cite{actr_homepage} often referred to as the \emph{vanilla} implementation. There are a lot of extensions to this implementation, which partly have been included to the original package in later versions like the ACT-R/PM extension that has been included in ACT-R 6.0 \cite[264]{actr_reference}. The implementation comes with an experiment environment which offers a graphical user interface to load, execute and observe models which communicates through a network interface with the ACT-R core.

\citeauthor{stewart_deconstructing_2007} have built an implementation in Python, which also had the aim to simplify and harmonize parts of the ACT-R theory by finding the central components of the theory \cites{stewart_deconstructing_2006,stewart_deconstructing_2007}. The authors describe another approach of implementing ACT-R without sticking too much to the classic implementations. For instance, the architecture has been reduced two only two components (the procedural and the declarative module which will be described in section~\ref{actr_description}) and build the rest of the architecture by using those two modules and combining them in different ways. Additionally, there is no possibility to translate traditional ACT-R models automatically to Python code. Hence, the focus of the work was different from the aims in this work.

Furthermore, there are two different implementations in Java: \emph{jACT-R} \cite{jactr} and \emph{ACT-R: The Java Simulation \& Development Environment} \cite{java_actr}. The latter one is capable of executing original ACT-R models and offers an advanced graphical user interface. The focus of the project was to make ACT-R more portable with the help of Java, since Lisp's ``extensibility for different task implementations and different hardware platforms has been lagging compared to more modern languages'' \cite{java_actr_benefits}. In jACT-R, the focus was to offer a clean and exchangeable interface to all the components, so different versions of the ACT-R theory can be mixed \cite{jactr_benefits} and models are defined using XML. Due to the modular design defining various interfaces which can be exchanged, jACT-R is highly adaptable to personal needs. However, since Java is the host language, there is no expected gain in declarativity and model analysis for both implementations.

There are approaches to implement psychological models using declarative and logic programming languages. In \cite{pereira_modellingmorality_2007} \citeauthor{pereira_modellingmorality_2007} present computational models for cognitive reasoning in the context of moral dilemmas using prospective logic programs. \citeauthor{balduccini_formalization_2010} show in \cite{balduccini_formalization_2010} how psychological knowledge can be formalized and reasoning over this knowledge can be achieved using answer set programming. However, both approaches are detached from a broadly distributed cognitive architecture like ACT-R. Nevertheless, in \cite[726]{balduccini_formalization_2010} it is emphasized, that for psychological ``theories of a more qualitative or logical nature [\dots] are not easy to formalize in'' the way of neural-networks or similar approaches, but need a more abstract approach.

\section{Overview}

This work is divided in several parts. First of all, an important task was the identification of the fundamental parts of the ACT-R theory. Hence, a description of the theory is given in chapter~\ref{actr_description}. Chapter~\ref{chr_introduction} gives a very brief introduction to Constraint Handling Rules. The result of the implementation of ACT-R in CHR is described in chapter~\ref{implementation}, which first formalizes some of the parts of the ACT-R theory before it suggests how to implement the afore described concepts in Constraint Handling Rules. Afterwards, the work of chapter~\ref{implementation} is demonstrated in some example models in section~\ref{example_models}. Chapter~\ref{conclusion} summarizes the work and gives an outlook to future work.

The work is accompanied by a CD with the source code and a digital version of the text (see appendix~\ref{appendix:cd_content}). The current version of the source code can also be downloaded at GitHub.

\chapter{Description of ACT-R}
\label{actr_description}

Adaptive Control of Thought-Rational  (ACT-R) is a cognitive architecture, that allows to implement cognitive models that are executable by a computer to produce experimental results that can be compared to experimental data from experiments that have been conducted with humans.

Because of the underlying theory which is the basis for the ACT-R cognitive architecture, modeling is facilitated, since the underlying concepts have not to be modelled again and again. On the other hand, it constrains the modeling process to, ideally, only plausible models.

When talking about ACT-R, one can refer to the theory or the implementation. The theory gives a view which abstracts from implementational details that may be concerned when talking about implementation \FIXME{source}. In this work, implementation always refers to the vanilla Lisp implementation that can be downloaded from \cite{actr_homepage}.

In this chapter, a short overview over the theory of ACT-R is given. First, the description is informal to provide a general image of how ACT-R works. Then, some important parts of the system are defined more formally in chapter \ref{implementation}, as soon as it is needed in the implementation. All of the information in this chapter refers to the theory. Implementation is discussed in chapter \ref{implementation}. A lot of the information in this chapter is based on \cite{anderson_how_2007, anderson_integrated_2004, taatgen_modeling_2006}, where a much more comprehensive discussion of the ACT-R theory including complex examples, referrings to the neuro-biology and the reasons why this particular modeling of human cognition has been chosen. In this work, only the basic concepts of ACT-R are presented.

\section{Procedural and Declarative Knowledge}

A central idea of ACT-R is the distinction between \emph{declarative} and \emph{procedural knowledge.} The declarative knowledge consists of simple facts, whereas the procedural knowledge contains information on what to do with those facts.

\subsection{Modular organization}

This approach leads to a modular organization of ACT-R with modules for each purpose needed to simulate human cognition. Figure \ref{fig:modular_organization} provides an overview of some of the default modules of ACT-R. For example, the declarative module stores the factual information (the declarative knowledge), the visual module perceives and processes the visual field, the procedural module holds the procedural information and controls the computational process. 

Each module is independent from the other modules and computations in the modules can be performed parallel to other modules, for instance: The declarative module can search a specific fact while the visual module processes the visual field. Additionally, within one module computations are executed massively parallel, e.g., the visual module can process the entire visual field at once to determine the location of a certain object, which implies the processing of a huge amount of data at a time.

However, each module can perform its computation only locally and has no access to computations of other modules. To communicate, modules have associated \emph{buffers}, where they can put a limited amount of information -- one primitive knowledge element -- and the procedural module can access each of these buffers. The information in a buffer could be one single fact retrieved from declarative memory or one visual object from the visual field perceived by the visual module. Information between modules is exchanged by the procedural module taking information from one buffer and putting it into another (with an optional computation on the way). This leads to a serial bottleneck in the computation, since every communication between modules has to go its way through the procedural module.

In figure \ref{fig:recognize_act} the general computational process is illustrated by showing the \emph{recognize-act-cycle}: The procedural information is stored as rules that have a \emph{condition} and an \emph{action}. The condition refers to the so-called \emph{working memory}, which basically is the content of all the buffers. In the recognize-phase of the cycle, a suitable rule that matches the current state of the working memory is searched. If the condition of a rule holds, it \emph{fires} and performs its actions -- this is the act-phase of the cycle. Those actions can cause changes on the buffers, so the next rule may match the current state in the next recognize-part of the cycle. In the following sections, some of the modules and their precise interaction will be described in more detail.

\subsection{Declarative Knowledge}

The declarative module organizes the factual knowledge as an associative memory. I.e., it consists of a set of concepts that are connected to each other in a certain way. Such elementary concepts are represented in form of chunks that can be seen as basic knowledge elements. They can have names, but those names are not critical for the description of the facts and just for readability in the theory. So, the definition of a chunk is based only on its connections. 

Chunks can have slots that are connected to other chunks or elements. Such an element can be regarded as a chunk without any slots. For instance, the fact $5 + 2 = 7$ can be modeled as a chunk that is connected to the numbers 5, 2 and 7 (see figure \ref{fig:chunk_addition_fact}). Notice that in the figure each slot has an individual name. This is necessary to distinguish the connections of the chunks, otherwise the summands would be indistinguishable from the sum in the example.

Thus, chunks are defined by their name and the values of their slots. When talking about chunk descriptions, often the term \emph{slot-value pairs} is used especially for partial chunk descriptions, i.e. descriptions which do not have a value for all possible slots. This simply refers to an arbitrary chunk that has the specified values in its slots (and the others are ignored).

Each chunk is associated with a chunk-type that determines the slots a chunk can have. For example, the fact in figure \ref{fig:chunk_addition_fact} has the type \verb|addition-fact|. All chunks of this type must provide the slots \verb|arg1|, \verb|arg2| and \verb|sum|.

\label{millers_number}
For the chunk-types there is no upper limit of slots they can define. However, it is suggested to limit the number of slots to Miller's Number of $7 \pm 2$, for the reason of plausibility \cite[230]{stewart_deconstructing_2007}. 

\subsubsection{Buffers}

As mentioned before, modules communicate through buffers by putting a limited amount of information into their associated buffers. More precisely, each buffer can hold only \emph{one chunk at a time}.

For example, the declarative module has the retrieval buffer associated with it, which can hold one specific declarative chunk. The declarative module can put chunks into the buffer that can be processed by the procedural module, which is described in the next section.

\subsection{Procedural Knowledge}
\label{procedural_knowledge}

Procedural Knowledge in ACT-R is formulated as a set of condition-action rules. Each rule defines in its condition-part the circumstances under which it can be applied. Those conditions refer to the current chunks in the respective buffer. The condition-part of a rule defines which kind of chunk with which slot values must be present in which buffer for the rule to fire. For example, one rule in the process of adding the numbers 5 and 2 could have the conditions that there is a chunk of type \verb|addition-fact| in the retrieval buffer with 5 and 2 in its argument-slots and specify certain actions if this is the case.

If the chunks in the buffers match all the conditions stated in a rule, it can be applied (``fired''), which leads its action-part to be performed. Possible actions are changes of some of the values in the chunk of a buffer, the clearing of a buffer or a buffer request, which leads the corresponding module to put a certain chunk into the requested buffer. Buffer requests are also stated in form of a (partial) chunk description\footnote{A partial chunk description is just a chunk description that does not specify all slots that are available as defined in the chunk-type.} where chunk-type and slots encode the query of the request. So all the arguments and even the task which should be performed by the module are specified through a chunk representation. The actual semantics of a request depends on the module. For example, the declarative module will search a chunk that matches the chunk in the description of the request. One production rule, for instance, in the process of adding the numbers 5 and 2 could be, if the wrong \verb|addition-fact| chunk is stored in the retrieval buffer, a retrieval request will be performed, which states that the declarative module should put a chunk into the retrieval buffer, that has 5 and 2 in its argument slots and is of type \verb|addition-fact|. After the successful performance of the request, a chunk with 5 and 2 in its argument slots will be stored in the retrieval buffer, that also has a value for the sum. The actions are described in more detail in the following section.

Although the term \emph{module} is used for the procedural system, it differs a lot from the other modules: In contrast to other modules, the procedural module has no own buffers, but can access the buffers of all the other modules. ``It really is just a system of mapping cortical buffers to other cortical buffers'' \cite[p. 54]{anderson_how_2007}.

The procedural system can only fire one rule at once and it takes 50\,ms for a rule to fire \cite[p. 54]{anderson_how_2007}. After firing the selected rule, the next recognize cycle starts and a suitable rule will be detected and caused to fire. During this time, other modules may perform requests triggered by the action of the last rule. Sometimes, rules have to wait for results of certain modules and they cannot fire before those results are available. Those two facts illustrate how the procedural module can become a serial bottleneck in the computation process.

\subsubsection{Description of Procedural Actions}
\label{description_of_proc_actions}

In this section, the actions that can be performed by a production rule are described in more detail than before. The information in this section has been taken from \FIXME{source} and is -- in this degree of detail -- not part of the theory, but focuses more on the implementation to give a more detailed understanding of the concepts needed in chapter~\ref{implementation}.

\begin{description}
 \item[Buffer Modification:] 
 
 An in-place operation, that overwrites the slot values of a chunk in a buffer with the specified values in the action of the rule.
 
 \item[Buffer Request:] 
 
 A buffer request will cause the corresponding module to calculate some kind of result that will be placed into the requested buffer. The input values of this computation are given as chunks with a type and slot-value pairs specified in the request. For instance, the declarative module could search for a chunk that has the specified values in its slots.
 
 The execution of the request is independent from the execution of production rules and after the request has been stated by the procedural module, it can begin with the next recognize-cycle while the requested module calculates its result.
 
 Before the request is performed, the corresponding buffer will be cleared.
 
 \item[Buffer Clearing:] 
 
 If a buffer is cleared, its containing chunk will be placed into the declarative memory from where it can be retrieved later on. The clearing of a buffer with the implicit storing of the chunk in the declarative memory is an implementational detail which is very important for further considerations.
\end{description}

\subsubsection{Chunks as Central Data Structure}

As may have become obvious in the previous sections, chunks are the central data structures in ACT-R. They are used to model factual knowledge in the declarative memory, but are also used for communication: Requests are stated as chunks that encode the input of the request, for instance a chunk pattern for a result chunk the declarative memory should retrieve. The result of a request is a chunk placed into a buffer and even the procedural system, which technically is separated from the declarative knowledge, tries to match the chunks in the buffers in the condition part. Additionally, the action of a rule is specified by slot-value pairs that are basically just partial chunk descriptions. 

\subsubsection{Process of Rule Selection and Execution}
\label{process_of_rule_selection_and_execution}

\FIXME{add sources}

As stated above, the procedural module can execute only one rule at a time. If no rule has been selected to fire -- so no rule is in progress -- the procedural module is \emph{free} and therefore can select a matching rule according to the recognize-act-cycle. If a rule has been selected, the module is \emph{busy} and cannot choose another rule to fire. Between selection and firing of a rule the module has to wait $50\,\mathrm{ms}$. Then all in-place actions of the rule like modifying or clearing a buffer are performed. Afterwards, the requests are stated and the module is free. However, the requested modules most likely will take a certain time to perform the request. During this time the procedural module can select and fire the next matching rule nevertheless.

If at a certain time the procedural module is free, but there are no matching rules, the module waits until the system reaches a state where a rule matches. This is possible, since requests can take a certain time in which the procedural module is free and cannot find a matching rule. If the request has been performed, it usually causes a change of buffers. When the content of a buffer has changed, this could provoke the next rule to match and fire.

\subsection{Goal Module}

An essential part of human cognition is the ability to keep track of the current goal to achieve and to subordinate all actions to the goal \cite[p. 1041]{anderson_integrated_2004}. For complex cognitive tasks, several rules have to be applied in series and intermediate results must be stored (without changing of the environment). Another important aspect is that complex tasks may consist of several subgoals which have to be achieved to accomplish the main goal. For instance, if one wants to add two multi-digit numbers, he would add the columns and remember the results as intermediate results in each step. In ACT-R, the goal module with its goal buffer is used for this purpose: It is able to keep track of the current goal, introduce subgoals and remember intermediate results in its buffer. 

\subsubsection{Working memory}

The goal module and buffer are often referred to as \emph{working memory} \cite[1041]{anderson_integrated_2004}, but actually, as stated in \cite{anderson_working_1996}, it also can have another meaning: The usual definition in production systems is that everything which is present to the production rules and can match against them is part of the working memory. With this definition, all chunks in the buffers form the working memory.

In this work, the term \emph{working memory} will be used in this second meaning, since it discusses the topic from a computer science view and the second definition is related to production rule systems. When talking about the content of the goal buffer, this will be remarked explicitly.

\subsection{Other Modules}

In figure \ref{fig:modular_organization} some more modules are shown. In the following, a short description of some of those modules is given.

\subsubsection{The Outside World}

Since human cognition is embodied, there must be a way to interact with the outside world to simulate human cognition in realistic experiments. Therefore, ACT-R offers \emph{perceptual/motor modules} like the manual module for control of the hands, the visual module for perceiving and processing the visual field or the aural module to perceive sounds in the environment. Like with every other module, communication is achieved through the buffers of those modules. In the following, the visual module is described to exemplify the functionality of perceptual modules.

\paragraph{The Visual Module} The visual system of ACT-R separates vision into two parts: visual location and visual objects \cite[p. 1039]{anderson_integrated_2004}. There are two buffers for those purposes: the \emph{visual-location} buffer and the \emph{visual} buffer, which represents the visual objects \cite[unit~2]{actr_tutorial}. In the visual module it is not encoded how the light falls on the retina, but a more attentional approach has been chosen \cite[p. 1039]{anderson_integrated_2004}. 

Requests to the \emph{visual-location buffer} specify a series of constraints in form of slot-value pairs and the visual module puts a chunk representing the location of an object meeting those constraints into the visual-location buffer. Possible constraints are properties of objects like the color or the spatial location. The visual system can process such requests in parallel, i.e. that the whole visual field is processed massively parallel and, for example, the time of finding one green object surrounded by blue objects is constant, regardless of the number of blue objects. If more than one object meets the constraints, one of them will be chosen at random \cites[1039]{anderson_integrated_2004}[68]{anderson_how_2007}.

Requests to the \emph{visual-object system} specify a visual location and the visual module will move its attention to that location, create a new chunk representing the object at that location and put that chunk into the visual buffer \cite[unit 2, chapter 2.5.3]{actr_tutorial}. 

These two kinds of requests to the visual module are summarized in table~\ref{tab:visual_module_requests}.

\begin{table}[hbt]
\caption{Requests to the visual module}
\label{tab:visual_module_requests}
\begin{center}
\begin{tabular}{|l|ll|}
\hline
 & Visual location buffer & Visual (object) buffer\\
\hline
Input & Object Constraints & Visual location\\
Output & Visual location & Visual object\\
\hline
\end{tabular}
\end{center}
\end{table}

The visual system and its capabilities are described in detail in \cite[unit 2]{actr_tutorial} where also the implementational details of the system are regarded.

\subsubsection{The Imaginal Module}

The imaginal module is capable of creating new chunks. This is useful, if for instance the visual module produces a lot of new information in sequence (like reading a sequence of letters), but the visual-object buffer can hold only one chunk at once. To solve this problem, all the information could be stored in the slots of the goal chunk. However, since a goal chunk with a large amount of slots seems to be unplausible\footnote{As described in section \ref{millers_number}, one should stick to $7 \pm 2$ slots for each chunk.} and the number of read instances would have to be known in advance due to the static chunk-type definition, a better way to deal with this problem is to create new knowledge elements.

This task can be achieved by using the imaginal module: On a request, it creates a new chunk of the type and with the slots stated in the request and puts it into its \emph{imaginal buffer}. Since the chunk in a buffer is stored in the declarative memory when the buffer is cleared\footnote{see section \ref{description_of_proc_actions}}, an unlimited amount of data can be produced and remembered by stating retrieval requests later on.

It is important to mention that it takes the imaginal module $.2\,\mathrm{ms}$ to create a chunk. This amount of time is constant, but can be set by the modeler. Additionally, the imaginal module can only produce one chunk at a time\footnote{like every module can only handle one request at a time}.

The imaginal module is described in \cite[unit 2]{actr_tutorial}.

\subsection{Example: Counting}
\label{example_counting}

The first ACT-R example model deals with the process of counting. This model relies on count facts a person has learned, e.g. ``the number after $2$ is $3$''. To model this in ACT-R, a chunk-type for those facts has to be defined: A chunk of type \emph{count-fact} has the slots \emph{first} and \emph{second}. The chunks in figure \ref{fig:example_counting_chunks} of this type model the facts that $3$ is the successor of $2$ and $4$ is the successor of $3$.

The next step is to define the goal chunk stored in the goal buffer. In this chunk it somehow has to be encoded that the current goal is to count. This can be modeled in ACT-R by the chunk-type. To track the current number in the counting process as an intermediate result, the goal chunk could have a slot which always holds the current number that has been counted to. This leads to a goal chunk as illustrated in figure \ref{fig:example_counting_goal_chunk}, where the current number is $2$. In this example we assume that the model starts with this goal chunk in the goal buffer and the first count fact has been retrieved:

\parbox{100mm}{\textbf{goal buffer:} goal-chunk of type count\\
\noindent\hspace*{20mm} current-number $2$}

\parbox{100mm}{\textbf{retrieval buffer:} b of type count-fact\\
\noindent\hspace*{20mm} first $2$\\
\noindent\hspace*{20mm} second $3$}

This notation indicates that the goal buffer holds a chunk with the name \emph{goal-chunk} of the type \emph{count}, which has the slot \emph{current-number} with the value $2$ (the same is valid analogously for the retrieval buffer).

Now the rules to implement counting can be defined as:

\begin{center}
\begin{tikzpicture}[node distance=0.5cm]
\node (count-rule) [rule, rectangle split, rectangle split parts=4, rectangle split part fill={Uulmin, Uulmin!45,Uulmin!45,Uulmin!80,Uulmin!80}] {
        \textbf{count-rule}
        \nodepart{second} IF the goal is to count, the current number is $n$
        \nodepart{third}AND the retrieval buffer holds a chunk of type \emph{count-fact} with the \emph{first} value $n$ and the \emph{second} value $m$
        \nodepart{fourth}THEN set the current number in the goal to $m$
                    AND send a retrieval request for a chunk that has $m$ in its \emph{first} slot
    };
\end{tikzpicture}
\end{center}

The rule matches the initial state: In the goal there is a chunk of type \emph{count}, that indicates that the goal is to count, the current number $n$ is $2$. In the retrieval buffer, there is a \emph{count-fact} with the \emph{first} number $n = 2$ and the \emph{second} number $m = 3$. 

After applying this rule, the current number will be $3$ and the next fact in the retrieval buffer will be a \emph{count-fact} with the \emph{first} value $3$ and a value in the \emph{second} slot, which will be the next number in the counting process. This illustrates the functionality of module requests: In the request a (potentially partial) chunk definition is stated and the corresponding module puts the result of the request in a fully defined chunk of some appropriate type into its buffer. For the declarative module, the request specifies the chunk-type and some slot values which describe the chunk that the module should be looking for. The result is a fully described chunk of that type with values for all slots, that describe an actual chunk from the declarative memory. \FIXME{chunks are copies!! describe somewhere}

The count-rule will be applicable as long as there are \emph{count-facts} in the declarative memory. Figure \ref{fig:example_counting_execution} illustrates the counting process.

In this example, the rules have been defined in a very informal way. In the following chapters which deal with implementation, a formalization of such rules will be discussed, that defines clearly what kinds of rules are allowed. Furthermore, it introduces a formalism to describe such rules uniquely and less verbosely. The following chapters will refer to this example and refine it gradually.

The example also uses the concept of \emph{variables}, which will be introduced more formally in chapter \ref{implementation} when talking about implementation. Variables allow rule conditions to act like patterns that can match various system states instead of defining a rule for each state, since computation is the same regardless of the actual values in the buffers.


\FIXME{figures}


\section{Serial and Parallel Aspects of ACT-R}
\label{serial_parallel_aspects}

In the previous sections there were some remarks on the serial and parallel aspects of ACT-R. According to \cite[p. 68]{anderson_how_2007}, four types of parallelism and seriality can be distinguished:

\begin{description}
 \item[Within-Module Parallelism:] As mentioned above, one module is able to explore a big amount of data in parallel. For example, the visual module can inspect the whole visual field or the declarative module performs a massively parallel search over all chunks.
 \item[Within-Module Seriality:] Since modules have to communicate, they have a limited amount of buffers and each of those buffers can only hold one chunk. For example, the visual module only can concentrate on one single visual object at one visual location, the declarative module only can have one single concept present, the production system can fire only one rule at a time, \dots 
 \item[Between-Module Parallelism:] Modules are independent of each other and their computations can be performed in parallel.
 \item[Between-Module Serialism:] However, if it comes to communication, everything must be exchanged via the procedural module that has access to all the buffers. Sometimes, the production system has to wait for a module to finish, since the next computation relies on this information. So, modules may have to wait for another module to finish its computation before they can start with theirs triggered by a production rule that states a request to those modules.
\end{description}

The procedural module is the central serial bottleneck in the system, since the whole communication between modules is going through the production system and the whole computation process is controlled there. The fact that only one rule can fire at a time leads to a serial overall computation. Another serial aspect is that some computations need to wait for the results of a module request. If no other rule matches in the time while the request is performed, the whole system has to wait for this calculation to finish. After the request, the module puts the result in its buffer and the rule needing the result of the computation can fire and computation is continued.

\section{Subsymbolic layer}
\label{subsymbolic_layer}

The previously discussed aspects of the ACT-R theory are part of the so-called symbolic layer. This layer only describes discrete knowledge structures without dealing with more complex questions like: 

\begin{itemize}
\item How long does it take to retrieve a certain chunk? 
\item Forgetting of chunks
\item If more than one rule matches, which one will be taken?
\end{itemize}

Therefore, ACT-R provides a subsymbolic layer that introduces ``neural-like activation processes that determine the availability of [\dots] symbolic structures'' \cite{anderson_implications_2000}.

\subsection{Activation of Chunks}
\label{activation}

The activation $A_i$ of a chunk $i$ is a numerical value that determines if and how fast a chunk can be retrieved by the declarative module. Suppose there are two chunks that encode addition facts for the same two arguments (let them be 5 and 2), but with different sums (6 and 7), for example. This could be the case, if, e.g., a child learned the wrong fact about the sum of 5 and 2. When stating a module request for an addition fact that encodes the sum of 5 and 2, somehow one of the two chunks has to be chosen by a certain method, since they are both matching the request. This is determined by the activation of the chunks: The chunk with the higher activation will be chosen.

Additionally, a very low chunk activation can prevent a chunk from being retrieved: If the activation $A_i$ is less than a certain \emph{threshold} $\tau$, the chunk $i$ cannot be found.

At last, activation determines also how fast a chunk is being retrieved: The higher the activation, the shorter the retrieval time.

\subsubsection{Base-Level Activation}
\label{base_level_activation}

The activation $A_i$ of a chunk $i$ is defined as:

\begin{equation}
 \label{eq:activation_equation_simpl}
 A_i = B_i + \Gamma
\end{equation}

where $B_i$ is the \emph{base-level activation} of the chunk $i$. $\Gamma$ is a context component that will be described later on. Equation \eqref{eq:activation_equation_simpl} is a simplified variant of the \emph{Activation Equation}.

The base-level activation is a value associated with each chunk and depends on how often a chunk has been practiced and when this practice has been performed. A chunk is \emph{practiced} when it is retrieved. Hence, $B_i$ of chunk $i$ is defined as:

\begin{equation}
\label{eq:base_level_learning}
B_i = \mathrm{ln}\left(\sum_{j=1}^n{t_j^{-d}}\right)
\end{equation}

where $t_j$ is the time since the $j$th practice, $n$ the number of overall practices of the chunk and $d$ is the decay rate that describes how fast the base-level activation decreases if a chunk has not been practiced (how fast a chunk will be forgotten). Usually, $d$ is set to $0.5$ \cite[p. 1042]{anderson_integrated_2004}. Equation \eqref{eq:base_level_learning} is called \emph{Base-Level Learning Equation}, as it defines the adaptive learning process of the base-level value.

This equation is the result of a rational analysis by Anderson and Schooler. It reflects the log odds that a chunk will reappear depending on when it has appeared in the past \cite[33]{taatgen_modeling_2006}. This analysis led to the \emph{power law of practice} \cite[1042]{anderson_integrated_2004}. In \cite[8--11]{anderson_implications_2000} equation \eqref{eq:base_level_learning} is motivated in more detail by describing the power law of learning/practice, the power law of forgetting and the multiplicative effect of practice and retention with some data. Shortly, it states, that if a particular fact is practiced, there is an improvement of performance which corresponds to a power law. At the same time, performance degrades with time corresponding to a power law. Additionally, they state that if a fact has been practiced a lot, it will not be forgotten for a longer time.

\subsubsection{Activation Spreading}

In ACT-R, the basic idea of activation is that it consists of two parts: The base-level component described above, and a context component. Every chunk that is in the current context has a certain amount of activation that can spread over the declarative memory and enhance activation of other chunks that are somehow connected to those chunks in the context. The activation equation \eqref{eq:activation_equation_simpl} is extended as follows:

\begin{equation}
\label{eq:activation_equation}
 A_i = B_i + \sum_{j \in C}{W_j S_{ji}} + \varepsilon
\end{equation}

where $W_j$ the \emph{attentional weighting} of chunk $j$, $S_{ji}$ the \emph{associative strength} from chunk $j$ to chunk $i$ and $C$ is the \emph{current context}\label{current_context}, usually defined as the set of all chunks that are in a buffer \cites[1042]{anderson_integrated_2004}[33]{taatgen_modeling_2006}[unit 5]{actr_tutorial}. $\varepsilon$ is a noise value ``generated according to a logistic distribution'' \cite[unit 4, p. 4]{actr_tutorial}. Figure \ref{fig:chunk_activation} illustrates the addition-fact $5 + 2 = 7$ with the corresponding quantities introduced in the last equation. 

The values for $W_j$ determines how much activation can come from a single source of activation from the current context. A source of activation is a chunk in the goal buffer or in all buffers\footnote{This is called the current context: Usually it means the set of all chunks in all buffers, but there are definitions in literature, that only call the chunk in the goal buffer current context.}, depending on the version of the ACT-R theory \cites[1042]{anderson_integrated_2004}[33]{taatgen_modeling_2006}[unit 5, p. 1]{actr_tutorial}. To limit the total amount of source of activation, $W_j$ is set to $\frac{1}{n}$, where $n$ is the number of sources of activation. With this equation, the total amount of activation that can spread over declarative memory is limited, since the more chunks are in the current context, the less important becomes a particular connection between a chunk from the context with a chunk from declarative memory.

Figure \ref{fig:activation_spreading} illustrates the activation spreading process.

\paragraph{Strength of Association and Fan effect}

In equation \eqref{eq:activation_equation} the strength of association $S_{ji}$ from a chunk $j$ to a chunk $i$ is used to determine the activation of a chunk $i$.  In the ACT-R theory, the value of $S_{ji}$ is determined by the following rule: If chunk $j$ is not a value in the slots of chunk $i$ and $j \neq i$, then $S_{ji}$ is set to 0. Otherwise $S_{ji}$ is set to: 

\begin{equation}
\label{eq:assoc_strength}
S_{ji} = S - \mathrm{ln}(\mathrm{fan}_j)
\end{equation}

where $\mathrm{fan}_j$ is the number of facts associated to term $j$ \cite[1042]{anderson_implications_2000}. In more detail: ``$\mathrm{fan}_j$ is the number of chunks in declarative memory in which $j$ is the value of a slot plus one for chunk $j$ being associated with itself'' \cite[unit 5, p. 2]{actr_tutorial}. Hence, equation \eqref{eq:assoc_strength} states that the associative strength from chunk $j$ to $i$ decreases the more facts are associated to $j$.

This is due to the \emph{fan effect:} The more facts a person studies about a certain concept, the more time he or she needs to retrieve a particular fact of that concept \cite[186]{anderson_fan_1999}. This has been demonstrated in an experiment presented in \cite{anderson_fan_1999}, where every participant studied facts about persons and locations like:

\begin{itemize}
 \item A hippie is in the park.
 \item A hippie is in the church.
 \item A captain is in the bank.
 \item \dots
\end{itemize}

For every person the participants studied either one, two or three facts. Afterwards, they were asked to identify targets, that are sentences they studied, and foils, i.e. sentences constructed from the same persons and locations, but that were not in the original set of sentences. Figure \ref{fig:fan_network} represents an example chunk network of the studied sentences (based on \cite[fig.~1]{anderson_fan_1999}).

``The term \emph{fan} refers to the number of facts associated with a particular concept'' \cite[186]{anderson_fan_1999}. In figure \ref{fig:fans}, some facts are shown with their fan, $S_{ji}$ and $B_i$ values.

The result of the experiment was, that the more facts are associated with a certain concept, the higher was the retrieval time for a particular fact about that concept.

In the ACT-R theory, this result has been integrated in the calculation of the strengths of association: In equation \eqref{eq:assoc_strength} the associative strength decreases with the number of associated elements. The value $S$ is a model-dependent constant, but in many models estimated about 2 \cite[1042]{anderson_integrated_2004}. Modelers should take notice of setting $S$ high enough that all associative strengths in the model are positive \cite[unit 5, p. 3]{actr_tutorial}.

\subsubsection{Latency of Retrieval}

As mentioned before, the activation of a chunk affects if the chunk can be retrieved (depending on a threshold and the activation values of the other matching chunks). In addition, activation also has an effect on the retrieval time of a chunk:

\begin{equation}
\label{eq:retrieval_latency}
T_i = F \cdot \mathrm{e}^{-A_i}
\end{equation}

where $T_i$ is the \emph{latency} of retrieving chunk $i$, $A_i$ the activation of this chunk, as defined in equation \eqref{eq:activation_equation}, and $F$ the \emph{latency factor}, which is usually estimated to be

\begin{equation}
F \approx 0.35\mathrm{e}^\tau
\end{equation}

where $\tau$ is the retrieval threshold as mentioned in section \ref{activation}, but $F$ can also be set individually by the modeller. Nevertheless, in \cite[1042]{anderson_integrated_2004} it is stated that the relationship of the retrieval threshold and the latency factor in equation \eqref{eq:retrieval_latency} seems to be suitable for a lot of models.

\subsection{Production Utility}
\label{production_utility}

For the production system, there is the subsymbolic concept of \emph{production utilities} to deal with competing strategies. For instance, if a child learns to add numbers, it may have learned different strategies to compute the result: One could be counting with the fingers and the other could be just retrieving a fact for the addition from declarative memory. If the child now has the goal to add two numbers, it somehow has to decide which strategy it will choose, since both of them match the context.

In ACT-R, the production utility is a number attached to each production rule in the system. Just like with activation of chunks, the production rule with the highest utility will be chosen, if there is more than one matching rule. The utilities can be set statically by the modeler, but they can also be learned automatically by practice.

In the current version of the ACT-R theory, a reinforcement learning rule based on the Rescorla-Wagner learning rule \cite{rescorla_wagner_1972} has been introduced. The utility $U_i$ of a production rule $i$ is defined as:

\begin{equation}
\label{eq:utility_learning}
U_i(n) = U_i(n - 1) + \alpha \left(R_i(n) - U_i(n - 1)\right)
\end{equation}

where $\alpha$ is the \emph{learning rate} which is usually set around .2 and $R_i(n)$ is the reward the production rule $i$ receives at its $n$\textsuperscript{th} application \cite[160--161]{anderson_how_2007}. This leads the utility of a production rule being gradually adjusted to the average reward the rule receives \cite[6--7]{actr_tutorial}. 

Usually, rewards can occur at every time and it is not clear which production rule will be strengthened by the reward. In \cite[161]{anderson_how_2007} an example is described, where a monkey receives a squirt of juice a second after he presses a button. The question now is, which production rule is rewarded, since between the reward and the firing of a rule there always is a break. In ACT-R, every production that has been fired since the last reward event will be rewarded, but the more time lies between the reward and the firing of the rule, the less is the reward this particular rule receives. The reward for a rule is defined as the amount of external reward minus the the time from the rule to the reward. This implies, that the reward has to be measured in units of time, e.g., how much time is a monkey willing to spend to get a squirt of juice? \cite[161]{anderson_how_2007}

In implementations of ACT-R, rewards can be triggered by the user at any time or can be associated with special production rules that model the successful achievement of a goal (they check, if the current state is a wanted state and then trigger a reward, so every rule that has led to the successful state will be rewarded).

It is important to mention that by the definition of the reward for a production rule, rules also can get a negative reward, if their selection was too long ago. If one wants to penalize all rules since the last reward, a rule that distributes a reward of 0 can be triggered, which leads all rules applied before being rewarded with a negative amount of reward \cite[unit 6, p. 8]{actr_tutorial}.

\section{Learning}

Learning in ACT-R can be divided into four types depending on the involvement of the symbolic or subsymbolic layer and the declarative or the procedural module. Table~\ref{tab:learning_types} names the four types that are described in this section.

\begin{table}[hbt]
\caption{ACT-R's Taxonomy of Learning \cite[92--95]{anderson_how_2007}}
\begin{center}
\begin{tabular}{|l|ll|}
\hline
 & Declarative & Procedural\\
\hline
Symbolic & Fact learning & Skill acquisition\\
Subsymbolic & Strengthening & Conditioning\\
\hline
\end{tabular}
\end{center}
\end{table}


\subsection{Symbolic Layer}

Symbolic learning somehow influences the objects of the symbolic layer, i.e. chunks and production rules, in a way that new objects are created or objects are merged. Those learning possibilities are described in the following.

\subsubsection{Fact Learning}



\subsubsection{Skill acquisition}

\subsection{Subsymbolic Layer}

The concepts introduced in section~\ref{subsymbolic_layer} are a kind of learning: The practice of particular facts strengthens the chunks encoding this fact and chunks that are not practiced are forgotten over time\footnote{This is the concept of base-level learning as described in section~\ref{base_level_activation}}. Additionally, associative weights are learned from the current context. These processes adapt to the problems a particular human mind is confronted with and work autonomously.

The same is valid for production rules: Over time, the experience tells us, which strategies might be successful in certain situations and which are not. This process is also called \emph{conditioning} and described in equation~\eqref{eq:utility_learning}.


\cite[chapter 4]{whitehill_understanding}
\FIXME{mention experiment environment}

\chapter{Constraint Handling Rules}
\label{chr_introduction}

This chapter only gives a very brief introduction to Constraint Handling Rules. A much more comprehensive introduction can be found in the book \citetitle{fru_chr_book_2009} by \citeauthor{fru_chr_book_2009} \cite{fru_chr_book_2009}, which gives a very detailed definition of the exact semantics of the language and analysis methods as well as practical examples. Additionally, there are a lot of tutorials, slides and exercises that can be found on \citetitle{chr_homepage} \cite{chr_homepage}. This little introduction is based on \cite{fru_chr_book_2009}.

Constraint Handling Rules (CHR) is a high-level declarative programming language originally designed to write constraint-solvers \cite[2]{chr_survey_tplp10}. CHR programs are defined by a sequence of rules, that operate on a \emph{constraint store}, which is a multiset of constraints (built from predicate symbols and arguments). I.e., a constraint \lstinline|c/n| with arity \lstinline|n| is written as \lstinline[mathescape]|c($t_1, \dots, t_n$)|, where \lstinline|c| is a predicate symbol and the $t_i$ -- the arguments of the constraint -- are logical terms. Constraints in the store can be \emph{ground}, i.e. they do not contain any variables, but they also may contain unbound variables. CHR is usually embedded in a \emph{host language}, which provides so-called \emph{built-in} constraints. Such built-in constraints are constraints that are implemented in the host language. In this work, Prolog is assumed to be the host language. There are three types of rules:

\begin{lstlisting}[mathescape,escapeinside={(*@}{@*)},caption={CHR rule types}, label=lst:chr_rule_types]
<<<simplification>>> @ <<H$_\mathtt{r}$>> <=> <<<G>>> | B. (*@\label{lst:chr_rule_types:simplification}@*)
<<<propagation>>> @ <<H$_\mathtt{k}$>> ==> <<<G>>> | B. (*@\label{lst:chr_rule_types:propagation}@*)
<<<simpagation>>> @ <<H$_\mathtt{k}$ \ H$_\mathtt{r}$>> <=> <<<G>>> | B. (*@\label{lst:chr_rule_types:simpagation}@*)
\end{lstlisting}

Rules consist of a \emph{head}~\lstinline[mathescape]|H$_\mathtt{*}$|, which is a conjunction of at least one constraint, an optional \emph{guard}~\lstinline|G|, which is a conjunction of built-in constraints, and a \emph{body}~\lstinline|B|, which is a conjunction of both built-in and CHR constraints. Optionally, the rules can be given names followed by the symbol \lstinline|@|.

At the beginning of a program, the user provides an initial constraint store, which is also called \emph{query} or \emph{goal} and the CHR program then begins to operate on this initial store by applying rules.

A rule can be applied on the constraint store, if there are constraints in the store that match the head of the rule, i.e. the constraints in the store are ``an instance of the head, [\dots so] the head serves as pattern'' \cite[11]{fru_chr_book_2009}. Note that in the matching process no variables of the query are bound, since CHR is a committed-choice language, so rule applications cannot be made undone by backtracking. If matching constraints have been found, the guard is checked and if it holds, the rule is applied. The result of the rule application is dependent on the type of the rule:

\begin{description}
 \item[Simplification rules (as in line~\ref{lst:chr_rule_types:simplification})] remove the constraints which match the head \lstinline[mathescape]|H$_\mathtt{r}$| (the \emph{``heads removed''}) from the store and add the constraints in the body \lstinline|B| to the store.
 \item[Propagation rules (as in line~\ref{lst:chr_rule_types:propagation})] add the constraints in the body \lstinline|B| to the store and keep the matching constraints from the head \lstinline[mathescape]|H$_\mathtt{k}$| (the \emph{``heads kept}'').
 \item[Simpagation rules (as in line~\ref{lst:chr_rule_types:simpagation})] are a mix of the first two rule types: They add the constraints in the body \lstinline|B| to the store, keep the matching constraints from the head \lstinline[mathescape]|H$_\mathtt{k}$| and remove the constraints from the head \lstinline[mathescape]|H$_\mathtt{r}$|. In general, simplification and propagation rules can be expressed by simpagation rules where one side of the head is empty, i.e. only contains the constraint \lstinline|true|.
\end{description}

\begin{example}[Minimum]
\label{ex:minimum}

This simple example has been taken from \cite[19\psqq]{fru_chr_book_2009} and describes a CHR program that computes the minimum among the numbers $n_i$ given as multiset of numbers in the query \lstinline[mathescape]|min($n_1$), ... , min($n_k$)|. The \lstinline|min/1| constraint is interpeted such that its containing number is a minimum candidate. The minimum computation is achieved by specifying one simpagation rule:

\begin{lstlisting}[caption={Minimum program}]
min(N) \ min(M) <=> N=<M | true.
\end{lstlisting}

For every pair of numbers in the store, the rule removes the greater one and keeps the smaller one. If this rule is applied to exhaustion, only one constraint is left -- the minimum. Then the rule is not applicable any more, since it lacks a partner constraint for the single \lstinline|min| constraint to match the head.

Note that in pure CHR no assumptions on the order of the application of the rules and the involved constraints are made. One possible derivation of the result could be (the constraints the rule is applied to are underlined in each step):

\begin{lstlisting}[escapeinside={(*@}{@*)},caption={Sample derivation of the minimum example},label=lst:sample_derivation_min]
(*@\underline{min(3)}@*), (*@\underline{min(1)}@*), min(4), min(0), min(2)
(*@\underline{min(1)}@*), (*@\underline{min(4)}@*), min(0), min(2)
(*@\underline{min(1)}@*), (*@\underline{min(0)}@*), min(2)
(*@\underline{min(0)}@*), (*@\underline{min(2)}@*)
min(0)
\end{lstlisting}
\end{example}

\begin{example}[Matching]
This example tries to clarify the matching process and is taken from \cite[12]{fru_chr_book_2009}. The following rules are added into individual, separated programs and some queries are tested in those programs.

\begin{lstlisting}
p(a) <=> true.
p(X) <=> true.
p(X) <=> X=a.
p(X) <=> X=a | true.
p(X) <=> X==a | true.
\end{lstlisting}

The query \lstinline|?- p(a)| matches all the rules, so the result will always be an empty constraint store, indicated by \lstinline|true|.

For the query \lstinline|?- p(b)| the first rule does not match and \lstinline|p(b)| is added to the constraint store. The second rule is applicable, since \lstinline|p(X)| is a pattern for \lstinline|p(a)|. For the third rule, the result of the computation is \lstinline|false|, because the rule is applicable, but the unification \lstinline|a=b| in the body will fail. Since CHR is a committed-choice language, the rule selection will not be made undone. The last two rules are not applicable, since the guard fails.

The query \lstinline|?- p(Y)| does not match the first rule, since \lstinline|p(a)| is not a pattern \lstinline|p(Y)| -- the goal is more general than the head of the rule and no bindings of the constraints in the goal will be performed for the matching. The second rule does match, because the \lstinline|X| from the head of the rule can be bound to \lstinline|Y| from the goal. The result of the application of the third rule is \lstinline|Y=a|, because the rule does match and leads to a binding of \lstinline|Y| to \lstinline|a| in the body. The last two rules do not fire, because their guards fail. If \lstinline|Y| has been bound to \lstinline|a| explicitly before (e.g. by another rule), the rule could fire.
\end{example}


There are various definitions of the operational semantics -- i.e. the behaviour of CHR programs -- for different purposes and there may come more in the future \cite[11]{fru_chr_book_2009}. In this work, the implementation from the K.U. Leuven as shipped with SWI-Prolog is used\cite{swi_prolog}. This version implements the so-called \emph{refined operational semantics}. Constraints in the goal are processed from left to right. When entering the constraint store, a CHR constraint becomes \emph{active}, i.e. each rule, which contains the active constraint in the head, is checked for applicability. The order of the rule applicability checks is from top to bottom. The first rule that matches is fired. If the active constraint is removed by the rule application, the next constraint from the body will be added and become active. Otherwise, if the active constraint is kept, the next rule below the first matching rule is checked for applicability, etc. If no rule matches, the constraint becomes passive and is actually put to the constraint store, where it waits to become the partner constraint in a matching rule and the next constraint from the query is added. A passive constraint becomes active again, if its containing variables are bound. 

If a rule fires, its body is processed from left to right and behaves like a procedure call: It will add its constraints one after another (and they will become active sequentially). When all constraints from the body have been added and no constraint is active anymore, the next constraint from the query is added. If no constraints are left in the query, the program terminates and the final constraint store is printed.

In the implementation of ACT-R, it was necessary at some points to rely on the order of rule applications from top to bottom. This is very common, for instance, if in an advanced version of the minimum computation in example~\ref{ex:minimum} the process should be triggered by a \lstinline|get_min/1| constraint that also gets the result bound to its argument. I.e., after the complete computation of the query in listing~\ref{lst:sample_derivation_min} triggered by \lstinline|get_min(Min)|, the variable \lstinline|Min| should be bound to \lstinline|Min=0|, the minimal number in the query. This can be achieved as follows:

\begin{lstlisting}[caption={Minimum program with trigger}]
get_min(_), min(N) \ min(M) <=> N=<M | true.
get_min(Min), min(N) <=> Min=N.
\end{lstlisting}

Obviously, the result of this program is dependent on the rule application order: The second rule is applicable as soon as the minimum computation has been triggered by \lstinline|get_min| and one single minimum candidate \lstinline|min(N)|. After the application of the second rule, the first one will not be applicable anymore, because the \lstinline|get_min/1| constraint is removed from the store. Hence, it must be ensured, that this rule is not applied before the first rule has not been applied to exhaustion. This is achieved by the refined operational semantics. The program computes the minimum of all numbers that are in the store at the time the \lstinline|get_min/1| trigger constraint is introduced. If there are added new minimum candidates afterwards, they will not be respected in this particular minimum computation. However, if later on a new minimum computation is triggered, all those minimum constraints will be part of the computation.



\chapter{Implementation of ACT-R in CHR}

After the comprehensive but at some point informal overview of the ACT-R theory in chapter \ref{actr_description}, this chapter presents a possible implementation of the described concepts of the ACT-R theory in CHR.

For the implementation, some special cases and details that are not exactly defined in theory have to be considered. Hence, some concepts of the theory that are implemented in this work are formalized first. The implementation in form of CHR rules sticks to those formalisms and is often very similar to them.

Additionally, the implementation is described incrementally, ie. first, a very minimal subset of ACT-R is presented that will be refined gradually with the progress of this chapter. In the end, an overview of the actual implementation as a result of this work is given.

Some of the definitions in this chapter result directly from the theory, some of them needed a further analysis of the official ACT-R 6.0 Reference Manual \cite{actr_reference} or the tutorials \cite{actr_tutorial}. 

\section{Declarative and Procedural Knowledge}

The basic idea of the implementation is to represent declarative knowledge, working memory etc. as constraints and to translate the ACT-R production rules to CHR rules. This approach leads to a very compact and direct translation of ACT-R models to Constraint Handling Rules.

In addition to the production rules there will be rules that implement parts of the framework of ACT-R, for example rules that implement basic chunk operations like modifying or deleting chunks from declarative memory or a buffer. Those parts of the system are described as well as the central data structures and the translation.

First, a formalization of declarative knowledge in form of chunk networks and their implementation in CHR is given. Then, the working memory -- also referred to as the buffer system -- is explored and the implementation is discussed. After those definitions of the basic data structures of ACT-R, the procedural system is described including the translation of ACT-R production rules to CHR rules using the previously defined data structures.

Furthermore, the reproduction of ACT-Rs modular architecture is shown and the implementation of the declarative module is presented.

After this overview of the basic concepts of ACT-R, the description goes into more detail about timing issues and the subsymbolic layer.

\section{Chunk Stores}

Since chunks are the central data-structure of ACT-R used for representation of declarative knowledge and to exchange information between modules and to state requests, this section first deals with this central part of ACT-R.

\subsection{Formal Representation of Chunks}

In multiple parts of ACT-R it is necessary to store chunks and then operate on then. Hence, the abstract data structure of such a chunk store is defined.

Since chunk stores have been referred to as networks in the previous chapters, the general idea of this definition of a chunk store bases upon a relation that represents such a network.

\begin{definition}[chunk-store]
A \emph{chunk-store} $\Sigma$ is a tuple $(C,E,\mathcal{T},HasSlot,Isa)$, where $C$ is a set of chunks and $E$ a set of primitive elements, with $C \cap E = \emptyset$. $V = C \cup E \cup \{ \mathtt{nil} \}$ are the \emph{values} of~$\Sigma$ and $\mathcal{T}$ a set of chunk-types. A chunk-type $T = (t,S) \in \mathcal{T}$ is a tuple with a \emph{unique}\footnote{$\forall (t,S), (t',S') \in \mathcal{T}: t = t' \Rightarrow S = S'$ } type name $t$ and a set of slots $S$. The set of all slot-names is $\mathcal{S}$. 

$HasSlot \subseteq C \times \mathcal{S} \times V$ and $Isa \subseteq C \times T$ are relations and are defined as follows:

\begin{itemize}
 \item $c \enspace Isa \enspace T \Leftrightarrow$ chunk $c$ is of type $T$.
 \item $(c,s,v) \in HasSlot \Leftrightarrow v$ is the value of slot $s$ of $c$. This can also be written as $c \overset{s}{\longrightarrow} v$ and is spoken ``$c$ is connected to $v$'' or ``$v$ is in the slot $s$ of $c$''.
\end{itemize}

The $Isa$ relation has to be right-unique and left-total, so each chunk has to have exactly one type.

% The following function is defined:
% 
% \begin{align*}
%  %\item $slots: C \rightarrow \mathcal{S} \times V$ \\
%  %$slots(c) = \{ (s,v) | (c,s,v) \in HasSlot \}$ 
%  slots: \mathcal{T} \rightarrow \mathcal{S} &\\
%        slots((t,S)) = S&
% \end{align*}


A chunk-store is \emph{type-consistent}, iff $\forall (c,(t,S)) \in Isa: \forall s \in S \enspace \exists ! (c,s,v) \in HasSlot$. So every chunk must have exactly one value for each slot of its type and only describe slots of its type. Empty slots are represented by the value \verb|nil|. Since every chunk has exactly one type, this is valid for all chunks in the store.

%iff $\forall(c,s,v) \in HasSlot: c \enspace Isa \enspace (T,S) \Rightarrow \exists s' \in S: s=s'$ and $\forall c \in C: \enspace c \enspace Isa \enspace (T,S) \Rightarrow \forall s \in S: \exists (c',s',v) \in HasSlot: c=c', s=s'$. 

\end{definition}


\begin{definition}[abstract methods of a chunk store]
\label{def:abstract_methods_chunk_store}
The following methods can be defined over a chunk store $\Sigma = (C, E, \mathcal{T}, HasSlot, Isa)$:

\begin{description}
 \item[\texttt{chunk-type(name slot\textsubscript{1} slot\textsubscript{2} ... slot\textsubscript{n})}] adds the type $T = (\mathtt{name},\{\mathtt{slot_1}, \dots, \mathtt{slot_n}\})$ to the store, ie. $\mathcal{T'} = \mathcal{T} \cup \{T\}$. 
 \item[\texttt{add-chunk(name isa type slot\textsubscript{1} val\textsubscript{1} ... slot\textsubscript{n} val\textsubscript{n})}] adds a new chunk to the store, ie. $C' = C \cup \{ \mathtt{name} \}$, $Isa' = Isa \cup (\mathtt{name}, (\mathtt{type}, slots(\mathtt{type}))$ and $HasSlot' = \bigcup_{i = 1}^n{\mathtt{(name,slot_i,val_i)}} \cup HasSlot.$ Note, that due to the expansion of $C$, the condition that $C$ and $E$ have to be disjoint may be violated. To fix this violation, the element can be removed from $E$: $E' = (E \cup C) - (E \cap C).$ 
 
 Additionally, a valid mechanism to restore type-consistency may be introduced: It might happen, that not all slots are specified in the call of the \verb|add-chunk| method. Since it is claimed by the definition of $HasSlot$ that for all slots $s$ of a chunk $c$ there must be a $(c,s,v) \in HasSlot$, in implementations the unspecified slots are initialized as empty slots, represented by the empty value \verb|nil|. Furthermore, slots specified in the call of the method that are not a member of the chunk's type should cause an error to preserve type-consistency.
  \item[\texttt{alter-slot(name slot\textsubscript{1} val\textsubscript{1} ... slot\textsubscript{n} val\textsubscript{n})}] changes the slot values of a chunk identified by its name. Only existing slots can be altered.
  \item[\texttt{remove-chunk(name)}] removes the chunk with the given name from $C$ and all of its occurrences in $Isa$ and $HasSlot$.
  \item[\texttt{return-chunk(name)}] gets a chunk name as input and returns a chunk specification, ie. the name, type and all slot-value pairs of this chunk in the store.
\end{description} 
\end{definition}

\begin{example}
 \label{ex:addition_fact_formal}
 The addition-fact chunk in figure \ref{fig:chunk_addition_fact} and its chunk-type are defined as follows:\\
\begin{lstlisting}
chunk-type(addition-fact arg1 arg2 sum)
add-chunk(a isa addition-fact arg1 5 arg2 2 sum 7)
\end{lstlisting}
 
 This leads to the following chunk-store: 
 \begin{align*}
 (\{a\}, \{2,5,7\},\\ 
 \{(addition-fact, \{arg1, arg2, sum\})\},\\
 \{(a,arg1,5), (a,arg2,2), (a,sum,7)\},\\
 \{(a, addition-fact)\}).
 \end{align*}
 
 $slots(a) = \{(arg1,5), (arg2,2), (sum,7)\}$ and $slots((addition-fact,\{arg1, arg2, sum\}) = \{arg1, arg2, sum\}$. Hence, the store is type-consistent.
\end{example}

\subsection{Representation of Chunks in CHR}

Declarative knowledge is represented as a network of chunks, defined by the two relations $Isa$, specifying the belonging of a chunk to a type, and $HasSlot$, specifying the slot-value pairs of a chunk. Those relations can be translated directly into CHR by defining the following constraints representing the relations and sets:

\begin{lstlisting}
:- chr_constraint chunk_type(+).
% chunk_type(ChunkTypeName)

:- chr_constraint chunk_type_has_slot(+,+).
% chunk_type_has_slot(ChunkTypeName, SlotName).
\end{lstlisting}

The \verb|chunk_type/1| constraint represents the set $\mathcal{T}$ of chunk-types in the store, but refers only to the chunk-type names. The set of slots of a chunk-type is specified by the \verb|chunk_type_has_slot/2| constraint\footnote{For a chunk-type $T \in \mathcal{T}$, with $T = (t, S)$, there exists a \texttt{chunk\_type(t)} and for every slot $s \in S$ there is a \texttt{chunk\_type\_has\_slot(t,s)} in the constraint store.}.

For the chunks:

\begin{lstlisting}
:- chr_constraint chunk(+,+).
% chunk(ChunkName, ChunkType)

:- chr_constraint chunk_has_slot(+,+,+).
% chunk_has_slot(ChunkName, SlotName, Value)
\end{lstlisting}

The \verb|chunk/2| constraint represents both the set $C$ of chunks and the $Isa$ relation, since the presence of a constraint \verb|chunk(c,t)| signifies, that chunk \verb|c| is of a type $T = (\mathtt{t},S)$.

The $HasSlot$ relation is represented by the \verb|chunk_has_slot(c,s,v)| constraint, which really is just a direct translation of an element $(c,s,v) \in HasSlot$.

Note that all values in the just presented constraints have to be ground. This is a demand claimed by the original ACT-R implementation and makes sense, since each value in a slot of a chunk is a real, ground value and the concept of variables does not have an advantage in this context, because every element that can be stored in the brain is assumed to be known by the brain.

Additionally, from the definition of a chunk store it is known, that the $HasSlot$ and the $Isa$ relations have to be left-complete. \FIXME{correct expression! correct definitions!} Therefore, for every chunk \verb|c| in the store, exactly one \verb|isa(c,t)| constraint has to be in the store. For each \verb|chunk_type_has_slot(t,s)| constraint, a \verb|chunk_has_slot(c,s,v)| constraint has to be defined. If one wants to express, that a chunk has an empty slot, he might use \verb|nil| for the value to indicate that. Note that \verb|nil| must not be a chunk name or chunk-type name.

\begin{example}
\label{ex:addition_fact_chr}
The chunk and chunk-type in example~\ref{ex:addition_fact_formal} are represented as:

\begin{lstlisting}
chunk_type(addition-fact)
chunk_type_has_slot(addition-fact,arg1)
chunk_type_has_slot(addition-fact,arg2)
chunk_type_has_slot(addition-fact,sum)

chunk(a,addition-fact)
chunk_has_slot(a,arg1,5)
chunk_has_slot(a,arg2,2)
chunk_has_slot(a,sum,7)
\end{lstlisting}
\end{example}


\subsubsection{Distinction of Elements and Chunks}

A chunk store distinguishes between a set of chunks $C$ and a set of elements $E$. For implementational reasons it can be helpful if there are only chunks in the system, because elements just behave like chunks with no slots. Hence, a chunk-type \verb|chunk| with no slots will be added automatically to the store. Each element $e \in E$ and added as a chunk of type \verb|chunk| to the set of chunks $C$. After this operation $E = \emptyset$, and for every former element $e$ of $E$: $e \in C$, $(e,(chunk,\emptyset)) \in Isa$.

So $E$ is represented now by $\{ c \in C | c Isa (chunk,\emptyset)$ in the implementation.

\begin{example}
The chunk representation from example~\ref{ex:addition_fact_chr} is changed to:

\begin{lstlisting}
chunk_type(addition-fact)
chunk_type_has_slot(addition-fact,arg1)
chunk_type_has_slot(addition-fact,arg2)
chunk_type_has_slot(addition-fact,sum)

chunk_type(chunk)

chunk(a,addition-fact)
chunk_has_slot(a,arg1,5)
chunk_has_slot(a,arg2,2)
chunk_has_slot(a,sum,7)

chunk(5,chunk)
chunk(2,chunk)
chunk(7,chunk)
\end{lstlisting}

\end{example}


\subsubsection{Simple Implementation of the Default Methods}
\label{chunk_specification}

To implement the methods in definition~\ref{def:abstract_methods_chunk_store}, first a data type for chunk specifications has to be introduced. From this specification the correct constraints modeling the chunk-store are added or modified.

The straight-forward definition of a data type for chunk specifications is just to use the specification like in definition~\ref{def:abstract_methods_chunk_store}: Since \verb|(name isa type slot_1 val_1 \dots slot_n val_n)| is just a list in LISP and specifies a chunk uniquely, a similar Prolog term can be used:

\begin{lstlisting}
:- chr_type chunk_def ---> nil; chunk(any, any, slot_list).
:- chr_type list(T) ---> []; [T | list(T)].
% a list of slot-value pairs
:- chr_type slot_list == list(pair(any,any)).
:- chr_type pair(T1,T2) ---> (T1,T2).
\end{lstlisting}

This definition states that a chunk is either \verb|nil|, ie. an empty chunk, or a term \verb|chunk(Name, Type, SVP)|, where \verb|SVP| is a list of slot-value pairs. This is the direct translation of the chunk-specification used in the definition, amended by the \verb|nil| construct, that may be needed for later purposes.

The default methods can be implemented as follows:

\paragraph{add\_chunk}

This method creates the chunks and elements of the chunk store. The set $E$ of elements is minimal, ie. only elements that appear in the slots of a chunk but are not chunks themselves are members of $E$. However, the set $E$ is never constructed explicitly, but represented by chunks of the special type \verb|chunk| that provides no slots. So each value in the slot of a chunk that is added to the store and that is not an element of the chunk store yet, gets its own chunk of type \verb|chunk|. As soon as a chunk with the name of such a primitive element is added to the store, the chunk of type \verb|chunk| is removed from the store.

\begin{lstlisting}[caption={Rules for \texttt{add\_chunk}}, label=lst:add_chunk_rules]
% empty chunk will not be added
add_chunk(nil) <=> true.
  
% initialize all slots with nil
add_chunk(chunk(Name,Type, _)), chunk_type_has_slot(Type,S) ==> 
  chunk_has_slot(Name,S,nil).

% chunk has been initialized with empty slots -> actually add chunk
add_chunk(chunk(Name,Type, Slots)) <=>
  do_add_chunk(chunk(Name,Type,Slots)).
\end{lstlisting}

First, all \verb|chunk_type_has_slot| constraints are added to the store and initialized with \verb|nil| as slot value. This leads to complete chunk specifications that are consistent to the type as demanded by a type-consistent chunk-store.

If all slots have been initialized, \verb|do_add_chunk| performs the actual setting of the real slot values:
  
\begin{lstlisting}[caption={Additional rules for adding chunks}]  
% base case
do_add_chunk(chunk(Name, Type, [])) <=> chunk(Name, Type). 

% overwrite slots with empty values
chunk(V,_) \ do_add_chunk(chunk(Name, Type, [(S,V)|Rest])), chunk_has_slot(Name,S,nil)  <=>
  chunk_has_slot(Name,S,V), 
  do_add_chunk(chunk(Name,Type,Rest)).

% overwrite slots with empty values  
do_add_chunk(chunk(Name, Type, [(S,V)|Rest])), chunk_has_slot(Name,S,nil)  <=> 
  V == nil | % do not add chunk(nil,chunk)
  chunk_has_slot(Name,S,V), 
  do_add_chunk(chunk(Name,Type,Rest)).  

% overwrite slots with empty values  
do_add_chunk(chunk(Name, Type, [(S,V)|Rest])), chunk_has_slot(Name,S,nil)  <=> 
  V \== nil |
  chunk_has_slot(Name,S,V), 
  chunk(V,chunk), % no chunk for slot value found => add chunk of type chunk 
  
do_add_chunk(_) <=> false.
\end{lstlisting}

The first rule is the base case, where no slots have to be added any more. Then, as a last step the actual \verb|chunk| constraint of the chunk that is added to the store is created.

The second rule deals with the case, that a slot-value pair has to be added with a value that is already described by a chunk. Then the \verb|nil|-initialized slot of this chunk is removed and replaced by another slot containing the actual value.

The next rule ensures that the helper chunk specification \verb|nil| will not get a chunk in the store, even if it is in the slots of a chunk.

Otherwise, if the value of the slot to be added is not \verb|nil|, the next rule can fire and the slot with the actual value will replace the previously \verb|nil|-initialized slot with the actual value. Additionally, since the first rule obviously did not fire for this constellation, \verb|V| is a value different from \verb|nil| that does not have a chunk in the store. Hence, it must be a primitive element. Thus a new chunk of type \verb|chunk| is added to the store for this value.

If no rule matches, the user tried to create a chunk with slots that are not specified in the chunk-type. This leads to an error.

On top of the rules in listing~\ref{lst:add_chunk_rules}, there must be added a rule that deletes a primitive element (ie. a chunk of type \verb|chunk|), if the user introduces a real chunk with the name of this element:

\begin{lstlisting}[caption={Clean up primitive elements}]  
% delete chunk of Type chunk, if real chunk is added
add_chunk(chunk(Name,_,_)) \ chunk(Name,Type) <=> 
  Type == chunk |
  true.
\end{lstlisting}


\paragraph{add\_chunk\_type}

The following rules create a new chunk type:

\begin{lstlisting}[caption={rules for \texttt{add\_chunk\_type}}]
add_chunk_type(CT, []) <=> 
  chunk_type(CT).
add_chunk_type(CT, [S|Ss]) <=> 
  chunk_type_has_slot(CT, S), 
  add_chunk_type(CT, Ss).
\end{lstlisting}

\paragraph{alter\_slot}

This method replaces the value of an existing slot for a given chunk, but only if it is a valid slot for the chunk-type of the altered chunk.

\begin{lstlisting}
alter_slot(Chunk,Slot,Value), chunk_has_slot(Chunk,Slot,_) <=>
  chunk_has_slot(Chunk,Slot,Value).
  
alter_slot(Chunk,Slot,Value) <=>
  false.
\end{lstlisting}

The first rule replaces the existing \verb|chunk_has_slot| constraint by a new one. This is called \emph{destructive assignment} as described in \cite[32]{fru_chr_book_2009}. The second rule only matches, if the first did not match (due to the refined operational semantics of CHR). This is only the case, if it is tried to alter a slot with a non-existing \verb|chunk_has_slot| constraint. However, since the chunk descriptions are complete, the slot cannot be valid for the type of the chunk and the altering has to fail.

\paragraph{remove\_chunk}

This method removes all occurrences of a chunk.

\begin{lstlisting}
remove_chunk(Name) \ chunk(Name, _) <=> true.
remove_chunk(Name) \ chunk_has_slot(Name, _, _) <=> true.
remove_chunk(_) <=> true.
\end{lstlisting}

\paragraph{return\_chunk}

This method creates a chunk specification as defined in section~\ref{chunk_specification} from the chunk name of a chunk in the store.

\begin{lstlisting}[caption={rules for \texttt{return\_chunk}}]
chunk(ChunkName, ChunkType) \ return_chunk(ChunkName,Res) <=> 
  var(Res) | 
  build_chunk_list(chunk(ChunkName, ChunkType, []),Res).

chunk_has_slot(ChunkName, S, V) \ build_chunk_list(chunk(ChunkName, ChunkType, L), Res) <=> 
  \+member((S,V),L) | 
  build_chunk_list(chunk(ChunkName, ChunkType, [(S,V)|L]),Res).
  
build_chunk_list(X,Res) <=> Res=X.
\end{lstlisting}

The first rule creates the initial chunk specification with name and type set, but without any slot specification. This initial representation is handed to the \verb|build_chunk_list| constraint.

The second rule adds a slot-value pair from the store to the list of slot-value pairs in the specification and builds the next chunk specification from this new representation.

In the last rule, the process terminates, if no other rule can fire any more. Then the result is bound to the handed specification.

\subsubsection{Checking Consistency and Type-Consistency}

At the moment, there are no rules that check the consistency of the chunk store. However, if the default methods for adding chunks are used, a type-consistent store is built automatically, since for every chunk has exactly one chunk-type\footnote{left-totality and right-uniqueness of $Isa$} and all slots from its chunk-type are described and only those slots are described (satisfies type-consistency). Additionally, there are no two different slot descriptions for the same chunk and every chunk in the store is described\footnote{This is demanded by the type-consistency: Since $Isa$ is left-total, every chunk is in the $Isa$ relation. Type-consisteny demands, that every chunk in the $Isa$ relation has a value for all slots of its type.} (satisfies the definition of a chunk-store).

Rules for checking those constraints could be added easily to the implementation.

\section{Procedural Module}

The part of the system, where the computations are performed, is the procedural module. It is the central component, that holds all the production rules, the working memory (in the buffer system) and organizes communication between modules (through buffers and requests). In the following, all of those subcomponents of the procedural module are described.

\subsection{Buffer System}

The buffer system can be regarded as a chunk-store, that is enhanced by buffers. A buffer can hold only one chunk at a time. The procedural module has a set $B$ of buffers, a chunk-store $\Sigma$ and a relation between the buffers and the chunks in $\Sigma$.

\begin{definition}[buffer system]
\label{def:buffer_system}
A \emph{buffer system} is a tuple $(B,\Sigma,Holds)$, where $B$ is a set of buffers, $\Sigma = (C, E, \mathcal{T}, HasSlot, Isa)$ a type-consistent chunk-store and $Holds \subseteq B \times (C \cup \{ \mathtt{nil} \})$ a right-unique and left-total relation, that assigns every buffer at most one chunk that it holds. If a buffer $b$ is empty, ie. it does not hold a chunk, then $(b,\mathtt{nil}) \in Holds$.

A buffer system is \emph{consistent}, if every chunk that appears in $Holds$ is a member of $C$ and $\Sigma$ is a type-consistent chunk-store.

A buffer system is \emph{clean}, if its chunk-store only holds chunks which appear in $Holds$.
\end{definition}

For the implementation of a buffer system, the code of a chunk-store can be extended by a \verb|buffer/2| constraint, that encodes the set $B$ and the relation $Holds$ at once, since the relation is complete by definition\footnote{$\forall b \in B \enspace \exists c \in (C \cup \{ \mathtt{nil} \}): (b,c) \in Holds$}.

\subsubsection{Destructive Assignment and Consistency}
\label{destructive_assignment}

The demand of $Holds$ being right-unique\footnote{$\forall b \in B \enspace \forall c, d \in (C \cup \{ \mathtt{nil} \}): (b,c), (b,d) \in Holds \Rightarrow b = c$} is a form of destructive assignment as described in \cite[p. 32]{fru_chr_book_2009}, ie. if a new chunk is assigned to a buffer, the old \verb|buffer| constraint is removed and a new \verb|buffer| constraint is introduced, holding the new chunk:

\begin{lstlisting}
set_buffer(B, C) \ buffer(B, _) <=> buffer(B, C).
\end{lstlisting}

This rule ensures that only one \verb|buffer| constraint exists for each buffer in $B$.

At the beginning of the program, a \verb|buffer| constraint has to be added for all the available buffers of the modules. This problem is discussed in section~\ref{initialization}.

In addition, if a new chunk is put into a buffer, it also has to be present in the chunk-store, since the production system relies on the knowledge about the chunks in its buffers and chunks are essentially defined by their slots (\emph{consistency property} in definition~\ref{def:buffer_system}). Hence, every time a chunk is stored in a buffer, the \verb|add_chunk| method described in definition~\ref{def:abstract_methods_chunk_store} has to be called. However, the chunks in the slots are not loaded, since they are not present for working memory. They only appear as primitive elements in the chunk-store. Figure~\ref{fig:buffer_system}

%This process is discussed later when talking about buffer requests in section~\ref{buffer_requests}.

\subsubsection{Buffer States}

Another formal detail of the buffer system is that buffers can have various states: \emph{busy}, \emph{free} and \emph{error}. A module is busy, if it is completing a request and free otherwise.

Since a module can only handle one request at a time and requests may need a certain time (like the retrieval request for example), the procedural module could state another request to a busy module. This is called \emph{jamming} which leads to error messages and should be avoided. One technique to avoid module jamming is to \emph{query} the buffer state in the conditional part of a production rule \cite[unit 2, p. 9]{actr_tutorial}. The possibility to query buffer states is discussed in the next section.

A buffer's state is set to \emph{error}, if a request was unsuccessful because of an invalid request specification or, in case of the declarative module for instance, a chunk that could not be found.

In CHR, a buffer state can be represented by a \verb|buffer_state(b,s)| constraint, which signifies that buffer \verb|b| has the state \verb|s|. Since every buffer has exactly one state all the time, it is required, that for every buffer there is such a constraint and it is ensured, that only one \verb|buffer_state| constraint is present for each buffer. This can be achieved by the destructive assignment method described in section~\ref{destructive_assignment}. 

At the beginning of the program, when a buffer is created (a \verb|buffer| constraint is placed into the store), a corresponding \verb|buffer_state| constraint has to be added. The initial state can be set to \verb|free|, since no request is being computed at the time of creation.

\subsection{Production Rules}

Production rules consist of a \emph{condition} part and an \emph{action} part. Syntactically, in ACT-R the condition is separated from the action by \verb|==>|. Additionally, each production rule has a name. Thus, a rule is defined by:

\begin{verbatim}
(p name condition* ==> action*)
\end{verbatim}

The condition part is also called the \emph{left hand side} of a rule (LHS) and the action part is called \emph{right hand side} (RHS).

\subsubsection{The Left Hand Side of a Rule}

Generally, a condition is either a \emph{buffer test}, ie. a specification of slot-value pairs that are checked against the chunk in the specified buffer or a \emph{buffer query}, ie. a check of the state of a buffer's module (either busy, free or error). A buffer test on the LHS of a rule is indicated by a \verb|=| followed by the buffer name of the tested buffer; a query is indicated by a \verb|?| in front of the buffer name.

The LHS of a rule may contain bound or unbound variables: \verb|=varname| is a variable with name \verb|varname|.

If the chunks in the buffers pass all buffer tests specified by the rule, the rule can fire, ie. its right hand side will be applied. The LHS is a conjunction of buffer tests, ie. there is no specific order for the tests \cite[p. 165]{actr_reference}.

\begin{example}[counting example -- left hand side]
The left hand side of the counting rule specified in the example in section~\ref{example_counting} could be defined as follows:

\begin{lstlisting}
(p count-rule
  =goal> 
    isa    count
    number =n
  =retrieval>
    isa    count-fact
    first  =n
    second =m
==>
  ... )
\end{lstlisting}

The condition part consists of two buffer tests:

\begin{enumerate}
 \item The goal buffer is tested for a chunk of type \verb|count| and a slot with name \verb|number|. The value of the slot is bound to the variable \verb|=n|.
 \item The retrieval buffer is tested for a chunk of type \verb|count-fact| that has the variable \verb|=n| in its \verb|first| slot (with the same value as the \verb|number| slot of the chunk in the goal buffer, since \verb|=n| has been bound to that value), and another value in its second slot which is bound to the variable \verb|=m|.
\end{enumerate}

\end{example}

\subsubsection{The Right Hand Side of a Rule}

For the right hand side of a rule the following actions are allowed:

\begin{description}
 \item[Buffer Modification] 
 \item[Buffer Request]
 \item[Buffer Clearing]
\end{description}


\begin{example}[counting example]
\label{ex:counting}
The counting rule specified in the example in section~\ref{example_counting} could be defined as follows:

\begin{lstlisting}
(p count-rule
  =goal> 
    isa    count
    number =n
  =retrieval>
    isa    count-fact
    first  =n
    second =m
==>
  =goal>
    number =m
  +retrieval>
    isa    count-fact
    first  =m
)
\end{lstlisting}

\FIXME{add description}

\end{example}

\subsubsection{Direct Translation of Buffer Tests}

An ACT-R production rule of the form

\begin{lstlisting}[mathescape]
(p name
  =buffer$_\mathtt{1}$>
    isa    type$_1$
    slot$_\mathtt{1,1}$  val$_\mathtt{1,1}$
    ...
    slot$_\mathtt{1,n}$  val$_\mathtt{1,n}$
  ...
  =buffer$_\mathtt{k}$>
    isa    type$_\mathtt{k}$
    slot$_\mathtt{k,1}$  val$_\mathtt{k,1}$
    ...
    slot$_\mathtt{k,m}$  val$_\mathtt{k,m}$
==>
... )
\end{lstlisting}

states formally, that:

If $buffer_1 \enspace Holds \enspace c_1 \enspace \wedge \enspace c \enspace Isa \enspace type_1 \enspace \wedge \enspace c_1 \xrightarrow{slot_{1,1}} val_{1,1} \enspace \wedge \enspace \dots \enspace \wedge \enspace buffer_k \enspace Holds \enspace c_k \enspace \wedge \enspace c_k \enspace Isa \enspace type_k \enspace \wedge \enspace \dots$ is true, then the rule matches and the RHS should be performed.

This can be directly translated into a CHR rule:

\begin{lstlisting}[mathescape]
name @
  buffer(buffer$_\mathtt{1}$,C$_\mathtt{1}$),
  chunk(C$_\mathtt{1}$,type1),
  chunk_has_slot(C$_\mathtt{1}$,slot$_\mathtt{1,1}$,val$_\mathtt{1,1}$),
  ...
  chunk_has_slot(C$_\mathtt{1}$,slot$_\mathtt{1,n}$,val$_\mathtt{1,n}$),
  ...
  buffer(bufferk,C$_\mathtt{k}$),
  chunk(C$_\mathtt{k}$,type$_\mathtt{k}$),
  chunk_has_slot(C$_\mathtt{k}$,slot$_\mathtt{k,1}$,val$_\mathtt{k,1}$),
  ...
  chunk_has_slot(C$_\mathtt{k}$,slot$_\mathtt{k,m}$,val$_\mathtt{k,m}$)
==>
  ...
\end{lstlisting}

This rule checks the buffer system for the existence of a buffer holding a particular chunk and then checks the chunk store of the buffer system for that chunk with the type and slots specified in the ACT-R rule. The rule is a propagation rule, because the information of the chunk-store should not be removed.

If the values in the slot tests are variables, they can be directly translated to Prolog variables.

The CHR rule only fires, if all the checked buffers hold chunks that meet the requirements specified in the slot tests of the ACT-R rule. Since those slot-tests are just a conjunction of relation-membership tests and the CHR rule is a translation of these tests into constraints, both are equivalent. In detail: \FIXME{geht besser}

\begin{itemize}
 \item If a checked buffer \verb|b| holds no chunk, the constraint \verb|buffer(b,nil)| will be present, but the chunk store will not hold any of the required \verb|chunk| or \verb|chunk_has_slot| constraints and the rule will not fire.
 \item If a checked buffer \verb|b| holds a chunk, but the chunk does not meet one of the requirements in its slots, the rule does not fire.
 \item The rule only fires, if for all checked buffers there are valid \verb|buffer|, \verb|chunk| and \verb|chunk_has_slot| constraints present that meet all the requirements specified by the ACT-R rule.
 \item Variables on the LHS of a rule are bound to the values of the actual constraints that are tried for the matching. After the matching, each variable from the rule has a ground value bound to it, because there are no variables in the implementation of the buffer store. This corresponds to the semantics of a ACT-R production rule with variables on the LHS.
\end{itemize}

\FIXME{ist die begründung schlüssig? evtl section über variablen hier einfügen}

\subsubsection{Translation of Actions}
\label{translation_of_actions}

For each action type a constraint

\begin{lstlisting}
buffer_action(buffer,chunk-specification)
\end{lstlisting}

with corresponding rules that handle the action have to be added. Note, that the buffer action is defined by the action name, the buffer name and a chunk specification, because actions in ACT-R are defined through a buffer modifier that specifies the action, the name of the buffer and a list of slot-value pairs. Hence, this is a direct translation to CHR.

\paragraph{Buffer Modifications}

The modification of a buffer takes an incomplete chunk specification and modifies the given slots of the chunk in the specified buffer. This can be implemented as follows:

\begin{lstlisting}
buffer(BufName, OldChunk) \ buffer_change(BufName, chunk(_,_,SVs)) <=>
  alter_slots(OldChunk,SVs).
\end{lstlisting}

This implementation uses a generalization of the \verb|alter_slot| method as described in definition~\ref{def:abstract_methods_chunk_store} and section~\ref{chunk_specification}:

\begin{lstlisting}  
alter_slots(_,[]) <=> true.
alter_slots(Chunk,[(S,V)|SVs]) <=> 
  alter_slot(Chunk,S,V),
  alter_slots(Chunk,SVs). 
\end{lstlisting}

Note that types cannot be changed. This corresponds to the grammar definition of ACT-R as presented in section~\ref{production_rule_grammar}.

\paragraph{Buffer Clearings}

When clearing a buffer, the chunk that was stored in the buffer will be removed from the chunk store and \verb|nil| will be written to the store. Additionally, the chunk goes to the declarative memory.

\begin{lstlisting}
do_buffer_clear(BufName), buffer(BufName, ModName, Chunk) <=> 
  write_to_dm(Chunk), 
  delete_chunk(Chunk), 
  buffer(BufName, nil).
\end{lstlisting}

\verb|write_to_dm| handles the writing of the chunk to the declarative memory:

\begin{lstlisting}
write_to_dm(ChunkName) <=> return_chunk(ChunkName, ResChunk), add_dm(ResChunk).
\end{lstlisting}

where \verb|add_dm| is basically just a global wrapper for the \verb|add_chunk| method of the declarative memory and will be explained in section~\ref{global_method_for_adding_chunks}.

\paragraph{Buffer Requests} The buffer requests have to be handled a little bit differently from the other actions. Therefore, they will be explained in section~\ref{interface_for_module_requests}. The changes to the buffer system are presented in section~\ref{how_the_buffer_system_states_a_request} and an example implementation of the module request interface for retrieval requests is given in section~\ref{retrieval_requests}.


\begin{example}[counting example in CHR -- simple]
The production rule in example~\ref{ex:counting} can be translated to:

\begin{lstlisting}
count-rule @
  buffer(goal,C1), 
    chunk(C1,count),
    chunk_has_slot(number,N),
  buffer(retrieval,C2),
    chunk(C2,count-fact),
    chunk_has_slot(first,N),
    chunk_has_slot(second,M)
==>
  buffer_change(goal,chunk(_,_,[(number,M)])),
  buffer_request(retrieval,chunk(_,count-fact,[first,M)])).
\end{lstlisting}

\end{example}


\subsection{Translation of Buffer Queries}

A buffer query

\begin{lstlisting}
...
?buffer>
  state  bstate 
...
==> ...
\end{lstlisting}

on the LHS of a production rule can be translated to the following CHR rule head:

\begin{lstlisting}
...
buffer_state(buffer,bstate) ... ==> ...
\end{lstlisting}

\subsubsection{The Production Rule Grammar}

The discussed concepts lead to the following grammar for production rules, which is a simplified version of the actual grammar used in the original ACT-R implementation \cite[p. 162]{actr_reference}. 


\begin{lstlisting}{caption={The ACT-R production rule grammar} label=lst:production_rule_grammar}
production-definition ::= (p name condition* ==> action*)
name ::= a symbol that serves as the name of the production for reference
condition ::= [ buffer-test | query ]
action ::= [buffer-modification | request | buffer-clearing | output ]
buffer-test ::= =buffer-name> isa chunk-type slot-test*
buffer-name ::= a symbol which is the name of a buffer
chunk-type ::= a symbol which is the name of a chunk-type in the model
slot-test ::= {slot-modifier} slot-name slot-value
slot-modifier ::= [= | - | < | > | <= | >=]
slot-name ::= a symbol which names a possible slot in the specified chunk-type
slot-value ::= a variable or any Lisp value
query ::= ?buffer-name> query-test*
query-test ::= {-} queried-item query-value
queried-item ::= a symbol which names a valid query for the specified buffer
query-value ::= a bound-variable or any Lisp value
buffer-modification ::= =buffer-name> slot-value-pair*
slot-value-pair ::= slot-name bound-slot-value
bound-slot-value ::= a bound variable or any Lisp value
request ::= +buffer-name> isa chunk-type request-spec*
request-spec ::= {slot-modifier} slot-value-pair
request-parameter ::= a Lisp keyword naming a request parameter provided by the buffer specified
buffer-clearing ::= -buffer-name>
variable ::= a symbol which starts with the character =
output ::= !output! [ output-value ]
output-value ::= any Lisp value or a bound-variable
bound-variable ::= a variable which is used in the buffer-test conditions of the production (including a
variable which names the buffer that is tested in a buffer-test or dynamic-buffer-test) or is bound with
an explicit binding in the production
\end{lstlisting}

Some of the details in this grammar that have not been discussed yet are presented in the following.

\subsubsection{The Order of Rule Applications}

In the current translation scheme, the order of the rule applications does not match the semantics as described in the ACT-R theory\footnote{see section~\ref{procedural_knowledge}}. Consider the following rules in a compact notation:

\begin{lstlisting}
=b1>
  isa foo
  s1  v1
==>
=b1>
  s1  v2
=b2>
  s   x
\end{lstlisting}

and

\begin{lstlisting}
=b1>
  isa foo
  s1  v2
==>
=b2>
  s   y
=b1>
  s1  v3 % for termination
\end{lstlisting}

In the semantics of ACT-R, if the first rule matches, all the buffer modifications are performed first. After that, the procedural module can look for the next matching rule, which is the second one (due to the result of the first rule). This rule then would overwrite the value \verb|x| in the \verb|s| slot of buffer \verb|b2| with \verb|y|.

\begin{lstlisting}
buffer(b1,C),
  chunk(C,foo),
  chunk_has_slot(C,s1,v1)
==>
buffer_change(b1,chunk(_,_,[(s1,v2)])),
buffer_change(b2,chunk(_,_,[(s,x)])).

buffer(b1,C),
  chunk(C,foo),
  chunk_has_slot(C,s1,v2)
==>
buffer_change(b2,chunk(_,_,[(s,y)])),
buffer_change(b1,chunk(_,_,[(s1,v3)])). % this is for termination
\end{lstlisting}

For the query

\begin{lstlisting}
chunk(c,foo), chunk_has_slot(c,s1,v1), chunk(c2,bar), chunk_has_slot(c2,s,v), buffer(b2,c2), buffer(b1,c).
\end{lstlisting}

the result would be

\begin{lstlisting}
buffer(b1,c)
buffer(b2,c2)
chunk(c2,bar)
chunk(c,foo)
chunk_has_slot(c2,s,x)
chunk_has_slot(c,s1,v3)
\end{lstlisting}

Ie., the buffer \verb|b2| holds the chunk with value \verb|x|. This is due to the fact, that after the first rule has modified the buffer \verb|b1|, the second rule matches and will be fired immediately, which then changes the value of \verb|b1| to \verb|y|. Afterwards, the second action of the first rule will be executed and the \verb|s| slot of the chunk in buffer \verb|b2| will be overwritten with \verb|x|.

Hence, somehow the fact that the procedural module is busy and cannot fire another rule, has to be modeled. This can be achieved by a simple phase constraint \verb|fire| that will be added after the complete execution of a rule and will be removed as soon as a rule is executed.

For every production rule, the rule

\begin{lstlisting}
{ buffer_tests } ==> { actions }
\end{lstlisting}

has to be changed to a simpagation rule

\begin{lstlisting}
{ buffer_tests } \ fire <=> { actions }, fire.
\end{lstlisting}

It is important, that the adding of \verb|fire| is the last action of a rule.

The example would be modified as follows:

\begin{lstlisting}
buffer(b1,C),
  chunk(C,foo),
  chunk_has_slot(C,s1,v1)
  \ fire
<=>
buffer_change(b1,chunk(_,_,[(s1,v2)])),
buffer_change(b2,chunk(_,_,[(s,x)])),
fire.

buffer(b1,C),
  chunk(C,foo),
  chunk_has_slot(C,s1,v2)
  \ fire
<=>
buffer_change(b2,chunk(_,_,[(s,y)])),
fire.
\end{lstlisting}


The test query

\begin{lstlisting}
chunk(c,foo), chunk_has_slot(c,s1,v1), chunk(c2,bar), chunk_has_slot(c2,s,v), buffer(b2,c2), buffer(b1,c), fire.
\end{lstlisting}

yields

\begin{lstlisting}
buffer(b1,c)
buffer(b2,c2)
chunk(c2,bar)
chunk(c,foo)
chunk_has_slot(c,s1,v3)
chunk_has_slot(c2,s,y)
fire
\end{lstlisting}

The fire constraint has to be added at the end of the query. This ensures the correct semantics and a completely constructed buffer system.\footnote{This was actually the reason, why in the first example without the \texttt{fire} constraint, the buffer \texttt{b1} was created at the end of the query: If it would be created at an earlier point, the first rule would have matched immediately and the computation would have yielded a different result.}

In appendix~\ref{app:ex:rule_order} a minimal executable example is provided.

\subsubsection{Bound and Unbound Variables}
\label{bound_and_unbound_variables}

According to the ACT-R production rule grammar in listing~\ref{lst:production_rule_grammar}, unbound variables can only appear on the left hand side of a rule. Hence, no new variables are introduced on the right hand side. Since all the elements in the buffer store are completely described and ground, every variable on the LHS of a rule will be bound to a ground value. This simplifies the rule selection and application process a lot, since every value of the calculation is known after the matching. This enables simple implementations of arithmetic tests, for example (see section~\ref{slot_modifiers}).

\subsubsection{Double Chunk Checks}

A problem that has not been addressed yet, is that ACT-R allows buffer tests like the following:

\begin{lstlisting}
=buffer>
  isa  foo
  bar  spam
  bar  spam
\end{lstlisting}

In the logical reading, this would signify that $buffer \enspace Holds \enspace c \enspace \wedge \enspace c \enspace Isa \enspace foo \enspace \wedge \enspace c \xrightarrow{bar} spam \enspace \wedge \enspace c \xrightarrow{bar} spam$ which is equivalent to the test with only one check of the \verb|bar| slot\footnote{$x \wedge x = x$}.

However, in CHR the following rule resulting from the simple translation scheme would not match:

\begin{lstlisting}
buffer(buffer,C),
chunk(C,foo),
chunk_has_slot(C,bar,spam),
chunk_has_slot(C,bar,spam)
\end{lstlisting}

Because there are no two identical \verb|chunk_has_slot| constraints in the store, the rule could not fire. When translating such a rule, the second test has to be eliminated.

However, this does not the trick for all possible cases. For example, suppose a test

\begin{lstlisting}
=buffer>
  isa  foo
  bar  spam
  bar  =x
\end{lstlisting}

where the second test refers to a variable (or even both tests are variables). This problem could be solved by adding a guard to the rule from the simple translation:

\begin{lstlisting}
buffer(buffer,C),
chunk(C,foo),
chunk_has_slot(C,bar,spam)
==> spam == X
\end{lstlisting}

Since \verb|X| must be bound after the matching\footnote{see section~\ref{bound_and_unbound_variables}}, the test will always give the correct result. This also works for the first example, where the test would be \verb|spam == spam|, which is always true and hence could be reduced to the rule with the second test eliminated and no guard\footnote{true guards can always be reduced}.

It also works if the two slot-tests were contradictory, for example: 

\begin{lstlisting}
bar  spam
bar  eggs
\end{lstlisting}

would describe a rule that never can fire. The translation models exactly this behaviour:

\begin{lstlisting}
buffer(buffer,C),
chunk(C,foo),
chunk_has_slot(C,bar,spam)
==> spam == eggs
\end{lstlisting}

since the built-in syntactic-equality test of Prolog would never return true for those two constants.

\subsubsection{Slot Modifiers}
\label{slot_modifiers}

In ACT-R, slot-tests can be preceded by \emph{slot modifiers}. Those modifiers allow to specify tests like inequality (\verb|-|) or arithmetic comparisons (\verb|<, >, <=, >=|) of the slot value of a chunk with the specified variable or value. Since the slots in a chunk store are always fully defined with ground values, those tests are decidable.

If no slot modifier is specified in a slot test, the default modifier \verb|=| is used, that states that the chunk in the specified buffer must have the specified value in the specified slot. This default semantics has been used in the previous sections when translating simple ACT-R rules to CHR and is performed automatically by the matching of CHR.

To translate the other slot modifiers to CHR, another CHR mechanism can be used: Guards. Since the allowed modifiers are all default built-in constraints\footnote{ie. Prolog predicates}, a slot test with a modifier

\begin{lstlisting}
...
=buffer>
  ...
  ~slot  val
...
==>
\end{lstlisting}

where \verb|~| stands for one of the modifiers in \{ \verb|=,-,<,>,<=,>=| \} can be translated as follows:

\begin{lstlisting}
buffer(buffer,C),
  ...
  chunk_has_slot(C,slot,V),
  ...
==>
  V # val |
  ...
\end{lstlisting}

where \verb|#| is the placeholder for the built-in constraint that computes the test specified by \verb|~| and \verb|V| is a fresh variable that has not been used in the rule, yet.

For arithmetic slot modifiers the values being compared have to be numbers. If a value is not a number, the arithmetic test will fail and the rule cannot be applied \cite{actr_reference}.

Note, that slot tests with modifiers other than \verb|=| do not bind variables, but only perform simple checks, like it is with guards in CHR. If \verb|val| is an unbound variable and is never bound to a value on LHS, the default implementation throws a warning, and the rule will not match. Therefore, to handle this case, the rule translation scheme has to be extended by an additional guard check \verb|ground(Val)|, where \verb|Val| is the Prolog variable that replaces each occurrence of the variable \verb|val|.

As with normal slot tests, it is important to mention that if there are several tests on the same slot, the \verb|chunk_has_slot| constraint must appear only once on the LHS of the CHR rule, since every slot-value pair is unique in the constraint store. Ie., if the first slot test of a particular slot appears on the LHS of the ACT-R rule, a \verb|chunk_has_slot| constraint has to be added to the LHS of the CHR rule. For every other occurrence of this slot in a slot test, only guard checks are added.

\begin{example}
To clarify the details of the matching concept in ACT-R, here are some examples and their behaviour:

\begin{lstlisting}
=buffer>
  isa    foo
  -spam  =bar
\end{lstlisting}

will throw a warning when loading the model. When running it, the rule will never fire, since no chunk value will match the inequality to the unbound variable \verb|bar|.

\begin{lstlisting}
=buffer>
  isa    foo
  -spam  =bar
  eggs   =bar
\end{lstlisting}

will fire, if there is a chunk whose value in \verb|eggs| is different from the value in \verb|spam|.

\begin{lstlisting}
=buffer>
  isa    foo
  spam  =bar
  spam  =eggs
\end{lstlisting}

matches for every value of the spam slot. The translation to CHR is:

\begin{lstlisting}
buffer(buffer,C),
  chunk(C,foo),
  chunk_has_slot(C,spam,Bar),
==>
  Bar=Eggs |
  ...
\end{lstlisting}

\begin{lstlisting}
=buffer>
  isa    foo
  spam  =bar
  spam  =eggs
  ham   =eggs
\end{lstlisting}

will match all chunks which have the same value in \verb|spam| and \verb|ham|. \FIXME{CHR translation???}

\end{example}

\paragraph{Relation to Negation-as-Absence}

Negation-as-absence as described in \cite[147\psqq]{fru_chr_book_2009} is a concept provided by many production rule systems. It enables the programmer to negate a fact in the sense that the fact is not in the store and therefore the rule is applicable -- so the programmer asks for the absence of a particular fact.

At the first glance, the slot modifiers seem to implement this concept, but this is not the case: All negated slot tests in ACT-R can be reduced to simple built-in guard checks, because the chunks in the store are always described completely and the values are ground, so simple built-in checks work automatically. This also works for invalid slot-tests that ask for slots which are not offered by the chunk-type they ask for. Then there will never be a constraint matching the head of the rule due to type consistency.

With these restrictions, ACT-R avoids the problems that come with negation-as-absence as they are explained in \cite[147\psqq]{fru_chr_book_2009} and the translation of such tests to CHR is very simple.

\subsubsection{Empty Slots}

An important special case in the semantics of ACT-R production rules is, that if there is a slot test specified, then a potential chunk only matches, if it really has a value in this slot. Chunks that have \verb|nil| in a slot specified in a buffer test, will not match the test. Hence, variables can not be used to test if two slots have the same value and the value is \verb|nil|, since every positive slot test involving \verb|nil| fails automatically \cite[p. 164, section ``Variables'', last sentence]{actr_reference}.

In CHR this special case can be handled, by adding a guard for each variable occurring in a positive slot-test checking that this variable does not equal \verb|nil|.

For negated slot tests, this is not the case: 

\begin{lstlisting}
=buffer>
  isa    foo
  -spam  4
\end{lstlisting}

matches also a chunk with an empty \verb|spam| slot (\verb|nil| in its \verb|spam| slot).



\subsubsection{Outputs}

The production system of ACT-R also provides methods to produce side-effects. In this work, only a subset of those methods is concerned: the outputs. Outputs can appear on the right hand side of an ACT-R rule:

\begin{lstlisting}
=buffer>
  isa  foo
==>
  !output! (a1 a2 a3)
\end{lstlisting}

The argument of such an output call is a list of Lisp-symbols, so it is possible to hand variables or terms.

This mechanism can be translated to Prolog directly:

\begin{lstlisting}
output([]) :-
  nl.
output([X|Xs]) :-
  write(X),
  output(Xs).
\end{lstlisting}

The \verb|X| have to be Prolog terms.

In ACT-R, function calls like \verb|!eval!| or \verb|!bind!| are allowed, but they are ignored in this work.

\section{Modular Organization}

The term \emph{module} is highly overloaded: In ACT-R it describes independent parts of human cognition, whereas in the world of programming the term is used in a slightly different manner. In the following, implementational modules will always be named explicitly as \emph{Prolog modules}.

Nevertheless, the modular organization of ACT-R with its independent modules can be implemented by defining a Prolog module for each ACT-R module and adding some other modules around them. In the following, the concept of Prolog modules is explained.

\subsection{Prolog Modules}

Defining a new module creates a new namespace for all CHR constraints and Prolog predicates, which is illustrated in the following example:

\begin{example}[Prolog Modules and CHR]

In this example, two modules \verb|mod1| and \verb|mod2| are defined, with partially overlapping constraints. \verb|mod2| exports the constraint \verb|c|. In the following, the behaviour an interaction of the modules is explored.

\begin{lstlisting}[caption={Definition of Module 1},label=lst:mod1]
:- module(mod1,[]).
:- use_module(library(chr)).

:- use_module(mod2).

:- chr_constraint a/0, b/0.

a <=> c.
\end{lstlisting}

\begin{lstlisting}[caption={Definition of Module 2},label=lst:mod2]
:- module(mod2,[c/0]).
:- use_module(library(chr)).

:- use_module(mod1).

:- chr_constraint a/0, b/0, c/0. 

a <=> b.
b <=> mod1:a.
\end{lstlisting}


In this definition, two new modules \verb|mod1| and \verb|mod2| are created and only the \verb|c| constraint of \verb|mod2| is exported, indicated by the lists in the module definitions.

The CHR constraint \verb|a| in listing~\ref{lst:mod1} is internally represented as \verb|mod1:a|, so it lives in its own namespace and does not pollute other namespaces. The constraint can appear on the right hand side of rules of other modules, but has to be called explicitly with its full namespace identifier. In line~8 of listing~\ref{lst:mod2}, the presence of the local \verb|a| constraint leads the rule to fire and \verb|mod2:a| is replaced by \verb|mod2:b|, which leads the rule in line~9 to fire and replaces the local \verb|mod2:b| constraint by an external \verb|mod1:a| constraint. So, external constraints can be called by their complete identifiers.

However, on the left hand side of a rule, only the constraints local to the current module can appear. 

Exported constraints can only appear once in a program, since they can be called without their namespace definition, which is demonstrated in line~8 of listing~\ref{lst:mod1}, where \verb|mod2:c| is called in \verb|mod1| without referring to \verb|mod2| explicitly.
\end{example}

\subsection{Interface for Module Requests}
\label{interface_for_module_requests}

The architecture of ACT-R provides an infrastructure for the procedural module to state requests to all the other modules. To implement this concept as general as possible, an interface has to be defined, which allows the adding of new modules to the system by just implementing this interface.

\begin{lstlisting}[caption={Simple Interface ``Module''},label=lst:interface_module]
module_request(+BufName,+Chunk,-ResChunk,-ResState)
\end{lstlisting}

The arguments of such a request are:

\begin{description}
 \item[BufName] The name of the requested buffer, eg. \verb|retrieval|.
 \item[Chunk] A chunk specification that represents the arguments of the request. The form of the allowed chunk specifications and the semantics of the request are module-dependent. For example: \verb|chunk(_,t,[(foo,bar),(spam,eggs)])| could describe a chunk, that should be retrieved from declarative memory.
\end{description}

The request provides the following result:

\begin{description}
 \item[ResChunk] The resulting chunk in form of a chunk specification. The actual result and its semantics depend on the particular module.
 \item[ResState] The state of the buffer after the request. For example, if no matching chunk could be retrieved from declarative memory, the state would be \verb|error|.
\end{description}

This interface will be extended later on.

\subsection{How the Buffer System States a Request}
\label{how_the_buffer_system_states_a_request}

Every module that can handle a request implements the interface in listing~\ref{lst:interface_module}. When the buffer system gets a call \verb|buffer_request(buffer,chunk-specification)|, it simply can call the \verb|module_request| method of the corresponding module. This can be achieved by \verb|ModName:module_request(...)|, where \verb|ModName| is the name of the corresponding module.

Hence, the buffer system must now, which buffer belongs to which module. The \verb|buffer/2| constraint therefore has to be extended to a \verb|buffer/3| constraint, that also holds the module name of its module:

\begin{lstlisting}
buffer(BufName,ModName,Chunk)
\end{lstlisting}

A buffer request can now be handled as follows:

\begin{lstlisting}
buffer(BufName, ModName, _), buffer_request(BufName, Chunk) <=>
  set_buffer_state(BufName,busy),
  buffer_clear(BufName), % clear buffer immediately
  ModName:module_request(BufName, Chunk, ResChunk,ResState),
(ResState=error, 
  buffer(BufName, ModName, nil),
  set_buffer_state(BufName,error) ;
  
  ResState = free,
  ResChunk = chunk(ResChunkName,_,_),
  add_chunk(ResChunk), 
  buffer(BufName, ModName, ResChunkName),
  set_buffer_state(BufName,free)).
\end{lstlisting}

When getting a buffer request, the buffer state is set to \verb|busy| first. Then the buffer is cleared immediately, which leads the chunk in the buffer being added to the declarative memory. Then the module request is stated according to the interface. If the resulting state is \verb|error|, then the buffer gets this state and the content of the buffer is \verb|nil|, otherwise, if the resulting state is \verb|free|, the result will be added to the buffer and the state will eventually be set to \verb|free|. 

\subsection{Components of the Implementation}

uml component diagram + discussion



\section{Declarative Module}

The Declarative Module is a \emph{chunk store}, that additionally implements the \emph{module} interface. Therefore some rules to handle requests that find certain chunks in the chunk store have to be implemented. 

\subsection{Global Method for Adding Chunks}
\label{global_method_for_adding_chunks}

Since the declarative module is very important to the whole theory it kind of is a special module. Therefore, its Prolog module exports a predicate \verb|add_dm| that is just a wrapper for its chunk store. Hence, with the command \verb|add_dm| chunks can be added to the chunk store of the declarative module from every part of the program. This is eg. used for the clearing of buffers in the buffer system, where the chunk of a buffer goes to the declarative memory when the buffer is cleared. This is the implementation of the command:

\begin{lstlisting}
add_dm(ChunkDef) <=> add_chunk(ChunkDef).
\end{lstlisting}



\subsection{Retrieval Requests}
\label{retrieval_requests}

A retrieval request gets an incomplete chunk specification as input and returns a chunk, whose slots match the provided chunk pattern.

\subsubsection{Chunk Patterns}

The chunk patterns are transmitted in form of chunk specifications as defined in section~\ref{chunk_specification}. Since those specifications may be incomplete, variables are considered as place-holders for values in the result. The result always is a complete and ground chunk specification, because every chunk in a chunk store has to be defined completely; empty slots are indicated by the value nil.

\begin{example}
In this example some possible requests are discussed.

\begin{enumerate}
 \item Request:
\begin{lstlisting}
chunk(foo,bar,_)        
\end{lstlisting}

A chunk with name \verb|foo|, type \verb|bar| and arbitrary slot values is requested.

 \item Request:
\begin{lstlisting}
chunk(_,bar,_)        
\end{lstlisting}

This request is satisfied by every chunk of type \verb|bar|.

 \item Request:
\begin{lstlisting}
chunk(_,t,[(foo,bar),(spam,eggs)])        
\end{lstlisting}

The most common case of requests is a specification of the type and a (possibly incomplete) number of slot-value pairs for that type. If a type does not provide a specified slot, the request is invalid and no chunk will be returned.

\end{enumerate}

\end{example}

\subsubsection{Finding Chunks}

In this section, a CHR constraint \verb|find_chunk/3| will be defined, that produces a \verb|match_set/1| constraint for each chunk that matches a specified pattern. Eventually, the match set will be collected and returned.

\begin{lstlisting}
find_chunk(N1,T1,Ss), chunk(N2,T2) ==> 
  unifiable((N1,T1),(N2,T2),_), 
  nonvar(Ss) | 
  test_slots(N2,Ss), 
  match_set([N2]).
  
find_chunk(N1,T1,Ss), chunk(N2,T2) ==> 
  unifiable((N1,T1),(N2,T2),_), 
  var(Ss) | 
  test_slots(N2,[]), 
  match_set([N2]).

find_chunk(_,_,_) <=> true.
\end{lstlisting}

First, for each chunk in the store, whose name and type is unifiable with the specified name and type, will be part of the initial match set. If in the chunk specification the name and type are variables, each chunk will match. For the unification test, the \verb|unifiable/3| predicate of Prolog is used, because the unification should not be performed but only tested. 

If name and type match the pattern, then the slots have to be tested.

The rule in line~7 is for chunk specifications that do not specify the slots. In this case, no slots have to be tested. If all chunks have been tested or no chunk matches at all, the process is finished (rule in line ~13).

After adding each matching candidate to the match set, whose name and type have already been checked, the match set is pruned from chunks that have non-matching slot values:

\begin{lstlisting}
test_slots(_,[]) <=> true.

chunk_has_slot(N,S,V1), match_set([N]) 
\ test_slots(N,[(S,V2)|Ss]) <=> 
  unifiable(V1,V2,_) | 
  test_slots(N,Ss).

chunk_has_slot(N,S,V1) 
\ test_slots(N,[(S,V2)|_]), match_set([N]) <=> 
    \+unifiable(V1,V2,_) | 
    true.

test_slots(N,_) \ match_set([N]) <=> true.
\end{lstlisting}

The first rule is the base case, where no slots have to be tested any more and the test is finished and has been successful.

In line~3, the rule applies, if there is at least one slot \verb|(S,V2)| that has to be tested and a $HasSlot$ relation of the kind \verb|N| $\xrightarrow[]{\mathtt{S}}$ \verb|V1| with the slot \verb|S| to be tested, that is still in the match set, so no conflicting slot has been found yet. If the values \verb|V1| and \verb|V2| are unifiable, ie. the both are the same constant or at least one is a variable, then the test passes and the chunk \verb|N| remains in the match set and the rest of the slot tests are performed.

The second rule in line~8 applied if the guard of the first rule did not hold, so the values \verb|V1| and \verb|V2| are not unifiable, but there is a connection \verb|N| $\xrightarrow[]{\mathtt{S}}$ \verb|V1| and in the request it has been specified that the value of slot \verb|S| has to be \verb|V2|. In this case, \verb|V1| $\neq$ \verb|V2|, so the test fails and the chunk \verb|N| has to be removed from the match set, since one of its slots does not match.

If the last two rules cannot be applied, the chunk does not provide a slot that, however, has been specified in the request. Hence, the chunk does not match and the last rule therefore deletes it from the match set. \FIXME{make simpagation rule to simplification rule?}

If those rules have been applied exhaustively, only the matching chunks will remain in the match set: If there would be an outstanding slot test, one of the rules would be applicable and the chunk would be removed from the match set, if the test fails (and the test would also be removed, because it has been performed). If the test is successful, the chunk will remain part of the match set, but the test will be removed. So the match set is correct and complete.

However, since the match set is distributed over a set of \verb|match_set| constraints, it would be desirable to collect all those matches in one set. This can be triggered by a \verb|collect_matches/1| constraint, that gets the complete match set in its arguments:

\begin{lstlisting}
collect_matches(_) \ match_set(L1), match_set(L2) <=> 
  append(L1,L2,L), 
  match_set(L).
  
collect_matches(Res), match_set(L) <=> Res=L.

collect_matches(Res) <=> Res=[].
\end{lstlisting}

The first rule merges two match sets to one singe merge set containing a list with all the chunks of the former sets, if the \verb|collect_matches| trigger is present. 

In the second rule, if no two match sets are in the store, the result of the \verb|collect_matches| operation is the match set. The same applies for the last rule, where no match set is in the store and therefore the result is empty.

Note, that this implies, that the rules have to be applied from top to bottom, left to right\footnote{This is called the refined operational semantics of CHR}.

The symbolic layer does not implement any rule for which chunk will be returned, if there are more than one in the match set. In this implementation, the first chunk in the list is chosen. The module request is now implemented as follows:

\begin{lstlisting}
module_request(retrieval,chunk(Name,Type,Slots),ResChunk,ResState) <=> 
  find_chunk(Name,Type,Slots),
  collect_matches(Res),
  first(Res,Chunk),
  return_chunk(Chunk,ResChunk),
  get_state(ResChunk,ResState).
\end{lstlisting}

where \verb|first(L,E)| gets a list \verb|L| and returns its first element \verb|E| or \verb|nil|, if the list was empty.

With \verb|return_chunk/2| and \verb|get_state/2|, the actual results of the request are computed:

By now, the variable \verb|Chunk| holds the name of the chunk to return, but in the specification of the module request, a complete chunk specification is demanded. \verb|return_chunk/2| is defined as a default method of a chunk store that gets a chunk name as its first argument and returns a chunk specification created from the values in the chunk store as its second argument.

The resulting state of the request is computed as follows: If the result chunk is \verb|nil|, then no chunk was in the match set, so the state of the declarative module will be \verb|error|. In any other case, the state is \verb|free| after the request has been performed.

\subsection{Chunk Merging}

An important technique used in ACT-R's declarative memory module, is the chunk merging: If a chunk enters the chunk store and all of its slots and values are the same as of a chunk already in the store, then those two chunks get merged. The merged chunk can be called with both names.

If a chunk entering the declarative memory has the same name as a chunk in that already is in the store, the new chunk gets a new name and if its slots are the same as the slots of the old chunk, they will be merged and the new name will be deleted.

\begin{lstlisting}
chunk(Name,Type) \ add_chunk(chunk(Name,Type,Slots)) <=>
  Type \== chunk |
  add_chunk(chunk(Name:new,Type,Slots)).

% first check, if identical chunk exists
add_chunk(chunk(C,T,S)) ==> 
  T \== chunk | 
  check_identical_chunks(chunk(C,T,S)).
\end{lstlisting}

The rules are added before the empty slot initialization in listing~\ref{lst:add_chunk_rules}. The first rule handles the case where a chunk with a already allocated name is added to the store. Then it gets a new name (which basically is just the old name extended by \verb|:new|, and tries to add this new chunk to the memory. 

The second rule adds a identity check for each chunk to be added except for primitive elements, since their slots are identical for every name, because every primitive element has no slots. Nevertheless they have to be distinguishable by their names and therefore cannot be merged.

The identity check is implemented as follows:

\begin{lstlisting}
check_identical_chunks(nil) <=> true.

chunk(NameOld,Type), check_identical_chunks(chunk(NameNew,Type,Slots)) ==> 
  check_identical_chunks(chunk(NameNew,Type,Slots),NameOld).
  
check_identical_chunks(_) <=> true.
\end{lstlisting}

The rule in line~3 adds for each identity check and each chunk in the store a pairwise identity check which is performed by the following rules:

\begin{lstlisting}
chunk(NameOld, Type) \ check_identical_chunks(chunk(NameNew,Type,[]),NameOld) <=> 
  identical(NameOld,NameNew).
  
chunk(NameOld, Type), chunk_has_slot(NameOld, S, V) \ check_identical_chunks(chunk(NameNew,Type,[(S,V)|Rest]),NameOld) <=> 
  check_identical_chunks(chunk(NameNew,Type,Rest),NameOld).
  
chunk(NameOld, Type), chunk_has_slot(NameOld, S, VOld) \ check_identical_chunks(chunk(_,Type,[(S,VNew)|_]),NameOld) <=> 
  VOld \== VNew |
  true.
    
remove_duplicates @ identical(N,N) <=> true.

% abort checking for identical chunks if one has been found
cleanup_identical_chunk_check @ identical(NameOld,NameNew) \ check_identical_chunks(chunk(NameNew,_,_),NameOld) <=> true.
\end{lstlisting}

The first rule is the base case, where no slots have to be tested any more. Then the chunks are identical and an \verb|identical/2| constraint is added to the store for those two chunks.

The next rule applies for an identity check for a slot-value pair that is successful, whereas the third rule handles the opposite case and aborts the check for this pair of chunks without adding an \verb|identical| constraint.

In the fourth rule redundant information is removed and the lust rule aborts the identity check as soon as one matching chunk has been found.

Note that this implementation assumes that the chunk specification of the chunk entering the declarative memory is complete. If it is not complete, the correct semantics of the adding is that all unspecified slots are empty so their values equal \verb|nil|. Nevertheless, this implementation would merge the chunk with a chunk that has values in those unspecified slots. This could be improved by completing the chunk specifications before checking for identical chunks in the store.

For the easier use of the identical constraint at other points, transitive identities can be reduced to the only real chunk in the store (the one with the \verb|chunk| constraint):

\begin{lstlisting}
reduce @ chunk(C1,_), identical(C1,C2) \ identical(C2,C3) <=> identical(C1,C3).
\end{lstlisting}

Additionally, the newly created names can be deleted if the chunk was identical to another chunk with a real name, since the new nomenclature was only store internal, so only the original name must be kept:

\begin{lstlisting}
identical(C,C:new) <=> true.
\end{lstlisting}

In the declarative module, the chunk search must be extended to find merged chunks by both names. This can be achieved as follows:

\begin{lstlisting}
identical(C1,C2) \ find_chunk(C2,T2.Ss2)  <=> ground(C2) | find_chunk(C1,T2,Ss2).
\end{lstlisting}

so the chunk search of a chunk that is only a pointer to another chunk can be reduced to the search of this chunk. \FIXME{maybe this can cause problems if a production rule refers to a particular name and gets a chunk with the name of its brother..}

The process of chunk merging is described in \cite[??]{actr_reference}. The treating of identical names is not described there, but has been tested in the original implementation: ACT-R handles those identical names similarly to this approach by adding a sequential number to the identical name. For example, the chunk with name \verb|a| would be called \verb|a-0| if there already was a chunk \verb|a| in the store.

\section{Initialization}
\label{initialization}

In the examples the models had to be run by stating complex queries which create all necessary buffers and add all chunk types and chunks to the declarative memory manually. In the original ACT-R implementation, the command \verb|run| is used to run a model. This behaviour can be transferred to the CHR implementation easily by adding a \verb|run| constraint and a rule for this constraint, that performs all the initialization work.

\FIXME{modify count example from above}

\section{Timing in ACT-R}

So far, the execution order of the production rules has been controlled by the \verb|fire| constraint -- a phase constraint that simulated the occupation of the production system while a rule is executed.

However, certain buffer actions like buffer requests may take some time until they are finished. The procedural system is free to fire the next rule after all actions have been started\footnote{see chapters \ref{serial_parallel_aspects} and \ref{process_of_rule_selection_and_execution}} and the requests are performed in parallel to that.

Additionally, for the simulation it may be interesting to explore how much time certain actions have taken, especially when it comes to the subsymbolic layer.

Those aspects cannot be implemented easily using the current approach with phase constraints. Hence, the idea of introducing a central scheduling unit is a possible solution of those requirements: The unit has a serialized ordered list of events with particular timings. If a new event is scheduled, the time it is executed must be known. The scheduling unit inserts the event at the right position of the list preserving the ordered time condition. Figure~\ref{fig:scheduler} illustrates this approach.

The system just removes such events from the queue and executes them, which leads to new events in the queue. The queue organizes the right order of the events.

With this approach, the simulation of a parallel execution of ACT-R can be achieved: Each buffer action on the RHS of a rule just schedules an event that actually performs the action at a specified time (the current time plus its duration). 

The heart of the scheduler is a \emph{priority queue} which is described in the next section.

\subsection{Priority Queue}

A \emph{priority queue} is an abstract data structure that serves objects by their priority. It provides the following abstract methods:

\begin{description}
 \item[enqueue with priority] An object with a particular priority is inserted to the queue. 
 \item[dequeue highest priority] The object with the highest priority is removed from the queue and returned.
\end{description}

\subsubsection{Objects}

In the implementation of the scheduler, the priority queue contains objects of the form \verb|q(Time,Priority,Event)|. The priority of such a queue object is composed from the \verb|Time| and the \verb|Priority|. The order between the elements is defined as follows:

\begin{definition}[ordered time-priority condition]
\verb|q(T1,P1,E1)| $\prec$ \verb|q(T2,P2,E2)|, if \verb|T1| $<$ \verb|T2|. In the case that \verb|T1 = T2|, then \verb|q(T1,P1,E1)| $\prec$ \verb|q(T2,P2,E2)| if \verb|P1| $>$ \verb|P2|. So, events with smaller times will be returned first. If two events appear at the same time, it is possible to define a priority and the event with the higher priority will be returned first. 
\end{definition}

\subsubsection{Representation of the Queue}

The representation of the priority queue is inspired by \cite[38\psqq]{fru_chr_book_2009}: An order constraint \verb|A --> B| is introduced, which states that \verb|A| will be returned before \verb|B| (or \verb|B| is the direct successor of \verb|A|). The beginning of the queue will be defined by a start symbol \verb|s|, so the first real element is the successor of \verb|s|. A possible queue could be:

\begin{lstlisting}
s         --> q(1,0,e1)
q(1,0,e1) --> q(3,7,e2)
q(3,7,e2) --> q(3,2,e3)
\end{lstlisting}

This queue achieves the ordered time-priority condition, since the queue objects are in the correct order according to their times and priorities. It is also consistent in a sense that it has no gaps and no object has more than one successor.

In general, there can be defined some rules to make such a queue consistent, ie. every object only has one successor and the queue achieves the time-priority condition:

\begin{lstlisting}
A --> A <=> true.

_ --> s <=> false.

A --> B, A --> C <=>
  leq(B,C) |
  A --> B,
  B --> C.
\end{lstlisting}

The first rule states, that if an object is its own successor, this information can be deleted. The second rule states, that nothing can be the predecessor of the start symbol. The last rule is the most important one: If one object has two successors, then these connections have to be divided into two connections according to the defined time-priority condition. This condition can be implemented as follows:

\begin{lstlisting}
leq(s,_).

% Time1 < Time2 -> event with time1 first, priority does not matter
leq(q(Time1,_,_), q(Time2,_,_)) :- 
  Time1 < Time2.

% same time: event with higher priority first
leq(q(Time,Priority1,_), q(Time,Priority2,_)) :- 
  Priority1 >= Priority2.
\end{lstlisting}

The first predicate states that the start symbol is is less than every other object. The other two rules implement the time-priority condition directly.

Note that two objects are considered the same, iff. their time, priority and event are syntactically the same. If an object is altered in one of the \verb|-->| constraints, it has to be edited in every other occurrence in the   list to avoid gaps.

Another important property is, that the list does not have any gaps, so it must be possible to track the queue from every element backwards to the start symbol. This condition is not achieved by the rules above, but since a priority queue only offers two mechanisms to modify it, a lot of those problems can be avoided:

\begin{description}
 \item[add\_q(Time,Priority,Event)] Enqueues an object with the specified properties. Ie., a new \verb|q(Time,Priority,Event)| object will be created and the following constraint will be added: \verb|s --> q(Time,Priority,Event)|. The rules presented above will lead to a linear, serialized list achieving the time-priority condition without gaps.
\begin{lstlisting}
add_q(Time,Priority,Evt) <=>
  s --> q(Time,Priority,Evt).
\end{lstlisting}
 \item[de\_q(X)] Dequeues the first element of the queue according to the time-priority condition and binds its value to \verb|X|.
\begin{lstlisting}
de_q(X), s --> A, A --> B <=>
  X = A,
  s --> B.

de_q(X), s --> A <=>
  X = A.  
  
de_q(X) <=> X = nil.
\end{lstlisting}

The first object just is the successor of \verb|s|, since the list has been constructed preserving the correct order and the property, that everything starts at \verb|s|. If the first object has a successor, this object is the new first object. If there are no order constraints left, the queue is empty and \verb|de_q| returns \verb|nil|.

\end{description}

\subsubsection{Special Operation for ACT-R}

In this implementation, another default method is added to the priority queue: 

\begin{description}
 \item[after\_next\_event(E)] Adds the event \verb|E| to the queue, after the first event without destroying the consistency and the time-priority condition of the queue.
\end{description}

To implement this method, the time and priorities have to be set such that the time-priority condition does hold:

\begin{lstlisting}
s --> q(Time,P1,E1) \ after_next_event(Evt) <=> 
  NP1 is P1 + 2, % increase priority of first event, so it still has highest priority
  P is P1 + 1, % priority of event that is added, ensured that it is higher than of the former second event (because it is P1+1)
  de_q(_), % remove head of queue
  add_q(Time,NP1,E1), % add head of queue again with new priority. Will be first again, because it has old prio (which is higher than prio of all successors)
  add_q(Time,P,Evt). % add new event. Will be < than Prio of head but it is ensured that it is higher than prio of second event
\end{lstlisting}

If the first event is \verb|q(Time,P1,E1)| and a new event \verb|Evt| has to be added after this event, the times of the two events are the same, the priority of the first event is \verb|P1 + 2| and the priority of the new event is \verb|P1 + 1|. The first event is removed from the queue and would be added with its new priority to the queue again, as it is with the new event. 

This is correct: The first event will be the first event again, because its old priority was higher than any other priority at that time point and since the new priority is even higher than that, no other event from the queue will have a higher priority. The new event also has a higher priority than every other event in the queue, but a lower priority than the first event, so it will be added after the first event.

By the \verb|de_q| and \verb|add_q| actions it is ensured, that no garbage of the old event remains in the queue and the events are added correctly through a official method of the queue.

\subsection{Scheduler}

The scheduler component is a own module that manages events by feeding a priority queue and controls the recognize-act cycle. It also holds the current time of the system. 

\subsubsection{Current Time}

The current time can be saved in a \verb|now/1| constraint. It is important that there is only one such constraint and that time increases monotonically.

Other modules can access the time only by a \verb|get_now/1| that only returns the current time saved in the \verb|now/1| constraint. The current time cannot be set from outside, but is determined by the last event dequeued from the priority queue.

\subsubsection{Recognize-Act Cycle}

As described before, the procedural module can only fire one rule at a time. When executing the RHS of a rule, all its actions are added to the scheduler with the time point when their execution is finished. For example: If on the RHS of the firing rule a retrieval request has to be performed, an event will be added to the priority queue with the time $Now + Duration$, so the chunk retrieved from the declarative memory will be written to the retrieval buffer at this time point.

After all events of the RHS have been added to the scheduler, the procedural module is free again and therefore the next rule can fire. The event of firing will be added to the priority queue as well by the \verb|fire| constraint at the end of each production rule. The rule will be added at the current time point, but with low priority, since it has to be ensured, that every action of the previously executed rule has been performed yet (in the sense that a corresponding event has been added at the specified time point). In many cases, no other rule will be applicable at this time, because most of the meaningful production rules have to wait for the results of the production rule before. 

Many of the buffer actions are performed immediately, so an event with the current time point is added for the buffer modifications or clearings. They are performed in a certain order defined by their priority as shown in table~\ref{tab:action_priorities}. The request actions are performed at last, but they perform a buffer clearing immediately and then start their calculation which can take some time. 

If no rule is applicable, the next time that a rule could be applicable is after having performed the next event. So, the next \verb|fire| event is scheduled directly after the first event in the queue. This simulates the behaviour that the procedural module stays ready to fire the next rule, without polling at every time point if a rule is applicable, but only reacting on changes to the buffer system. 


The following enumeration summarizes the recognize-act cycle with a scheduler:

\begin{enumerate}
 \item The next event is removed from the queue, the current time is set to the time of the event and the event is performed.
 \begin{enumerate}
 \item The dequeued event is a \verb|fire| constraint: The rule that matches all its conditions is fired and removes the \verb|fire| constraint.
 \item The actions of the rule are scheduled in the queue. Modifications and Clearings have the current time point, requests have a time point in the future depending on the module.
 \item The last action of the rule is to add a \verb|fire| constraint to the queue with the current time point and a low priority. This simulates that the procedural module is free again, after all in-place actions of a rule have been fired.
 \item There are two possibilities:
 \begin{enumerate}
  \item \emph{The next rule matches:} It will be performed like the last rule.
  \item \emph{No rule matches:} The next time, it could be possible that a rule can fire, is when something in the buffers changed. This only can happen, after the next event has been performed. So the next \verb|fire| event will be added to the queue by \verb|after_next_event| which has been described above.
 \end{enumerate}
  \end{enumerate}
 \item Go to point 1. This is performed until there are no events in the queue.
\end{enumerate}


The following parts are necessary to implement this cycle:

\paragraph{Start Next Cycle} 

The constraint \verb|nextcyc| leads the system to remove the next event from queue and perform it. Performing is done by a \verb|call_event| constraint:

\begin{lstlisting}
% After an event has been performed, nextcyc is triggered. 
%This leads to the next event in the queue to be performed.
nextcyc <=> de_q(Evt), call_event(Evt).
\end{lstlisting}

\paragraph{Call an Event} 

Event calling just takes a queue element and sets the current time to the time of the event and performs a Prolog \verb|call|. Additionally, a message is printed to the screen. After the event has been executed, the next cycle is initiated.

If the queue element is \verb|nil| (so no event has been in the queue), the computation is finished and the current time is removed.

\begin{lstlisting}
% no event in queue -> do nothing and remove current time
call_event(nil) \ now(_) <=> write('No more events in queue. End of computation.'),nl.

call_event(q(Time,Priority,Evt)), now(Now) <=> 
  Now =< Time | 
  now(Time),
  write(Now:Priority),
  write(' ... '),write('calling event: '), write(Evt),nl,
  call(Evt),
  nextcyc.
\end{lstlisting}

\paragraph{Changing the Buffer System}

For each buffer action, add a \verb|do_buffer_action| constraint, that actually performs the code specified in the former action. Modify the action as follows:

\begin{lstlisting}
% Schedule buffer_action
buffer_action(BufName, Chunk) <=> 
  get_now(Now),
  Time is Now + Duration, 
  add_q(Time, Priority, do_buffer_action(BufName, Chunk)). 
\end{lstlisting}

with appropriate values for \verb|Duration| and \verb|Priority|.

\paragraph{Production Rules}

As in the last version of the production system, each rule has the following structure:

\begin{lstlisting}
rule @
  {conditions} \ fire <=> {actions}, schedule_fire.
\end{lstlisting}

where schedule fire is defined as:

\begin{lstlisting}
schedule_fire <=> 
  get_now(Now),
  add_q(Now,0,fire).
\end{lstlisting}

As the last production rule, there has to be:

\begin{lstlisting}
no-rule @ 
  fire <=> no_rule.
\end{lstlisting}

which removes the fire constraint if still present and states that no rule has been fired (since the fire constraint is still present). In this case, a new \verb|fire| event is scheduled after the next event:

\begin{lstlisting}
no_rule <=> 
  write('No rule matches -> Schedule next conflict resolution event'),nl,
  after_next_event(do_conflict_resolution).
\end{lstlisting}

\section{Configuration}


\section{Subsymbolic Layer}

\subsection{Activation of Chunks}

\subsection{Production Utility}


\chapter{Example Models}
\label{example_models}

In this chapter, two ACT-R models are translated into CHR and the results are discussed. The examples are taken from \citetitle{actr_tutorial} \cite{actr_tutorial}.

\section{The Counting Model}

The first model is an extension of the example presented in section~\ref{example_counting}. The code has been published in unit 1 of \citetitle{actr_tutorial} \cite{actr_tutorial} as shipped with the source code of the vanilla ACT-R 6.0 implementation. Basically, the counting process is modeled by a set of static facts that a person has learned and retrieves for the counting process from declarative memory. The following model definition is discussed:

\raggedbottom
\begin{lstlisting}[escapeinside={(*@}{@*)}, caption={ACT-R code of the counting example}, label=exmod:lst:example_counting]
(define-model count (*@\label{exmod:lst:example_counting:define_model}@*)

  (chunk-type count-order first second) (*@\label{exmod:lst:example_counting:chunk_type}@*)
  (chunk-type count-from start end count)

  (add-dm (*@\label{exmod:lst:example_counting:add_dm}@*)
   (b ISA count-order first 1 second 2)
   (c ISA count-order first 2 second 3)
   (d ISA count-order first 3 second 4)
   (e ISA count-order first 4 second 5)
   (f ISA count-order first 5 second 6)
   (first-goal ISA count-from start 2 end 4)
   ) (*@\label{exmod:lst:example_counting:add_dm_end}@*)

  (P start (*@\label{exmod:lst:example_counting:start}@*)
     =goal>
        ISA         count-from
        start       =num1
        count       nil
   ==>
     =goal>
        count       =num1
     +retrieval>
        ISA         count-order
        first       =num1
  )

  (P increment (*@\label{exmod:lst:example_counting:increment}@*)
     =goal>
        ISA         count-from
        count       =num1
      - end         =num1
     =retrieval>
        ISA         count-order
        first       =num1
        second      =num2
   ==>
     =goal>
        count       =num2
     +retrieval>
        ISA         count-order
        first       =num2
     !output!       (=num1)
  )

  (P stop (*@\label{exmod:lst:example_counting:stop}@*)
     =goal>
        ISA         count-from
        count       =num
        end         =num
   ==>
     -goal>
     !output!       (=num)
  )
  
  (goal-focus first-goal) (*@\label{exmod:lst:example_counting:goal_focus} \enlargethispage{\baselineskip}@*)
) 
\end{lstlisting}
\flushbottom

First of all, in line~\ref{exmod:lst:example_counting:define_model} the model definition is initiated and a model name is given. Line~\ref{exmod:lst:example_counting:chunk_type} sq. add the two necessary chunk-types: A chunk-type for the goal-chunks and a chunk-type for the actual declarative data. These chunk-type definitions are global to the system, i.e. they are added to all chunk stores. The goal-chunks have the slots \lstinline|start| and \lstinline|end| which encode from which number the counting process should start and where it should end. The value in the slot \lstinline|count| saves the current number of the count-process, analogously as in section~\ref{example_counting}. Then the chunks are added from line~\ref{exmod:lst:example_counting:add_dm} to~\ref{exmod:lst:example_counting:add_dm_end} and are representations of the order of the natural numbers from one to six. The last chunk is the goal-chunk. Note that only the start and end numbers are specified, the current number will be set to \lstinline|nil|.

The first production rule \lstinline|start| in line~\ref{exmod:lst:example_counting:start} sqq. can only be applied, if the goal has an empty \lstinline|count| slot, but an actual value in the \lstinline|start| slot. This will be valid for the initial state as explained later. The production states a request for the first count-fact with the start number in its \lstinline|first| slot and sets the current number in the goal to the start number. The other slots in the goal buffer remain the same.

In line~\ref{exmod:lst:example_counting:increment} sqq. the main rule \lstinline|increment| is defined. It assumes that a count-order fact which matches the current number in the goal has been retrieved. Additionally, the counting process must not end with this number, indicated by the negated slot test on the slot \lstinline|end|. The action part then states a new request for the next count-fact and increments the current number. Once the rule is applicable, it will be applicable as long there are count-facts in the declarative memory and the specified end has not been reached, yet.

The last production rule \lstinline|stop| in line~\ref{exmod:lst:example_counting:stop} is applicable, as soon the \lstinline|increment| rule cannot be applied anymore due to the fact that the specified end of the counting process has been reached. Then, the last number will be printed and the goal buffer will be cleared.

The last function call in line~\ref{exmod:lst:example_counting:goal_focus} is an initialization method, which simply puts the previously defined chunk \lstinline|first-goal| into the goal buffer. This leads to an initial state where the rule \lstinline|start| is applicable. The output of the model is:

%\raggedbottom
\begin{lstlisting}
     0.000   GOAL             SET-BUFFER-CHUNK GOAL FIRST-GOAL 
     0.000   PROCEDURAL       CONFLICT-RESOLUTION 
     0.050   PROCEDURAL       PRODUCTION-FIRED START 
     0.050   PROCEDURAL       CLEAR-BUFFER RETRIEVAL 
     0.050   DECLARATIVE      START-RETRIEVAL 
     0.050   DECLARATIVE      RETRIEVED-CHUNK C 
     0.050   DECLARATIVE      SET-BUFFER-CHUNK RETRIEVAL C 
     0.050   PROCEDURAL       CONFLICT-RESOLUTION 
     0.100   PROCEDURAL       PRODUCTION-FIRED INCREMENT 
2 
     0.100   PROCEDURAL       CLEAR-BUFFER RETRIEVAL 
     0.100   DECLARATIVE      START-RETRIEVAL 
     0.100   DECLARATIVE      RETRIEVED-CHUNK D 
     0.100   DECLARATIVE      SET-BUFFER-CHUNK RETRIEVAL D 
     0.100   PROCEDURAL       CONFLICT-RESOLUTION 
     0.150   PROCEDURAL       PRODUCTION-FIRED INCREMENT 
3 
     0.150   PROCEDURAL       CLEAR-BUFFER RETRIEVAL 
     0.150   DECLARATIVE      START-RETRIEVAL 
     0.150   DECLARATIVE      RETRIEVED-CHUNK E 
     0.150   DECLARATIVE      SET-BUFFER-CHUNK RETRIEVAL E 
     0.150   PROCEDURAL       CONFLICT-RESOLUTION 
     0.200   PROCEDURAL       PRODUCTION-FIRED STOP 
4 
     0.200   PROCEDURAL       CLEAR-BUFFER GOAL 
     0.200   PROCEDURAL       CONFLICT-RESOLUTION 
     0.200   ------           Stopped because no events left to process 
\end{lstlisting}
%\flushbottom

The source file in listing \ref{exmod:lst:example_counting} can be translated by the provided CHR compiler which yields the following result:

\raggedbottom
\begin{lstlisting}[escapeinside={(*@}{@*)}, caption={Auto-generated CHR code of the counting example}, label=exmod:lst:example_counting_chr]
:- include('actr_core.pl').
:- chr_constraint run/0, fire/0.

delay-start @ 
  fire,
  buffer(goal,_,A),
    chunk(A,count-from),
    chunk_has_slot(A,start,B),
    chunk_has_slot(A,count,nil)
==>
  B\==nil |
  conflict_set(start).

start @ 
  buffer(goal,_,A),
    chunk(A,count-from),
    chunk_has_slot(A,start,B),
    chunk_has_slot(A,count,nil)
  \ apply_rule(start)
<=>
  B\==nil |
  buffer_change(goal,chunk(_,_,[ (count,B)])),
  buffer_request(retrieval,chunk(_,count-order,[ (first,B)])),
  conflict_resolution.

delay-increment @ 
  fire,
  buffer(goal,_,A),
    chunk(A,count-from),
    chunk_has_slot(A,count,C),
    chunk_has_slot(A,end,D),
  buffer(retrieval,_,B),
    chunk(B,count-order),
    chunk_has_slot(B,first,C),
    chunk_has_slot(B,second,E)
==>
  C\==nil,
  D\==C,
  E\==nil |
  conflict_set(increment).

increment @ 
  buffer(goal,_,A),
    chunk(A,count-from),
    chunk_has_slot(A,count,C),
    chunk_has_slot(A,end,D),
  buffer(retrieval,_,B),
    chunk(B,count-order),
    chunk_has_slot(B,first,C),
    chunk_has_slot(B,second,E)
  \ apply_rule(increment)
<=>
  C\==nil,
  D\==C,
  E\==nil |
  buffer_change(goal,chunk(_,_,[ (count,E)])),
  buffer_request(retrieval,chunk(_,count-order,[ (first,E)])),
  output(C),
  conflict_resolution.

delay-stop @
  fire,
  buffer(goal,_,A),
    chunk(A,count-from),
    chunk_has_slot(A,count,B),
    chunk_has_slot(A,end,B)
==>
  B\==nil | 
  conflict_set(stop).

stop @ 
  buffer(goal,_,A),
    chunk(A,count-from),
    chunk_has_slot(A,count,B),
    chunk_has_slot(A,end,B)
  \ apply_rule(stop)
<=>
  B\==nil |
  buffer_clear(goal),
  output(B),
  conflict_resolution.
  
init @ (*@\label{exmod:lst:example_counting_chr:init}@*)
run <=> true | 
  set_default_utilities([stop,increment,start]),
  add_buffer(retrieval,declarative_module),
  add_buffer(goal,declarative_module),
  lisp_chunktype([chunk]),
  lisp_chunktype([count-order,first,second]),
  lisp_chunktype([count-from,start,end,count]),
  lisp_adddm([[b,isa,count-order,first,1,second,2],
    [c,isa,count-order,first,2,second,3],
    [d,isa,count-order,first,3,second,4],
    [e,isa,count-order,first,4,second,5],
    [f,isa,count-order,first,5,second,6],
    [first-goal,isa,count-from,start,2,end,4]]),
  lisp_goalfocus([first-goal]),
  now(0),
  conflict_resolution,
  nextcyc.
  
no-rule @ (*@\label{exmod:lst:example_counting_chr:no_rule}@*)
fire <=> 
  true |
  conflict_set([]),
  choose.
\end{lstlisting}
\flushbottom

First of all, it is interesting to note, that there are some more CHR rules than production rules in the original code in listing~\ref{exmod:lst:example_counting}. One very obvious extension is the rule \lstinline|init| in line~\ref{exmod:lst:example_counting_chr:init}. This rule initializes the model according to section~\ref{initialization}. First, it sets the default utilities for each of the production rules and creates the used buffers in the buffer system. Then, the Lisp functions from the original model definitions are called in their CHR versions, according to section~\ref{lisp_functions}. Those are the methods which create chunk-types and add the initial declarative knowledge to the declarative module. Note that also the artificial chunk-type \lstinline|chunk| with no slots is created. The last Lisp call moves the \lstinline|first-goal| chunk to the goal buffer. Finally, the current time is set to zero, a conflict resolution event is scheduled in the queue (according to section~\ref{implementation:conflict_resolution}) and the recognize-act cycle is started by \lstinline|nextcyc|, which will dequeue the first event from the scheduler as described in section~\ref{implementation:scheduler:recognize-act}.

Furthermore, each production rule from the original module has two correspondent CHR rules in the translation due to the conflict resolution process described in section~\ref{implementation:conflict_resolution}. The first rule delays the execution by adding the rule -- if applicable -- to the conflict set as soon as a new conflict resolution event (represented by the CHR constraint \lstinline|fire|) has been triggered. The second rule actually performs the rule, if it has been chosen by the conflict resolution process (indicated by the constraint \lstinline|apply_rule/1|). After the rule application, each rule schedules the next conflict resolution event by adding \lstinline|conflict_resolution| to the store. The guards of the production rules check that each tested slot actually has a value (so its value is not \lstinline|nil|) as described in section~\ref{empty_slots} (see page~\pageref{empty_slots}) and in the rule \lstinline|increment|, for example, a negated slot test is translated to a guard check according to the pattern presented in section~\ref{slot_modifiers} (see page~\pageref{slot_modifiers}).

The rule \lstinline|no-rule| in line~\ref{exmod:lst:example_counting_chr:no_rule} is the last rule tested in the collecting process of the conflict resolution. It removes the \lstinline|fire| constraint and triggers the choosing process, after it has added an empty rule to the conflict set. This is important for the choosing process to detect if no rule was applicable in this conflict resolution phase.

The output of the model is the following (pretty printed by hand):

\raggedbottom
\begin{lstlisting}
?- run.
0      ... calling event: do_conflict_resolution
           going to apply rule start
0.05   ... calling event: apply_rule(start)
           firing rule start
0.05   ... calling event: do_buffer_change(goal,chunk(_G17029,_G17030,[ (count,2)]))
0.05   ... calling event: start_request(retrieval,chunk(_G17476,count-order,[ (first,2)]))
           Started buffer request retrieval
           clear buffer retrieval
0.05   ... calling event: do_conflict_resolution
           No rule matches -> Schedule next conflict resolution event
1.05   ... calling event: do_buffer_request(retrieval,chunk(_G17476,count-order,[ (first,2)]))
           Retrieved chunk c
           Put chunk c into buffer
1.05   ... calling event: do_conflict_resolution
           going to apply rule increment
1.1:0  ... calling event: apply_rule(increment)
           firing rule increment
output:2
1.1    ... calling event: do_buffer_change(goal,chunk(_G33694,_G33695,[ (count,3)]))
1.1    ... calling event: start_request(retrieval,chunk(_G34139,count-order,[ (first,3)]))
           Started buffer request retrieval
           clear buffer retrieval
1.1    ... calling event: do_conflict_resolution
           No rule matches -> Schedule next conflict resolution event
2.1    ... calling event: do_buffer_request(retrieval,chunk(_G34139,count-order,[ (first,3)]))
           Retrieved chunk d
           Put chunk d into buffer
2.1    ... calling event: do_conflict_resolution
           going to apply rule increment
2.15   ... calling event: apply_rule(increment)
           firing rule increment
output:3
2.15   ... calling event: do_buffer_change(goal,chunk(_G25849,_G25850,[ (count,4)]))
2.15   ... calling event: start_request(retrieval,chunk(_G26294,count-order,[ (first,4)]))
           Started buffer request retrieval
           clear buffer retrieval
2.15   ... calling event: do_conflict_resolution
           going to apply rule stop
2.1999 ... calling event: apply_rule(stop)
           firing rule stop
output:4
2.1999 ... calling event: do_buffer_clear(goal)
           clear buffer goal
2.1999 ... calling event: do_conflict_resolution
           No rule matches -> Schedule next conflict resolution event
3.15   ... calling event: do_buffer_request(retrieval,chunk(_G26294,count-order,[ (first,4)]))
           performing request: retrieval
           Retrieved chunk e
           Put chunk e into buffer
3.15   ... calling event: do_conflict_resolution
           No rule matches -> Schedule next conflict resolution event
           No more events in queue. End of computation.
\end{lstlisting}
\flushbottom

Note that the output is very similar to the original output, especially the order of rule applications and events. However, the timings are not accurate, yet, since some constants are different from the original implementation, which can be fixed and does not really harm the theory. 

To demonstrate the subsymbolic layer, the original code is extended as follows:

\raggedbottom
\begin{lstlisting}[escapeinside={(*@}{@*)}]
(define-model count
  (sgp :esc t)

  (chunk-type count-order first second)
  (chunk-type goal-chunk goal start end count)

  (add-dm
  ...
  (d ISA count-order first 3 second 4)
  (d1 ISA count-order first 3 second 5)
  ...
  (second-goal ISA goal-chunk goal training1)
  )
  
  (P train1
    =goal>
      ISA           goal-chunk
      goal          training1
  ==>
    =goal>
      goal          training2
    +retrieval>
      ISA           count-order
      first         3
      second        4
  )

  (P train2
    =goal>
      ISA           goal-chunk
      goal          training2
    =retrieval>
      ISA           count-order
      first         3
      second        4
  ==>
    =goal>
      ISA           goal-chunk
      start         2
      end           4
      goal          count
    -retrieval> 
  )
  
  (P start
      { defined as in listing (*@\ref{exmod:lst:example_counting}@*) }
  )

  (P increment
    { defined as in listing (*@\ref{exmod:lst:example_counting}@*) }
  )

  (P incrementx
    =goal>
        ISA         goal-chunk
        goal        count
        count       =num1
      - end         =num1
    =retrieval>
        ISA         count-order
        first       =num1
        second      =num2
  ==>
    -goal>
    !output!       (wrong)
  )

  (P stop
    { defined as in listing (*@\ref{exmod:lst:example_counting}@*) }
  )

  (goal-focus second-goal)

  (spp increment :u 8 incrementx :u 0)
  (spp stop :reward 15)
)
\end{lstlisting}
\flushbottom

The subsymbolic layer is turned on and chunk \lstinline|d1| which encodes a false count fact is added. Additionally, there are two training rules, which just retrieve the fact \lstinline|d| to increase its activity and then reset the state to the initial state of the model in listing~\ref{exmod:lst:example_counting}. The goal is set to a chunk, which leads the first training rule to match at first. Furthermore, a broken rule which matches the same context as the original \lstinline|increment| rule is added. The initial utility of the correct \lstinline|increment| rule is set to 8, whereas the corrupted rule gets an initial value of 0. Additionally, the reward the rule \lstinline|stop| can distribute, is set to 15, so the reaching of the final state is rewarded and all rules which lead to that state get a certain amount of this reward. The full code can be found in appendix~\ref{app:ex:subsymbolic_layer}. 

When executing the resulting CHR model, this yields the following output (times have been cut after four digits):

\raggedbottom
\begin{lstlisting}
?- run.
0      ... calling event: do_conflict_resolution
           going to apply rule train1
0.05   ... calling event: apply_rule(train1)
           firing rule train1
0.05   ... calling event: do_buffer_change(goal,chunk(_G26126,_G26127,[ (goal,training2)]))
0.05   ... calling event: start_request(retrieval,chunk(_G26573,count-order,[ (first,3), (second,4)]))
           Started buffer request retrieval
           clear buffer retrieval:nil
0.05   ... calling event: do_conflict_resolution
           No rule matches -> Schedule next conflict resolution event
0.0974 ... calling event: do_buffer_request(retrieval,chunk(_G26573,count-order,[ (first,3), (second,4)]))
           performing request: retrieval
           Retrieved chunk d
           Put chunk d into buffer
0.0974 ... calling event: do_conflict_resolution
           going to apply rule train2
0.1474 ... calling event: apply_rule(train2)
           firing rule train2
0.1474 ... calling event: do_buffer_change(goal,chunk(_G44291,goal-chunk,[ (start,2), (end,4), (goal,count)]))
0.1474 ... calling event: do_buffer_clear(retrieval)
           clear buffer retrieval:d
0.1474 ... calling event: do_conflict_resolution
           going to apply rule start
0.1974 ... calling event: apply_rule(start)
           firing rule start
0.1974 ... calling event: do_buffer_change(goal,chunk(_G26238,_G26239,[ (count,2)]))
0.1974 ... calling event: start_request(retrieval,chunk(_G26683,count-order,[ (first,2)]))
           Started buffer request retrieval
           clear buffer retrieval:nil
0.1974 ... calling event: do_conflict_resolution
           No rule matches -> Schedule next conflict resolution event
0.2575 ... calling event: do_buffer_request(retrieval,chunk(_G26683,count-order,[ (first,2)]))
           performing request: retrieval
           Retrieved chunk c
           Put chunk c into buffer
0.2575 ... calling event: do_conflict_resolution
           going to apply rule increment
0.3075 ... calling event: apply_rule(increment)
           firing rule increment
output:2
0.3075 ... calling event: do_buffer_change(goal,chunk(_G44015,_G44016,[ (count,3)]))
0.3075 ... calling event: start_request(retrieval,chunk(_G44460,count-order,[ (first,3)]))
           Started buffer request retrieval
           clear buffer retrieval:c
0.3075 ... calling event: do_conflict_resolution
           No rule matches -> Schedule next conflict resolution event
0.3657 ... calling event: do_buffer_request(retrieval,chunk(_G44460,count-order,[ (first,3)]))
           performing request: retrieval
           Retrieved chunk d
           Put chunk d into buffer
0.3657 ... calling event: do_conflict_resolution
           going to apply rule increment
0.4157 ... calling event: apply_rule(increment)
           firing rule increment
output:3
0.4157 ... calling event: do_buffer_change(goal,chunk(_G68222,_G68223,[ (count,4)]))
0.4157 ... calling event: start_request(retrieval,chunk(_G68667,count-order,[ (first,4)]))
           Started buffer request retrieval
           clear buffer retrieval:d
0.4157 ... calling event: do_conflict_resolution
           going to apply rule stop
0.4657 ... calling event: apply_rule(stop)
           firing rule stop
           triggered reward for rule: stop
           triggered reward for rule: increment
           triggered reward for rule: increment
           triggered reward for rule: start
           triggered reward for rule: train2
           triggered reward for rule: train1
           reward triggered by rule stop
output:4
0.4657 ... calling event: do_buffer_clear(goal)
           clear buffer goal:first-goal
0.4657 ... calling event: do_conflict_resolution
           No rule matches -> Schedule next conflict resolution event
0.5029 ... calling event: do_buffer_request(retrieval,chunk(_G36175,count-order,[ (first,4)]))
           performing request: retrieval
           Retrieved chunk e
           Put chunk e into buffer
0.5029 ... calling event: do_conflict_resolution
           No rule matches -> Schedule next conflict resolution event
           No more events in queue. End of computation.
\end{lstlisting}
\flushbottom

Due to the training of chunk~\lstinline|d| which increased the activation of this chunk according to the base-level learning equation, the other matching chunk \lstinline|d1| is not being retrieved in the computational process.

Furthermore, the rule \lstinline|increment| is applied instead of \lstinline|incrementx|, although both are matching the context, due to the higher intitial utility value of \lstinline|increment|. The reward of the rule stop leads to the following utilities at the end of the computation (taken from the final CHR store):

\begin{lstlisting}
production_utility(train1,6.916852474603892)
production_utility(train2,6.9363461811027785)
production_utility(start,6.946346181102779)
production_utility(increment,10.480367881320248)
production_utility(stop,7.0)
production_utility(incrementx,0)
\end{lstlisting}

All rules except from \lstinline|increment| and \lstinline|incrementx| have been initialized with an utility value of 5. The two rules \lstinline|start|, \lstinline|increment|, \lstinline|stop| and the training productions have a higher utility value than in the beginning, whereas the other rules have their initial values. This is the case because only the rules \lstinline|start| and \lstinline|increment| have lead to the final state and therefore have received the reward distributed by the \lstinline|stop| rule.

Note that the times may vary from the execution of the same model in the vanilla implementation, since some constants are set differently.

%\section{The Addition Model}

\section{Modeling a Taxonomy of Animals and Their Properties}

In this example, a taxonomy of categories and properties, illustrated in figure~\ref{fig:semantic_model_example}, is modeled \cite[unit 1, pp. 24\psqq]{actr_tutorial}. The goal chunks of this model represent queries like ``Is a canary a bird?'' or ``Is a shark dangerous?''.

\begin{figure}[htb]
\centering
\begin{tikzpicture}[level distance=2cm]
\tikzstyle{every node}=[rectangle,text centered, text height=1.5ex, text depth=0.25ex]
\tikzstyle{category} = [draw];
\node[category] {animal}[sibling distance=3cm]
[grow=down]
child {node[category] {fish}[sibling distance=1.5cm]
  child {node[category] {shark}
     child[level distance=1cm, sibling distance=1.7cm] {node {dangerous}}
     child[level distance=1cm, sibling distance=1.7cm] {node {swims}}}
child[grow=left, level distance=1cm] {
     child {node {gills}}
     child {node {swims}} 
  }
  child {node[category] {salmon}
     child[level distance=1cm, sibling distance=1.5cm] {node {edible}}
     child[level distance=1cm, sibling distance=1.5cm] {node {swims}}
}
}
child[grow=right, sibling distance=1cm, level distance=1cm] {
child {node {moves}}
child {node {skin}}
}
child{node[category] {bird}[sibling distance=1.5cm]
  child {node[category] {canary}
     child[level distance=1cm, sibling distance=1.5cm] {node {yellow}}
     child[level distance=1cm, sibling distance=1.5cm] {node {sings}}
   }
  child[grow=right, level distance=1cm] {
    child {node {wings}}
    child {node {flies}}
  }
  child[level distance=2cm] {node[category] {ostrich}
      child[level distance=1cm, sibling distance=1.5cm] {node {can't fly}}
     child[level distance=1cm, sibling distance=1.5cm] {node {tall}}
  }
};
\end{tikzpicture}
\caption{An example taxonomy of categories and properties. The categories are marked by rectangles, the pure text nodes in the tree are properties. For example, a \emph{shark} is member of the category \emph{fish} and has the direct properties \emph{dangerous} and \emph{swims} and inherits the properties \emph{gills}, \emph{swims}, \emph{moves} and \emph{skin} from its parent categories.}
\label{fig:semantic_model_example}
\end{figure}

Chunks, which encode a property of an object have the form as illustrated in figure~\ref{fig:semantic_model_example:chunks:property}: Each property chunk has a slot \emph{object} which encodes the name of the object. The slot \emph{attribute} holds the name of the attribute like for example \emph{dangerous} or \emph{locomotion}. In the \emph{value} slot, the value of the attribute is defined. The value of the attribute \emph{locomotion} is \emph{swimming} in this example. The chunk in figure~\ref{fig:semantic_model_example:chunks:category} encodes the membership of the object \emph{shark} in the category \emph{fish}. 

\begin{figure}[htb]
\centering
\tikzstyle{chunk} = [ellipse, draw]
\tikzstyle{element} = [] 
\tikzstyle{slot} = [draw, -latex]   
\subfigure[]{
\begin{tikzpicture}[node distance = 2cm, auto]
 \node[chunk] (p1) {\parbox{1.5cm}{\centering p1:\\ property}}; 
 \node[element, left of=p1, node distance=3cm] (object) {shark}; 
 \node[element, above of=p1] (attribute) {dangerous}; 
 \node[element, below of=p1, node distance=2cm] (value) {true}; 

    % Draw edges
  \path[slot] (p1) -- node[above] {object} (object);
  \path[slot] (p1) -- node {attribute} (attribute);
  \path[slot] (p1) -- node {value} (value);
\end{tikzpicture}
\label{fig:semantic_model_example:chunks:property}
}
\qquad
\subfigure[]{
\begin{tikzpicture}[node distance = 2cm, auto]
 \node[chunk] (p3) {\parbox{1.5cm}{\centering p3:\\ property}}; 
 \node[element, left of=p3, node distance=3cm] (object) {shark}; 
 \node[element, above of=p3] (attribute) {category}; 
 \node[element, below of=p3, node distance=2cm] (value) {fish}; 

    % Draw edges
  \path[slot] (p3) -- node[above] {object} (object);
  \path[slot] (p3) -- node {attribute} (attribute);
  \path[slot] (p3) -- node {value} (value);
\end{tikzpicture}
\label{fig:semantic_model_example:chunks:category}
}
\caption{Two chunks of the type \emph{property} encoding properties of the object \emph{shark} as defined in figure~\ref{fig:semantic_model_example}. \subref{fig:semantic_model_example:chunks:property} This chunk encodes the fact, that a shark is dangerous. \subref{fig:semantic_model_example:chunks:category} The membership of an object in a category is also encoded by a \emph{property} chunk. The attribute for this category membership property is \emph{category} and the value encodes the name of the \emph{category}.}
\label{fig:semantic_model_example:chunks}
\end{figure}


To answer a query like ``Is a canary an animal?'' which cannot be answered directly from a \emph{property} chunk, the model must support a recursive search in the tree. The full model code and translation can be found in appendix~\ref{app:ex:semantic_model}. The model uses the concept of duplicate slot tests as described in section~\ref{implementation:duplicate_slot_tests} (see page~\pageref{implementation:duplicate_slot_tests}). However, the compiler cannot handle those tests in the current version and hence the compiled model has to be edited using the pattern from section~\ref{implementation:duplicate_slot_tests}:

\begin{lstlisting}
=retrieval>
      ISA         property
      object      =obj1
      attribute   category
      value       =obj2
    - value       =cat 
\end{lstlisting}

has to be translated to:

\begin{lstlisting}
chain-category@
  buffer(goal,_,A),
    chunk(A,is-member),
    chunk_has_slot(A,object,C),
    chunk_has_slot(A,category,D),
    chunk_has_slot(A,judgment,pending),
  buffer(retrieval,_,B),
    chunk(B,property),
    chunk_has_slot(B,object,C),
    chunk_has_slot(B,attribute,category),
    chunk_has_slot(B,value,E)
  \ apply_rule(chain-category) 
<=> 
  C\==nil,
  D\==nil,
  E\==nil,
  E\==D |
  ...
\end{lstlisting}

I.e., the duplicate slot test for the slot \lstinline|value| is reduced to one single head constraint (\lstinline|chunk_has_slot(B,value,E)|) and a guard, which states that \lstinline|=obj2| $\neq$ \lstinline|=cat| (the built-in check \lstinline|E \== D|). After this small change, the model can be run with various queries, i.e. goal chunks as described in \cite[unit 1, pp. 24\psqq]{actr_tutorial} and appendix~\ref{app:ex:semantic_model}.

%\section{The Fan Model}

%simplified version of fan model
\chapter{Conclusion}

easy translation
ACT-R avoids some of the problems production rule systems have by grounding everything.
problem: scheduling/cycle a little bit complicated



\section{Inventory: What does already work?}

\section{Future Work}

other modules
compiler as described in section...
experiment environment -> link to the REST interface by ...
adapt timings, set some constants to the correct values



% hier weitere Kapitel einbinden


%%
%% Anhänge
%%
\appendix
\chapter{CD Content}
\label{appendix:cd_content}

\begin{description}
 \item[Thesis] The thesis can be found in the folder \lstinline|thesis|. It contains the \LaTeX{} source files. The base document is \lstinline|thesis.tex|, the individual chapters can be found in \lstinline|chapters|. To compile the files, a lot of packages are needed. For example, the graphics are produced with Ti\emph{k}Z, so make sure, all necessary packages are installed on your system.
 \item[CHR-ACT-R] The source of the implementation presented in this work can be found in the directory \lstinline|CHR-ACT-R/src|. It is separated into two parts: the compiler and the framework.
 \begin{itemize}
  \item The compiler is in the directory \lstinline|compiler|. It can be started by consulting \lstinline|actr2chr.pl| in SWI-Prolog. The query \lstinline|compile_file(f).| compiles the file \lstinline|f|. There are several example model files in the directory of the compiler which all start with the prefix \lstinline|example_|.
  \item The framework can be found in the directory \lstinline|core|. To load a model, the compiled model file has to be consulted in SWI-Prolog. There are several compiled example models all starting with the prefix \lstinline|example_|. Make sure that the models are in the same folder as the framework. The query \lstinline|run.| runs the model.
 \end{itemize}
\end{description}

%\chapter{Quelltexte}

In diesem Anhang sind einige wichtige Quelltexte aufgeführt.

\begin{lstlisting}
:- use_module(library(chr)).

a(X) <=> check(X) | b.

check(13).
check(X) :-
  X <10.
\end{lstlisting}

%\chapter{Grammar for Production Rules}

The complete grammar for production rules as defined in \cite{actr_reference}:

\begin{lstlisting}
production-definition ::= p-name {doc-string} condition* ==> action*
p-name ::= a symbol that serves as the name of the production for reference
doc-string ::= a string which can be used to document the production
condition ::= [ buffer-test | query | eval | binding | multiple-value-binding]
action ::= [buffer-modification | request | buffer-clearing | modification-request | buffer-overwrite | eval |
binding | multiple-value-binding | output | !stop!]
buffer-test ::= =buffer-name> isa chunk-type slot-test*
buffer-name ::= a symbol which is the name of a buffer
chunk-type ::= a symbol which is the name of a chunk-type in the model
slot-test ::= {slot-modifier} slot-name slot-value
slot-modifier ::= [= | - | < | > | <= | >=]
slot-name ::= a symbol which names a possible slot in the specified chunk-type
slot-value ::= a variable or any Lisp value
query ::= ?buffer-name> query-test*
query-test ::= {-} queried-item query-value
queried-item ::= a symbol which names a valid query for the specified buffer
query-value ::= a bound-variable or any Lisp value
buffer-modification ::= =buffer-name> slot-value-pair*
slot-value-pair ::= slot-name bound-slot-value
bound-slot-value ::= a bound variable or any Lisp value
request ::= +buffer-name> [direct-value | isa chunk-type request-spec*]
request-spec ::= {slot-modifier} [slot-name | request-parameter] slot-value
request-parameter ::= a Lisp keyword naming a request parameter provided by the buffer specified
direct-value ::= a variable or Lisp symbol
buffer-clearing ::= -buffer-name>
modification-request ::= +buffer-name> slot-value-pair*
buffer-overwrite ::= =buffer-name> direct-value
variable ::= a symbol which starts with the character =
eval ::= [!eval! | !safe-eval!] form
binding ::= [!bind! | !safe-bind!] variable form
+
multiple-value-binding ::= !mv-bind! (variable ) form
output ::= !output! [ output-value | ( format-string format-args*) | (output-value*)]
output-value ::= any Lisp value or a bound-variable
format-string ::= a Lisp string which may contain format specific parameter processing character
format-args ::= any Lisp values, including bound-variables, which will be processed by the preceding
format-string
bound-variable
 ::= a variable which is used in the buffer-test conditions of the production (including a
variable which names the buffer that is tested in a buffer-test or dynamic-buffer-test) or is bound with
an explicit binding in the production
form ::= a valid Lisp form
\end{lstlisting}
\chapter{Executable Examples}

This appendix provides some of the examples that appeared in the work with a minimal environment that represents the current context where the examples appeared.

\section{Rule Order}
\label{app:ex:rule_order}

\begin{lstlisting}
:- use_module(library(chr)).

:- chr_type chunk_def ---> nil; chunk(any, any, slot_list).
:- chr_type list(T) ---> []; [T | list(T)].
:- chr_type slot_list == list(pair(any,any)). % a list of slot-value pairs
:- chr_type pair(T1,T2) ---> (T1,T2).

:- chr_type lchunk_defs == list(chunk_def).

:- chr_constraint buffer/2, buffer_change/2, alter_slots/2, alter_slot/3, chunk/2, chunk_has_slot/3,fire.

% Handle buffer_change
buffer(BufName, OldChunk) \ buffer_change(BufName, chunk(_,_,SVs)) <=>
  alter_slots(OldChunk,SVs).

  
alter_slots(_,[]) <=> true.
alter_slots(Chunk,[(S,V)|SVs]) <=> 
  alter_slot(Chunk,S,V),
  alter_slots(Chunk,SVs).
  
alter_slot(Chunk,Slot,Value), chunk_has_slot(Chunk,Slot,_) <=>
  chunk_has_slot(Chunk,Slot,Value).
  
alter_slot(Chunk,Slot,Value) <=>
  false. % since every chunk must be described completely, Slot cannot be a slot of the type of Chunk
  %chunk_has_slot(Chunk,Slot,Value).  

% first example without fire: 
  
 buffer(b1,C),
   chunk(C,foo),
   chunk_has_slot(C,s1,v1)
 ==>
 buffer_change(b1,chunk(_,_,[(s1,v2)])),
 buffer_change(b2,chunk(_,_,[(s,x)])).
 
 buffer(b1,C),
   chunk(C,foo),
   chunk_has_slot(C,s1,v2)
 ==>
 buffer_change(b2,chunk(_,_,[(s,y)])),
 buffer_change(b1,chunk(_,_,[(s1,v3)])).

% example with fire (uncomment it and add comments to the rules above) 
 
% buffer(b1,C),
%   chunk(C,foo),
%   chunk_has_slot(C,s1,v1)
%   \ fire
% <=>
% buffer_change(b1,chunk(_,_,[(s1,v2)])),
% buffer_change(b2,chunk(_,_,[(s,x)])),
% fire.
% 
% buffer(b1,C),
%   chunk(C,foo),
%   chunk_has_slot(C,s1,v2)
%   \ fire
% <=>
% buffer_change(b2,chunk(_,_,[(s,y)])),
% buffer_change(b1,chunk(_,_,[(s1,v3)])),
% fire.
\end{lstlisting}

\section{Subsymbolic Layer}
\label{app:ex:subsymbolic_layer}



%%
%% Nachspann
%%
\backmatter


%%
%% Literaturverzeichnis
%%
%\bibliography{bibliography}
\begin{flushleft} % kein Blocksatz in Literaturverzeichnis
\printbibliography 
\end{flushleft}

%%
%% Versicherung
%%
\cleardoublepage
\thispagestyle{empty}

Name: \fullname \hfill Matrikelnummer: \matnr \vspace{2cm}

\minisec{Erklärung}

Ich erkläre, dass ich die Arbeit selbständig verfasst und keine anderen als die angegebenen Quellen und Hilfsmittel verwendet habe.\vspace{2cm}

Ulm, den \dotfill

\hfill {\footnotesize \fullname}
\end{document}
