\chapter{Constraint Handling Rules}

This chapter only gives a very brief introduction to Constraint Handling Rules. A much more comprehensive introduction can be found in the book \citetitle{fru_chr_book_2009} by \citeauthor{fru_chr_book_2009} \cite{fru_chr_book_2009}, which gives a very detailed definition of the exact semantics of the language and analysis methods as well as practical examples. Additionally, there are a lot of tutorials, slides and exercises that can be found on \citetitle{chr_homepage} \cite{chr_homepage}. This little introduction is based on \cite{fru_chr_book_2009}.

Constraint Handling Rules (CHR) is a high-level declarative programming language originally designed to write constraint-solvers \cite[2]{chr_survey_tplp10}. CHR programs are defined by a sequence of rules, that operate on a \emph{constraint store}, which is a multiset of constraints (built from predicate symbols and arguments). I.e., a constraint \lstinline|c/n| with arity \lstinline|n| is written as \lstinline[mathescape]|c($t_1, \dots, t_n$)|, where \lstinline|c| is a predicate symbol and the $t_i$ -- the arguments of the constraint -- are logical terms. Constraints in the store can be \emph{ground}, i.e. they do not contain any variables, but they also may contain unbound variables. CHR is usually embedded in a \emph{host language}, which provides so-called \emph{built-in} constraints. Such built-in constraints are constraints that are implemented in the host language. In this work, Prolog is assumed to be the host language. There are three types of rules:

\begin{lstlisting}[mathescape,escapeinside={(*@}{@*)},caption={CHR rule types}, label=lst:chr_rule_types]
<<<simplification>>> @ <<H$_\mathtt{r}$>> <=> <<<G>>> | B. (*@\label{lst:chr_rule_types:simplification}@*)
<<<propagation>>> @ <<H$_\mathtt{k}$>> ==> <<<G>>> | B. (*@\label{lst:chr_rule_types:propagation}@*)
<<<simpagation>>> @ <<H$_\mathtt{k}$ \ H$_\mathtt{r}$>> <=> <<<G>>> | B. (*@\label{lst:chr_rule_types:simpagation}@*)
\end{lstlisting}

Rules consist of a \emph{head}~\lstinline[mathescape]|H$_\mathtt{*}$|, which is a conjunction of at least one constraint, an optional \emph{guard}~\lstinline|G|, which is a conjunction of built-in constraints, and a \emph{body}~\lstinline|B|, which is a conjunction of both built-in and CHR constraints. Optionally, the rules can be given names followed by the symbol \lstinline|@|.

At the beginning of a program, the user provides an initial constraint store, which is also called \emph{query} or \emph{goal} and the CHR program then begins to operate on this initial store by applying rules.

A rule can be applied on the constraint store, if there are constraints in the store that match the head of the rule, i.e. the constraints in the store are ``an instance of the head, [\dots so] the head serves as pattern'' \cite[11]{fru_chr_book_2009}. Note, that in the matching process, no variables of the query are bound, since CHR is a committed-choice language, so rule applications cannot be made undone by backtracking. If matching constraints have been found, the guard is checked and if it holds, the rule is applied. The result of the rule application is dependent on the type of the rule:

\begin{description}
 \item[Simplification rules (as in line~\ref{lst:chr_rule_types:simplification})] remove the constraints which match the head \lstinline[mathescape]|H$_\mathtt{r}$| (the \emph{``heads removed''}) from the store and add the constraints in the body \lstinline|B| to the store.
 \item[Propagation rules (as in line~\ref{lst:chr_rule_types:propagation})] add the constraints in the body \lstinline|B| to the store and keep the matching constraints from the head \lstinline[mathescape]|H$_\mathtt{k}$| (the \emph{``heads kept}'').
 \item[Simpagation rules (as in line~\ref{lst:chr_rule_types:simpagation})] are a mix of the first two rule types: They add the constraints in the body \lstinline|B| to the store, keep the matching constraints from the head \lstinline[mathescape]|H$_\mathtt{k}$| and remove the constraints from the head \lstinline[mathescape]|H$_\mathtt{r}$|. In general, simplification and propagation rules can be expressed by simpagation rules where one side of the head is empty, i.e. only contains the constraint \lstinline|true|.
\end{description}

\begin{example}[Minimum]
\label{ex:minimum}

This simple example has been taken from \cite[19\psqq]{fru_chr_book_2009} and describes a CHR program that computes the minimum among the numbers $n_i$ given as multiset of numbers in the query \lstinline[mathescape]|min($n_1$), ..., min($n_k$)|. The \lstinline|min/1| constraint is interpeted such that its containing number is a minimum candidate. The minimum computation is achieved by specifying one simpagation rule:

\begin{lstlisting}[caption={Minimum program}]
min(N) \ min(M) <=> N=<M | true.
\end{lstlisting}

For every pair of numbers in the store, the rule removes the greater one and keeps the smaller one. If this rule is applied to exhaustion, only one constraint is left -- the minimum. Then the rule is not applicable any more, since it lacks a partner constraint for the single \lstinline|min| constraint to match the head.

Note, that in pure CHR no assumptions on the order of the application of the rules and the involved constraints are made. One possible derivation of the result could be (the constraints the rule is applied to are underlined in each step):

\begin{lstlisting}[escapeinside={(*@}{@*)},caption={Sample derivation of the minimum example},label=lst:sample_derivation_min]
(*@\underline{min(3)}@*), (*@\underline{min(1)}@*), min(4), min(0), min(2)
(*@\underline{min(1)}@*), (*@\underline{min(4)}@*), min(0), min(2)
(*@\underline{min(1)}@*), (*@\underline{min(0)}@*), min(2)
(*@\underline{min(0)}@*), (*@\underline{min(2)}@*)
min(0)
\end{lstlisting}
\end{example}

\begin{example}[Matching]
This example tries to clarify the matching process and is taken from \cite[12]{fru_chr_book_2009}. The following rules are added into individual, separated programs and some queries are tested in those programs.

\begin{lstlisting}
p(a) <=> true.
p(X) <=> true.
p(X) <=> X=a.
p(X) <=> X=a | true.
p(X) <=> X==a | true.
\end{lstlisting}

The query \lstinline|?- p(a)| matches all the rules, so the result will always be an empty constraint store, indicated by \lstinline|true|.

For the query \lstinline|?- p(b)| the first rule does not match and \lstinline|p(b)| is added to the constraint store. The second rule is applicable, since \lstinline|p(X)| is a pattern for \lstinline|p(a)|. For the third rule, the result of the computation is \lstinline|false|, because the rule is applicable, but the unification \lstinline|a=b| in the body will fail. Since CHR is a committed-choice language, the rule selection will not be made undone. The last two rules are not applicable, since the guard fails.

The query \lstinline|?- p(Y)| does not match the first rule, since \lstinline|p(a)| is not a pattern \lstinline|p(Y)| -- the goal is more general than the head of the rule and no bindings of the constraints in the goal will be performed for the matching. The second rule does match, because the \lstinline|X| from the head of the rule can be bound to \lstinline|Y| from the goal. The result of the application of the third rule is \lstinline|Y=a|, because the rule does match and leads to a binding of \lstinline|Y| to \lstinline|a| in the body. The last two rules do not fire, because their guards fail. If \lstinline|Y| has been bound to \lstinline|a| explicitly before (e.g. by another rule), the rule could fire.
\end{example}


There are various definitions of the operational semantics -- i.e. the behaviour of CHR programs -- for different purposes and there may come more in the future \cite[11]{fru_chr_book_2009}. In this work, the implementation from the K.U. Leuven as shipped with SWI-Prolog is used\cite{swi_prolog}. This version implements the so-called \emph{refined operational semantics}. Constraints in the goal are processed from left to right. When entering the constraint store, a CHR constraint becomes \emph{active}, i.e. each rule, which contains the active constraint in the head, is checked for applicability. The order of the rule applicability checks is from top to bottom. The first rule that matches is fired. If the active constraint is removed by the rule application, the next constraint from the body will be added and become active. Otherwise, if the active constraint is kept, the next rule below the first matching rule is checked for applicability, etc. If no rule matches, the constraint becomes passive and is actually put to the constraint store, where it waits to become the partner constraint in a matching rule and the next constraint from the query is added. A passive constraint becomes active again, if its containing variables are bound. 

If a rule fires, its body is processed from left to right and behaves like a procedure call: It will add its constraints one after another (and they will become active sequentially). When all constraints from the body have been added and no constraint is active anymore, the next constraint from the query is added. If no constraints are left in the query, the program terminates and the final constraint store is printed.

In the implementation of ACT-R, it was necessary at some points to rely on the order of rule applications from top to bottom. This is very common, for instance, if in an advanced version of the minimum computation in example~\ref{ex:minimum} the process should be triggered by a \lstinline|get_min/1| constraint that also gets the result bound to its argument. I.e., after the complete computation of the query in listing~\ref{lst:sample_derivation_min} triggered by \lstinline|get_min(Min)|, the variable \lstinline|Min| should be bound to \lstinline|Min=0|, the minimal number in the query. This can be achieved as follows:

\begin{lstlisting}[caption={Minimum program with trigger}]
get_min(_), min(N) \ min(M) <=> N=<M | true.
get_min(Min), min(N) <=> Min=N.
\end{lstlisting}

Obviously, the result of this program is dependent on the rule application order: The second rule is applicable as soon as the minimum computation has been triggered by \lstinline|get_min| and one single minimum candidate \lstinline|min(N)|. After the application of the second rule, the first one will not be applicable anymore, because the \lstinline|get_min/1| constraint is removed from the store. Hence, it must be ensured, that this rule is not applied before the first rule has not been applied to exhaustion. This is achieved by the refined operational semantics. The program computes the minimum of all numbers that are in the store at the time the \lstinline|get_min/1| trigger constraint is introduced. If there are added new minimum candidates afterwards, they will not be respected in this particular minimum computation. However, if later on a new minimum computation is triggered, all those minimum constraints will be part of the computation.


