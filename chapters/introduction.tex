\chapter{Introduction}

\emph{Computational psychology} or \emph{computational cognitive modeling} is a research field in the cognitive sciences. Among the other approaches -- mathematical and verbal-conceptual modeling, ``computational modeling appears to be the most promising approach in many ways and offers the flexibility and the expressive power that no other approaches can match'' \cite[vii]{sun_introduction_2008}. It explores human cognition by implementing detailed computational models that enable computers to execute them and simulate human behaviour \cite[3]{sun_introduction_2008}. By conducting the same experiments with humans and with simulations of the suggested underlying cognitive models, the plausibility of models can be checked and models can be improved gradually. This approach is illustrated in figure~\ref{fig:cognitive_modeling_exp}. Furthermore, an important benefit of computational models is that they are -- as a matter of principle -- very detailed and have a very clear semantics, in order that they can be executed by a computer. 

However, psychology is experiencing a movement towards specialization \cite{anderson_integrated_2004}, i.e. there are a lot of independent, highly specialized fields that lack a more global view, which impedes cognitive modeling where a very detailed and complete view is necessary for execution:

\begin{quote}
``In 1972, [\dots it] seemed to me, also, that the cognitive revolution was already well in hand and established. Yet, I found myself concerned about the theory psychology was developing. [\dots] I tried to make that point by noting that what psychologists mostly did for theory was to go from dichotomy to dichotomy.'' \cite[1\psq]{newell_unified_1990}
\end{quote}

\citeauthor{newell_unified_1990} suggested in his book from \citeyear{newell_unified_1990} \cite{newell_unified_1990}, that for developing consistent models of cognition it is necessary to create a theory that tries to put all those highly specialized components together \cite[1036]{anderson_integrated_2004}. He therefore introduced the term \emph{cognitive architecture} \cite[5]{anderson_how_2007}, which today can be defined as ``a specification of the structure of the brain at a level of abstraction that explains how it achieves the function of the mind'' \cite[7]{anderson_how_2007}. A cognitive architecture provides the ability to create models for specific cognitive tasks \cite[29]{taatgen_modeling_2006} by offering ``representational formats together with reasoning and learning mechanisms to facilitate modeling'' \cite[29]{taatgen_modeling_2006}. Nevertheless, a cognitive architecture should also constrain modeling -- ideally it should only allow ``cognitive models that are cognitively plausible'' \cite[29]{taatgen_modeling_2006}.

Adaptive Control of Thought-Rational  (ACT-R) is a cognitive architecture, that ``is capable of interacting with the outside world, has been mapped onto brain structures, and is able to learn to interact with complex dynamic tasks'' \cite[29]{taatgen_modeling_2006}, so its theory is well-investigated. It also is one of the most popular cognitive architectures in the field \cite{rutledge2005can} and provides an implementation that allows modelers to execute their models by a computer and hence offers a platform for computational cognitive modeling as defined above.

\section{Motivation and Goal}

This work presents an implementation of ACT-R using Constraint Handling Rules. Constraint Handling Rules (CHR) is a high-level rule-based programming language with several very well-defined operational and declarative semantics \cite[49\psqq]{fru_chr_book_2009}, so the meaning of a CHR program is logically defined. Additionally, there are a lot of methods to analyze CHR programs \cite[96\psqq]{fru_chr_book_2009} and there are a lot of very nice properties each CHR program has automatically, like the anytime and online property \cite[83\psqq]{fru_chr_book_2009}. The declarativity of CHR programs does not affect the efficiency, so every algorithm that can be implemented efficiently in a imperative language can also be implemented efficiently in CHR and the constant slow-down of CHR compared to a C program is very low using high-optimizing compilers \cite[94]{fru_chr_book_2009}.

Since models are expressed by production rules in ACT-R, the idea to implement such models in CHR seems likely, because, at a first glance, the semantics of a CHR rule does not seem to be very different from an ACT-R production rule, so a lot of work like the \emph{efficient} implementation of the matching process can be saved. Due to the clear semantics and the several analysis methods, there is a hope that through the translation of the ACT-R theory to CHR there will be a benefit to the analysis of models and the inspection of the logical implications of a cognitive model.

The aim of this work is to implement the basic concepts of ACT-R and stick as closely as possible to the theory using the original implementation as a reference. Hence, a goal is to find translation schemes of ACT-R production rules to CHR rules and the implementation of the underlying cognitive architecture in CHR.

\section{Related Work}

There are several implementations of the ACT-R theory in different languages. First of all, there is the official ACT-R implementation in Lisp \cite{actr_homepage} often referred to as the \emph{vanilla} implementation. There are a lot of extensions to this implementation, which partly have been included to the original package in later versions like the ACT-R/PM extension that has been included in ACT-R 6.0 \cite[264]{actr_reference}. The implementation comes with an experiment environment which offers a graphical user interface to load, execute and observe models which communicates through a network interface with the ACT-R core.

\citeauthor{stewart_deconstructing_2007} have built an implementation in Python, which also had the aim to simplify and harmonize parts of the ACT-R theory by finding the central components of the theory \cites{stewart_deconstructing_2006,stewart_deconstructing_2007}. The authors describe another approach of implementing ACT-R without sticking too much to the classic implementations. For instance, the architecture has been reduced two only two components (the procedural and the declarative module which will be described in section~\ref{actr_description}) and build the rest of the architecture by using those two modules and combining them in different ways. Additionally, there is no possibility to translate traditional ACT-R models automatically to Python code. Hence, the focus of the work was different from the aims in this work.

Furthermore, there are two different implementations in Java: \emph{jACT-R} \cite{jactr} and \emph{ACT-R: The Java Simulation \& Development Environment} \cite{java_actr}. The latter one is capable of executing original ACT-R models and offers an advanced graphical user interface. The focus of the project was to make ACT-R more portable with the help of Java, since Lisp's ``extensibility for different task implementations and different hardware platforms has been lagging compared to more modern languages'' \cite{java_actr_benefits}. In jACT-R, the focus was to offer a clean and exchangeable interface to all the components, so different versions of the ACT-R theory can be mixed \cite{jactr_benefits} and models are defined using XML. Due to the modular design defining various interfaces which can be exchanged, jACT-R is highly adaptable to personal needs. However, since Java is the host language, there is no expected gain in declarativity and model analysis for both implementations.

There are approaches to implement psychological models using declarative and logic programming languages. In \cite{pereira_modellingmorality_2007} \citeauthor{pereira_modellingmorality_2007} present computational models for cognitive reasoning in the context of moral dilemmas using prospective logic programs. \citeauthor{balduccini_formalization_2010} show in \cite{balduccini_formalization_2010} how psychological knowledge can be formalized and reasoning over this knowledge can be achieved using answer set programming. However, both approaches are detached from a broadly distributed cognitive architecture like ACT-R. Nevertheless, in \cite[726]{balduccini_formalization_2010} it is emphasized, that for psychological ``theories of a more qualitative or logical nature [\dots] are not easy to formalize in'' the way of neural-networks or similar approaches, but need a more abstract approach.

\section{Overview}

This work is divided in several parts. First of all, an important task was the identification of the fundamental parts of the ACT-R theory. Hence, a description of the theory is given in chapter~\ref{actr_description}. Chapter~\ref{chr_introduction} gives a very brief introduction to Constraint Handling Rules. The result of the implementation of ACT-R in CHR is described in chapter~\ref{implementation}, which first formalizes some of the parts of the ACT-R theory before it suggests how to implement the afore described concepts in Constraint Handling Rules. Afterwards, the work of chapter~\ref{implementation} is demonstrated in some example models in section~\ref{example_models}. Chapter~\ref{conclusion} summarizes the work and gives an outlook to future work.
