\chapter*{Abstract}

\emph{Computational Cognitive Modeling} is a research field at the interface of computer science and the cognitive sciences. It enables researchers to build detailed cognitive models upon a cognitive architecture which provides some general assumptions about human cognition to simulate human behaviour. By conducting the same experiments with humans and an executable computational cognitive model, the plausibility of a model can be verified.

\emph{ACT-R} is a cognitive architecture which is widely used in the field of computational cognitive modeling. It is a production-rule system whose models are expressed by a set of declarative knowledge elements -- the data -- and a set of rules. The rules match the data and, if applied, have effects on the data. \emph{Constraint Handling Rules (CHR)} is a high-level rule-based formalism and programming language which offers a lot of analysis tools for its programs. 

This work first presents some fundamental aspects of ACT-R and then introduces a translation of ACT-R models to CHR programs. The implementation of ACT-R using Constraint Handling Rules shows that cognitive models can be expressed very elegantly in CHR. Since CHR provides a lot of analysis tools and its rules have a defined declarative semantics, this approach may support the verification of cognitive models. Additionally, the translation of the models is automated by a compiler, so it is not compulsory for modelers to learn a whole new language and legacy models may be translated to CHR.