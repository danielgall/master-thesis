\chapter{Conclusion}
\label{conclusion}

The goal of this thesis was to investigate the cognitive architecture \emph{ACT-R} to develop a translation scheme and an implementation of its fundamental concepts using Constraint Handling Rules. It has been shown that the fundamental aspects of the system can be implemented very elegantly in CHR and the translation process can be automated. Hence, this work enables modelers to translate their ACT-R models without changing them and they are executable immediately. Furthermore, with the translation of the rules, modelers can directly utilize the various analysis methods of CHR programs and analyze the implications of their models formally. The behaviour of the models and the framework can be adapted easily to individual needs by just changing some rules for example for some special cases of the base-level learning or the initialization. Since the framework consists of a manageable set of elegant rules, even deep changes of the underlying cognitive architecture beyond the default parameters are possible.

Another goal was to facilitate the comparison of rule-based systems. As described before in the description of the procedural module (see chapter~\ref{implementation:procedural_module}), the production rules of ACT-R avoid a lot of common problems of production rule systems like negation-as-absence and local variables. This is due to the -- compared to other rule-based systems -- simple architecture of the basic rules. For instance, the system does not allow to introduce new variables on the right hand side of a rule. Hence, all checks on the left hand side of an ACT-R production rule are simple matchings or comparisons of known values. This leads to very simple guard checks in the CHR translation; a lot of rules can even be reduced to a simple matching problem which is already supported automatically by CHR. The actions of a production rule are very limited, since they only affect the buffer system, i.e. an implementation only has to offer a limited amount of actions which can then be translated very easily.

One of the most complicated aspects of ACT-R implementations is the timing and the scheduling of conflict-resolution and module request events. However, even those concepts can be implemented elegantly using trigger constraints. The conflict resolution of ACT-R can be implemented by automated rule compilation, which produces two CHR rules out of one production rule. I.e., there is no unreasonable blow-up in the number of rules.

Thus, the implementation offers an elegant and adaptable cognitive architecture which sticks very close to the fundamental aspects of the ACT-R theory and reference implementation. It facilitates model analysis and comparison of the ACT-R production rule system to other rule-based formalisms.

\paragraph{Future Work}

This work has presented some of the fundamental concepts of ACT-R and mechanisms of translating them to CHR. However, there are some parts of the theory and common implementations which have not been regarded in this thesis. Simple ACT-R production rules without modified slot tests can be translated correctly. At the moment, the compiler lacks of translation methods for some allowed modifiers. Rules with duplicate slot tests cannot be translated correctly. Section~\ref{implementation:compiler:problems} gives a more detailed overview of the problems of the compiler.

ACT-R furthermore provides an experiment environment which allows the modeler to create graphical frontends which can be used by both ACT-R models and humans. This environment is called \emph{ACT-R GUI Interface (AGI)} \cite{agi_reference} and communication is implemented by a network interface. The perceptual/motor modules usually are used in combination with the AGI and therefore have not been implemented, yet. The implementation is also lacking the imaginal module, since the fundamental concepts can be shown using only the declarative and the goal module.

Additionally, only a subset of the production rule grammar has been implemented, yet, so modification requests, strict harvesting, explicit variable bindings and evaluation functions which may produce side-effects have been ignored in the current implementation and may be part of future versions.

In future work, the concepts of skill acquisition, i.e. the creation of new procedural knowledge, especially the concept of production rule compilation, may be investigated in addition. Furthermore, the possibilities of the analysis of ACT-R rules and the evaluation of experiments may be inspected.

